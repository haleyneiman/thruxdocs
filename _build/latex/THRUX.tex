%% Generated by Sphinx.
\def\sphinxdocclass{report}
\documentclass[letterpaper,10pt,english]{sphinxmanual}
\ifdefined\pdfpxdimen
   \let\sphinxpxdimen\pdfpxdimen\else\newdimen\sphinxpxdimen
\fi \sphinxpxdimen=.75bp\relax

\PassOptionsToPackage{warn}{textcomp}
\usepackage[utf8]{inputenc}
\ifdefined\DeclareUnicodeCharacter
% support both utf8 and utf8x syntaxes
  \ifdefined\DeclareUnicodeCharacterAsOptional
    \def\sphinxDUC#1{\DeclareUnicodeCharacter{"#1}}
  \else
    \let\sphinxDUC\DeclareUnicodeCharacter
  \fi
  \sphinxDUC{00A0}{\nobreakspace}
  \sphinxDUC{2500}{\sphinxunichar{2500}}
  \sphinxDUC{2502}{\sphinxunichar{2502}}
  \sphinxDUC{2514}{\sphinxunichar{2514}}
  \sphinxDUC{251C}{\sphinxunichar{251C}}
  \sphinxDUC{2572}{\textbackslash}
\fi
\usepackage{cmap}
\usepackage[T1]{fontenc}
\usepackage{amsmath,amssymb,amstext}
\usepackage{babel}



\usepackage{times}
\expandafter\ifx\csname T@LGR\endcsname\relax
\else
% LGR was declared as font encoding
  \substitutefont{LGR}{\rmdefault}{cmr}
  \substitutefont{LGR}{\sfdefault}{cmss}
  \substitutefont{LGR}{\ttdefault}{cmtt}
\fi
\expandafter\ifx\csname T@X2\endcsname\relax
  \expandafter\ifx\csname T@T2A\endcsname\relax
  \else
  % T2A was declared as font encoding
    \substitutefont{T2A}{\rmdefault}{cmr}
    \substitutefont{T2A}{\sfdefault}{cmss}
    \substitutefont{T2A}{\ttdefault}{cmtt}
  \fi
\else
% X2 was declared as font encoding
  \substitutefont{X2}{\rmdefault}{cmr}
  \substitutefont{X2}{\sfdefault}{cmss}
  \substitutefont{X2}{\ttdefault}{cmtt}
\fi


\usepackage[Bjarne]{fncychap}
\usepackage{sphinx}

\fvset{fontsize=\small}
\usepackage{geometry}

% Include hyperref last.
\usepackage{hyperref}
% Fix anchor placement for figures with captions.
\usepackage{hypcap}% it must be loaded after hyperref.
% Set up styles of URL: it should be placed after hyperref.
\urlstyle{same}
\addto\captionsenglish{\renewcommand{\contentsname}{Table Of Contents}}

\usepackage{sphinxmessages}
\setcounter{tocdepth}{2}

\setcounter{tocdepth}{8}
   \setcounter{secnumdepth}{5}

\title{THRUX Documentation}
\date{Nov 08, 2019}
\release{1.0.0.72}
\author{KT}
\newcommand{\sphinxlogo}{\vbox{}}
\renewcommand{\releasename}{Release}
\makeindex
\begin{document}

\pagestyle{empty}
\sphinxmaketitle
\pagestyle{plain}
\sphinxtableofcontents
\pagestyle{normal}
\phantomsection\label{\detokenize{docs/index::doc}}



\chapter{\sphinxstylestrong{Introduction}}
\label{\detokenize{docs/introduction/index-thrux:introduction}}\label{\detokenize{docs/introduction/index-thrux::doc}}
In order to deliver the best possible design, or construct the best building, Owners and Architects explore multitudes of design options, while Engineers are tasked with ensuring that each design is safe and stable.  There is an increasing demand for professionals to rapidly respond to design changes while still preserving quality.

Engineering tools are scattered from paper to spreadsheets to analysis software to drafting tools and finally to the deliverable of printed drawings.  The lack of continuity between tools causes inefficiencies, which ultimately is paid by the Owner.

THRUX is a design environment aimed to unite engineering tools of the construction industry to provide consistent, transparent and flexible designs.


\section{\sphinxstylestrong{What is THRUX?}}
\label{\detokenize{docs/introduction/index-thrux:what-is-thrux}}
THRUX is a cloud-based design environment suite that fills the market gap of streamlined building design calculations, by bridging the gap between Engineers/Designers, Architects and Contractors, providing iterative, cost-conscious design assistance.


\section{\sphinxstylestrong{New User?}}
\label{\detokenize{docs/introduction/index-thrux:new-user}}\label{\detokenize{docs/introduction/index-thrux:id1}}
Check out the following resources to help you get started:
\begin{itemize}
\item {} 
Our {\hyperref[\detokenize{docs/productoverview/index-product_overview:product-overview}]{\sphinxcrossref{\DUrole{std,std-ref}{Product Overview}}}} contains a high-level description of our services.

\item {} 
We’ve created a {\hyperref[\detokenize{docs/tutorial/index-tutorial:tutorial-project}]{\sphinxcrossref{\DUrole{std,std-ref}{Tutorial Project}}}} to quickly walk you through how to create a Project.

\item {} 
Our {\hyperref[\detokenize{docs/userguide/index-user_guide:user-guide}]{\sphinxcrossref{\DUrole{std,std-ref}{User Guide}}}} is a complete manual of all features.

\item {} 
See our {\hyperref[\detokenize{docs/faq:frequently-asked-questions}]{\sphinxcrossref{\DUrole{std,std-ref}{FAQ}}}} or our {\hyperref[\detokenize{docs/definitions/index-definitions:definitions}]{\sphinxcrossref{\DUrole{std,std-ref}{Definitions}}}} section for help.

\item {} 
Check out our \sphinxhref{https://www.thrux.io/videos}{Videos} page for a high level overview of our features, and our \sphinxhref{https://www.youtube.com/watch?v=c2Koj-hgpN8\&list=PLw-PkBFPFGnsdR3FGFkuQJsamVq38IB2q}{Tutorial Videos} to help you get started!

\item {} 
Follow us on \sphinxhref{https://www.linkedin.com/company/thrux/?viewAsMember=true}{LinkedIn}, \sphinxhref{https://www.instagram.com/thrux.io/}{Instagram}, and \sphinxhref{https://www.youtube.com/channel/UCkx1kvMvCRu6qVhzf3NJljQ/}{YouTube!}

\end{itemize}


\section{\sphinxstylestrong{Setting Up Your Account, Installation, and Receiving Updates}}
\label{\detokenize{docs/introduction/index-thrux:setting-up-your-account-installation-and-receiving-updates}}\label{\detokenize{docs/introduction/index-thrux:account-portal}}
THRUX is a cloud-based application and automatically updates each time it is opened.

For installation, first visit the Account Portal and create an Account:

\sphinxurl{http://thruxcoreweb.azurewebsites.net/AccountPortal}

\begin{figure}[H]
\centering
\capstart

\noindent\sphinxincludegraphics{{Account_Portal-1}.PNG}
\caption{Account Portal}\label{\detokenize{docs/introduction/index-thrux:id2}}\end{figure}

Then go to the Downloads page for the installation link.


\section{\sphinxstylestrong{Accounts, Projects, Branches, and Subcriptions}}
\label{\detokenize{docs/introduction/index-thrux:accounts-projects-branches-and-subcriptions}}
Projects are created by those who have an Account.  Accounts can be of a Basic, Professional, or Team subscription.


\subsection{THRUX Basic}
\label{\detokenize{docs/introduction/index-thrux:thrux-basic}}\begin{quote}

Basic accounts are able to create an unlimited number of Projects, and have access to the Help Center.
\end{quote}


\subsection{THRUX Professional}
\label{\detokenize{docs/introduction/index-thrux:thrux-professional}}\begin{quote}

Professional accounts include Basic functionality, but also provide the ability to interface with Revit, and access all MEP Trades.
\end{quote}


\subsection{THRUX Team}
\label{\detokenize{docs/introduction/index-thrux:thrux-team}}\begin{quote}

Team subscriptions are geared towards organizations, and allow collaboration between Projects across multiple Teams.
\end{quote}

Once an Account creates a Project, they become the owner of that Project.  An owner has the ability to invite others to join a Project.

Whenever a Project is created, an initial Branch, or Issuance, is created.  Think of the Branch as the base scheme, and it cannot be reassigned.

You can create Branches, or Issuances, off of the base scheme, and can then use analysis tools to compare Branches against the base Branch.


\section{\sphinxstylestrong{Storage and Recovery Options}}
\label{\detokenize{docs/introduction/index-thrux:storage-and-recovery-options}}
THRUX models are stored in the cloud and are periodically backed up.

THRUX utilizes Microsoft Azure Storage to store and replicate models in a highly available, secure, scalable, and geo-redundant manner.  Microsoft handles maintenance for their Azure storage facilities which are encrypted and accessible from anywhere in the world.  You have the ability to revert your Projects to a specific point in time.

For more information, click {\hyperref[\detokenize{docs/userguide/index-user_guide:recovery-options}]{\sphinxcrossref{\DUrole{std,std-ref}{here}}}}.


\section{\sphinxstylestrong{Contact Us/Support}}
\label{\detokenize{docs/introduction/index-thrux:contact-us-support}}\label{\detokenize{docs/introduction/index-thrux:support}}
If you would like to request a demonstration or for further technical assistance, contact us at:
\begin{itemize}
\item {} 
\sphinxhref{mailto:thruxservices@thrux.io}{thruxservices@thrux.io}

\item {} 
212-547-9802

\end{itemize}

Also, visit our website at: \sphinxurl{https://www.thrux.io} and get in touch with our support team.


\chapter{\sphinxstylestrong{Product Overview}}
\label{\detokenize{docs/index:product-overview}}\phantomsection\label{\detokenize{docs/productoverview/index-product_overview:product-overview}}
\sphinxstylestrong{Welcome to THRUX!}

Supplement this Product Overview with our {\hyperref[\detokenize{docs/userguide/index-user_guide:user-guide}]{\sphinxcrossref{\DUrole{std,std-ref}{User Guide}}}} guide and our {\hyperref[\detokenize{docs/faq:frequently-asked-questions}]{\sphinxcrossref{\DUrole{std,std-ref}{FAQ’s}}}} for more information.


\section{\sphinxstylestrong{Code-Compliant Design}}
\label{\detokenize{docs/productoverview/index-product_overview:code-compliant-design}}\label{\detokenize{docs/productoverview/index-product_overview::doc}}
Design at twice the speed as we are constantly validating your model against applicable codes.  Never worry about issuing avoidable code-violations again.  When you create a project, choose the applicable codes that are relevant to your Project (NEC 2008/2014, ASHRAE 90.1, etc.).  These codes are constantly being referenced to perform customizable auto-sizing calculations and validations.

\begin{figure}[H]
\centering
\capstart

\noindent\sphinxincludegraphics{{code-compliance-1}.PNG}
\caption{Schedules Workspace with Design Assistance turned off}\label{\detokenize{docs/productoverview/index-product_overview:id1}}\end{figure}

Also, THRUX provides different levels of {\hyperref[\detokenize{docs/userguide/explorersandutilitytools/statusbar/index-status_bar:design-assistance}]{\sphinxcrossref{\DUrole{std,std-ref}{Design Assistance}}}}, which enable you to fully modify parameters of a circuit while monitoring any violations of safety codes and standards.


\subsection{Flag Tracker}
\label{\detokenize{docs/productoverview/index-product_overview:flag-tracker}}
The {\hyperref[\detokenize{docs/userguide/buildingelectricalmodel/flagtracker/index-flag_tracker:flag-tracker}]{\sphinxcrossref{\DUrole{std,std-ref}{Flag Tracker}}}} is a tool which reports any violations of a Project’s applicable codes and safety standards.

\begin{figure}[H]
\centering
\capstart

\noindent\sphinxincludegraphics{{flagtracker-1}.PNG}
\caption{One-Line showing overload and voltage drop violations of applicable codes and safety standards}\label{\detokenize{docs/productoverview/index-product_overview:id2}}\end{figure}


\subsection{Studies}
\label{\detokenize{docs/productoverview/index-product_overview:studies}}
{\hyperref[\detokenize{docs/userguide/buildingelectricalmodel/studies/index-studies:studies}]{\sphinxcrossref{\DUrole{std,std-ref}{Studies}}}} allows a tabular view of common engineering reports such as Voltage Drop, Short Circuit, and Loading.

\begin{figure}[H]
\centering
\capstart

\noindent\sphinxincludegraphics{{studies-1}.PNG}
\caption{Studies displaying the Voltage Drop report}\label{\detokenize{docs/productoverview/index-product_overview:id3}}\end{figure}


\section{\sphinxstylestrong{Revit Interoperability}}
\label{\detokenize{docs/productoverview/index-product_overview:revit-interoperability}}
Import an Architectural model from Revit into an environment designed with MEP engineering as the focus, modify MEP design to accomodate new layouts, then synchronize MEP changes back to a Revit model.

Or, mock up SD level layouts prior to receiving detailed Revit models.

\begin{figure}[H]
\centering
\capstart

\noindent\sphinxincludegraphics{{revit-interop-1}.PNG}
\caption{Imported Revit model}\label{\detokenize{docs/productoverview/index-product_overview:id4}}\end{figure}


\section{\sphinxstylestrong{Pricing}}
\label{\detokenize{docs/productoverview/index-product_overview:pricing}}
Instantly translate design-changes to construction cost.  Provide an optimized design that fits your client’s budget and timeline.

All modeled equipment is cross referenced with equipment and labor rate catalogues which are fully customizable on a per project basis.

These catalogues can be synchronized with regional averages of other THRUX users.  As all major takeoffs are taken throughout the course of the design, Contractors save dozens of hours doing repetitive counts.


\subsection{Pricing Report}
\label{\detokenize{docs/productoverview/index-product_overview:pricing-report}}
The {\hyperref[\detokenize{docs/userguide/pricingmodel/pricingreport/index-pricing_report:pricing-report}]{\sphinxcrossref{\DUrole{std,std-ref}{Pricing Report}}}} allows you to view a breakout of order of magnitude estimates based on the material and labor rates for the model.

\begin{figure}[H]
\centering
\capstart

\noindent\sphinxincludegraphics{{pricing-report-1}.PNG}
\caption{Pricing Report displaying the net cost of the model}\label{\detokenize{docs/productoverview/index-product_overview:id5}}\end{figure}


\subsection{Price Monitoring}
\label{\detokenize{docs/productoverview/index-product_overview:price-monitoring}}
The {\hyperref[\detokenize{docs/userguide/pricingmodel/pricetracker/index-price_tracker:price-tracker}]{\sphinxcrossref{\DUrole{std,std-ref}{Price Tracker}}}} is a tool which live monitors the price of the model.

\begin{figure}[H]
\centering
\capstart

\noindent\sphinxincludegraphics{{pricetracker-1}.PNG}
\caption{View cost as the location of a main distribution board changes}\label{\detokenize{docs/productoverview/index-product_overview:id6}}\end{figure}


\section{\sphinxstylestrong{Branching}}
\label{\detokenize{docs/productoverview/index-product_overview:branching}}
{\hyperref[\detokenize{docs/userguide/projectmanagement/issuancelog/index-issuance_log:issuance-log}]{\sphinxcrossref{\DUrole{std,std-ref}{Branching}}}} provides the ability for you to study alternatives and identify the best path towards achieving the desired result.  This allows you to effectively manage multiple design schemes with a single unifying tool.  For example, parallel design paths such as exploring copper versus aluminum conductors, or utilizing bus duct versus pipe and wire can be fully studied and optimized.

Project phases can be managed by creating Branches to capture a project state at a specific point in time.

Branches can be compared to provide consolidated change reports to Contractors or Owners.  The ability to track changes between Branches allows your model to survive in the highly dynamic environment of construction and development.

\begin{figure}[H]
\centering
\capstart

\noindent\sphinxincludegraphics{{branching-1}.PNG}
\caption{One-Line and the Issuance Log displaying multiple Branches of the base model}\label{\detokenize{docs/productoverview/index-product_overview:id7}}\end{figure}

For a more in-depth guide, please see our {\hyperref[\detokenize{docs/userguide/index-user_guide:user-guide}]{\sphinxcrossref{\DUrole{std,std-ref}{User Guide}}}} guide.


\chapter{\sphinxstylestrong{Tutorial Project}}
\label{\detokenize{docs/index:tutorial-project}}

\section{\sphinxstylestrong{Creating a New Project}}
\label{\detokenize{docs/tutorial/index-tutorial:creating-a-new-project}}\label{\detokenize{docs/tutorial/index-tutorial:tutorial-project}}\label{\detokenize{docs/tutorial/index-tutorial::doc}}
Sign in with your Account and click on the New Project button (+ sign).

\begin{figure}[H]
\centering
\capstart

\noindent\sphinxincludegraphics{{NewProject}.PNG}
\caption{Homescreen}\label{\detokenize{docs/tutorial/index-tutorial:id2}}\end{figure}

Give your project a name, and click the Start button in the bottom right.

\begin{figure}[H]
\centering
\capstart

\noindent\sphinxincludegraphics{{NewProject-2}.PNG}
\caption{New Project - Default Settings}\label{\detokenize{docs/tutorial/index-tutorial:id3}}\end{figure}


\subsection{First Steps}
\label{\detokenize{docs/tutorial/index-tutorial:first-steps}}
A common question is: Where do I start?

Depending on the information you have, it may be better to start in one area than another.

What do you know about the physical shape of the project?  Do you know where your MER rooms are located?  Are these locations highly subject to change?

Is it a tall building or vertically scaling?  Is it a wider and longer scope with a few floors?

For a base building project, create the Architectural model.  It aids with automating equipment distances, and is also necessary to set up the {\hyperref[\detokenize{docs/userguide/buildingelectricalmodel/riser/index-riser:riser}]{\sphinxcrossref{\DUrole{std,std-ref}{Riser}}}}.

Further analysis can be done using the {\hyperref[\detokenize{docs/userguide/buildingelectricalmodel/one-line/index-one-line:one-line}]{\sphinxcrossref{\DUrole{std,std-ref}{One-Line}}}} or the {\hyperref[\detokenize{docs/userguide/buildingelectricalmodel/schedules/index-schedules:schedules}]{\sphinxcrossref{\DUrole{std,std-ref}{Schedules}}}}

Otherwise, it is best to start in the {\hyperref[\detokenize{docs/userguide/buildingelectricalmodel/riser/index-riser:riser}]{\sphinxcrossref{\DUrole{std,std-ref}{Riser}}}}.


\subsubsection{Roadmap}
\label{\detokenize{docs/tutorial/index-tutorial:roadmap}}\label{\detokenize{docs/tutorial/index-tutorial:id1}}
To help guide the design process, refer to the Roadmap.  Clicking on each node will bring you to that process.

\begin{figure}[H]
\centering
\capstart

\noindent\sphinxincludegraphics{{Roadmap}.PNG}
\caption{Roadmap}\label{\detokenize{docs/tutorial/index-tutorial:id4}}\end{figure}


\section{\sphinxstylestrong{Creating the Architectural Model}}
\label{\detokenize{docs/tutorial/index-tutorial:creating-the-architectural-model}}
In this sample project, we’re going to create a building with a 100 ft. by 100 ft. footprint, has 5 floors, and stands 50 ft. tall.

The purpose of developing the Architectural Model is to automate the calculation of Equipment distances.

Distances between Equipment are determined by the orthogonal route of their respective Room locations.

It is often necessary to route conduits through a Riser.

Click {\hyperref[\detokenize{docs/userguide/index-user_guide:architectural-workspaces}]{\sphinxcrossref{\DUrole{std,std-ref}{here}}}} for more information on the Architectural Model.


\subsection{Creating Columns and Floors}
\label{\detokenize{docs/tutorial/index-tutorial:creating-columns-and-floors}}
Open the {\hyperref[\detokenize{docs/userguide/definingarchitecturalelements/floorplans/index-floor-plans:floor-plans}]{\sphinxcrossref{\DUrole{std,std-ref}{Floor Plans}}}} and use the Setup Wizard to create Columns and Floors.

Click on Create Columns (X) and a wizard will prompt asking for a Prefix, Quantity, Offset, and Starting Dimension.

Create 11 columns, prefixed with the name “X”, with an offset of ten (10), and a starting dimension of zero (0).

\begin{figure}[H]
\centering
\capstart

\noindent\sphinxincludegraphics{{FloorPlans-SetupWizard-2}.PNG}
\caption{Creating X Columns}\label{\detokenize{docs/tutorial/index-tutorial:id5}}\end{figure}

Click on Create Columns (Y) and repeat this process for the “Y” Columns.

Then Click on Create Floors, and create 5 Floors which are spaced 10 feet apart.

If you have a general idea for your MER spaces and riser shaft locations, enable Creation Mode and create Rooms.

If you don’t, skip these steps and move straight to the next section {\hyperref[\detokenize{docs/tutorial/index-tutorial:modeling-electrical-system}]{\sphinxcrossref{\DUrole{std,std-ref}{Modeling the Electrical System}}}}.


\subsection{Creating Rooms}
\label{\detokenize{docs/tutorial/index-tutorial:creating-rooms}}
\begin{figure}[H]
\centering
\capstart

\noindent\sphinxincludegraphics{{FloorPlans-CreateRoom-1}.PNG}
\caption{Create a Room by Hovering Over a Column Section}\label{\detokenize{docs/tutorial/index-tutorial:id6}}\end{figure}

Give the Room a Name.

Assigning a Space Type and area are optional and used for loading calculations.


\subsection{Creating Stacked Rooms or a Riser}
\label{\detokenize{docs/tutorial/index-tutorial:creating-stacked-rooms-or-a-riser}}
Use Shift + Click to select multiple Floors.

Then hover over a region to create Stacked Rooms or a Riser.

\begin{figure}[H]
\centering
\capstart

\noindent\sphinxincludegraphics{{FloorPlans-multi_room-riser}.PNG}
\caption{Creating a stacked Room or a Riser while multiple Floors are selected}\label{\detokenize{docs/tutorial/index-tutorial:id7}}\end{figure}

The Riser displays as a hatched region.

\begin{figure}[H]
\centering
\capstart

\noindent\sphinxincludegraphics{{FloorPlans-CreateRiser}.PNG}
\caption{Creating a Riser}\label{\detokenize{docs/tutorial/index-tutorial:id8}}\end{figure}


\section{\sphinxstylestrong{Modeling the Electrical System}}
\label{\detokenize{docs/tutorial/index-tutorial:modeling-the-electrical-system}}\label{\detokenize{docs/tutorial/index-tutorial:modeling-electrical-system}}

\subsection{Riser}
\label{\detokenize{docs/tutorial/index-tutorial:riser}}
Once you have a base architectural model set up, click on {\hyperref[\detokenize{docs/userguide/buildingelectricalmodel/riser/index-riser:riser}]{\sphinxcrossref{\DUrole{std,std-ref}{Riser}}}} to open the Riser Workspace.

Floors are plotted automatically based on their elevation.

However, Rooms and Equipment are not automatically plotted and can be created outside of this Riser Workspace.

Instead, they are placed in an unplotted elements section and must be manually added to the Riser.

\begin{figure}[H]
\centering
\capstart

\noindent\sphinxincludegraphics{{Riser-Creation}.PNG}
\caption{Blank Riser Diagram}\label{\detokenize{docs/tutorial/index-tutorial:id9}}\end{figure}

Drag elements from the unplotted Elements section onto the Riser.

\begin{figure}[H]
\centering

\noindent\sphinxincludegraphics{{Riser-Unplotted}.PNG}
\end{figure}

Once an Equipment is dragged into a Room region, the location of the Equipment becomes associated with that Room.


\subsubsection{Placing Rooms}
\label{\detokenize{docs/tutorial/index-tutorial:placing-rooms}}
Begin by placing your equipment Rooms on the Riser.  Once an Equipment is placed in a Room region, its location becomes associated with the Room.

An Equipment does not need a Room location and there may be instances where an Equipment is not associated with a Room.

\begin{figure}[H]
\centering
\capstart

\noindent\sphinxincludegraphics{{Riser-PlaceRooms}.PNG}
\caption{Sample Riser Diagram without Equipment}\label{\detokenize{docs/tutorial/index-tutorial:id10}}\end{figure}

Once your Rooms are laid out, begin to model your Equipment.


\subsubsection{Creating a Source}
\label{\detokenize{docs/tutorial/index-tutorial:creating-a-source}}
Start with a source, or a Utility Equipment.

Click on Add Equipment, and click and drag a Utility onto the Riser.


\subsubsection{Creating Connections}
\label{\detokenize{docs/tutorial/index-tutorial:creating-connections}}
Select the Utility and then use the arrows pointing outwards to start creating outbound connections.

Draw connections using the left-click button.

\begin{figure}[H]
\centering
\capstart

\noindent\sphinxincludegraphics{{Riser-Connections-1}.PNG}
\caption{Creating an Outbound Connections}\label{\detokenize{docs/tutorial/index-tutorial:id11}}\end{figure}

Then use Enter to create an Equipment.

\begin{figure}[H]
\centering
\capstart

\noindent\sphinxincludegraphics{{Riser-Connections-2}.PNG}
\caption{Feeding a Distribution Equipment}\label{\detokenize{docs/tutorial/index-tutorial:id12}}\end{figure}

Create the arrangement shown below:

\begin{figure}[H]
\centering
\capstart

\noindent\sphinxincludegraphics{{Riser-Connections-4}.PNG}
\caption{A Distribution Board, Feeding a Step-Down Transformer, a Tap Node, and Panelboards}\label{\detokenize{docs/tutorial/index-tutorial:id13}}\end{figure}


\subsubsection{Copying and Pasting Equipment}
\label{\detokenize{docs/tutorial/index-tutorial:copying-and-pasting-equipment}}
To Copy and Paste Equipment, select a group of equipment by dragging a selection box around them.

The selected Equipment will be highlighted.

\begin{figure}[H]
\centering
\capstart

\noindent\sphinxincludegraphics{{Riser-Connections-5}.PNG}
\caption{A Group of Selected Equipment}\label{\detokenize{docs/tutorial/index-tutorial:id14}}\end{figure}

Then use CTRL+C to copy.

Use CTRL+V to paste.

\begin{figure}[H]
\centering
\capstart

\noindent\sphinxincludegraphics{{Riser-Connections-6}.PNG}
\caption{A Group of Selected Equipment}\label{\detokenize{docs/tutorial/index-tutorial:id15}}\end{figure}

Feed the equipment by creating outbound connections from the source (MDB).

Alternatively, use the arrows pointing inwards to create a connection which feeds the selected Equipment.

\begin{figure}[H]
\centering
\capstart

\noindent\sphinxincludegraphics{{Riser-Connections-7}.PNG}
\caption{Creating Outbound Connections}\label{\detokenize{docs/tutorial/index-tutorial:id16}}\end{figure}

Draw connections from the source to the load.

\begin{figure}[H]
\centering
\capstart

\noindent\sphinxincludegraphics{{Riser-Connections-8}.PNG}
\caption{Drawing Connections Between Equipment}\label{\detokenize{docs/tutorial/index-tutorial:id17}}\end{figure}


\subsubsection{Moving Equipment, Floors, or Rooms}
\label{\detokenize{docs/tutorial/index-tutorial:moving-equipment-floors-or-rooms}}
To move equipment, floors, or rooms, first select a group of equipment by using a selection box.  Then drag and drop the entities to the new location.

It is important to note that the elevations of the floors are disconnected from their visual representation.  Shifting a floor does not change its elevation.

\begin{figure}[H]
\centering
\capstart

\noindent\sphinxincludegraphics{{Riser-Move}.PNG}
\caption{Moving Equipment, Floors, and Rooms}\label{\detokenize{docs/tutorial/index-tutorial:id18}}\end{figure}


\subsubsection{Creating Transfer Switches}
\label{\detokenize{docs/tutorial/index-tutorial:creating-transfer-switches}}
Create a transfer switch by modeling an ATS/STS.

Create the emergency source Equipment and emergency Panelboards shown below:

\begin{figure}[H]
\centering
\capstart

\noindent\sphinxincludegraphics{{Riser-Connections-11}.PNG}
\caption{Creating Emergency Equipment}\label{\detokenize{docs/tutorial/index-tutorial:id19}}\end{figure}

Massage the layout of the Riser as needed.

\begin{figure}[H]
\centering
\capstart

\noindent\sphinxincludegraphics{{Riser-Completed}.PNG}
\caption{Example of Finished Riser Diagram}\label{\detokenize{docs/tutorial/index-tutorial:id20}}\end{figure}

For further analysis of your system, use the One-Line.


\subsection{One-Line}
\label{\detokenize{docs/tutorial/index-tutorial:one-line}}
The One-Line is a top-down view of your electrical system.  Power starts at a source and flows down to branch loads.  Refer to the {\hyperref[\detokenize{docs/userguide/buildingelectricalmodel/one-line/index-one-line:one-line}]{\sphinxcrossref{\DUrole{std,std-ref}{One-Line}}}} in the {\hyperref[\detokenize{docs/userguide/index-user_guide:user-guide}]{\sphinxcrossref{\DUrole{std,std-ref}{User Guide}}}} for more information.

The One-Line is generally used for analyzing your system from a power flow perspective, as opposed to the Riser’s construction and location perspective.

\begin{figure}[H]
\centering
\capstart

\noindent\sphinxincludegraphics{{One-Line-Sample}.PNG}
\caption{Sample One-Line}\label{\detokenize{docs/tutorial/index-tutorial:id21}}\end{figure}

The One-Line contains various tools to analyze and modify your system.

\begin{figure}[H]
\centering
\capstart

\noindent\sphinxincludegraphics{{One-Line-Load_Flow}.PNG}
\caption{Sample of the One-Line with the Load Flow view toggled on}\label{\detokenize{docs/tutorial/index-tutorial:id22}}\end{figure}


\subsection{Schedules}
\label{\detokenize{docs/tutorial/index-tutorial:schedules}}
The Schedules are a tabular representation of your distribution system.  It’s a much more rapid environment to create equipment.

It does not diagrammatically represent the locations of Equipment as well as a Riser diagram.

\begin{figure}[H]
\centering
\capstart

\noindent\sphinxincludegraphics{{Schedules-Sample}.PNG}
\caption{Sample of Schedules Workspace}\label{\detokenize{docs/tutorial/index-tutorial:id23}}\end{figure}

Refer to the {\hyperref[\detokenize{docs/userguide/buildingelectricalmodel/schedules/index-schedules:schedules}]{\sphinxcrossref{\DUrole{std,std-ref}{Schedules}}}} for more information.


\section{\sphinxstylestrong{Exporting, Studies, and Reporting}}
\label{\detokenize{docs/tutorial/index-tutorial:exporting-studies-and-reporting}}

\subsection{Exporting to AutoCAD}
\label{\detokenize{docs/tutorial/index-tutorial:exporting-to-autocad}}
The {\hyperref[\detokenize{docs/userguide/index-user_guide:electrical-workspaces}]{\sphinxcrossref{\DUrole{std,std-ref}{Electrical Workspaces}}}}: {\hyperref[\detokenize{docs/userguide/buildingelectricalmodel/one-line/index-one-line:one-line}]{\sphinxcrossref{\DUrole{std,std-ref}{One-Line}}}}, {\hyperref[\detokenize{docs/userguide/buildingelectricalmodel/riser/index-riser:riser}]{\sphinxcrossref{\DUrole{std,std-ref}{Riser}}}}, and {\hyperref[\detokenize{docs/userguide/buildingelectricalmodel/schedules/index-schedules:schedules}]{\sphinxcrossref{\DUrole{std,std-ref}{Schedules}}}} are all exportable to AutoCAD.

Use the Export button (down arrow) and use Export to AutoCAD..

\begin{figure}[H]
\centering
\capstart

\noindent\sphinxincludegraphics{{One-Line-Export}.PNG}
\caption{Exporting the One-Line}\label{\detokenize{docs/tutorial/index-tutorial:id24}}\end{figure}

\begin{figure}[H]
\centering
\capstart

\noindent\sphinxincludegraphics{{Riser-Export}.PNG}
\caption{Exporting the Riser}\label{\detokenize{docs/tutorial/index-tutorial:id25}}\end{figure}

\begin{figure}[H]
\centering
\capstart

\noindent\sphinxincludegraphics{{Schedules-Export}.PNG}
\caption{Exporting the Schedules}\label{\detokenize{docs/tutorial/index-tutorial:id26}}\end{figure}

The Studies are a reporting view of your design.  Reports like Voltage Drop, Loading, and Short Circuit are available.

\begin{figure}[H]
\centering
\capstart

\noindent\sphinxincludegraphics{{Studies-Print}.PNG}
\caption{Studies are printable and exportable to Excel}\label{\detokenize{docs/tutorial/index-tutorial:id27}}\end{figure}


\subsection{Studies and Reporting}
\label{\detokenize{docs/tutorial/index-tutorial:studies-and-reporting}}
The {\hyperref[\detokenize{docs/userguide/buildingelectricalmodel/studies/index-studies:studies}]{\sphinxcrossref{\DUrole{std,std-ref}{Studies}}}} and {\hyperref[\detokenize{docs/userguide/pricingmodel/pricingreport/index-pricing_report:pricing-report}]{\sphinxcrossref{\DUrole{std,std-ref}{Pricing Report}}}} are reporting mechanisms for engineering studies and pricing.

Both Workspaces are exportable to Excel.

\begin{figure}[H]
\centering
\capstart

\noindent\sphinxincludegraphics{{Studies-Export}.PNG}
\caption{Exporting the Studies to Excel}\label{\detokenize{docs/tutorial/index-tutorial:id28}}\end{figure}

\begin{figure}[H]
\centering
\capstart

\noindent\sphinxincludegraphics{{Pricing-Report-Export}.PNG}
\caption{Exporting the Pricing Report to Excel}\label{\detokenize{docs/tutorial/index-tutorial:id29}}\end{figure}

For a complete guide of all features, please refer to the {\hyperref[\detokenize{docs/userguide/index-user_guide:user-guide}]{\sphinxcrossref{\DUrole{std,std-ref}{User Guide}}}}.


\chapter{\sphinxstylestrong{User Guide - Getting Started}}
\label{\detokenize{docs/index:user-guide-getting-started}}

\section{\sphinxstylestrong{Signing In}}
\label{\detokenize{docs/userguide/index-user_guide:signing-in}}\label{\detokenize{docs/userguide/index-user_guide:user-guide}}\label{\detokenize{docs/userguide/index-user_guide::doc}}
Once the application is open, sign in with your account.

\begin{figure}[H]
\centering
\capstart

\noindent\sphinxincludegraphics{{SignIn}.PNG}
\caption{Sign In at the Top Right}\label{\detokenize{docs/userguide/index-user_guide:id14}}\end{figure}


\section{\sphinxstylestrong{Navigating THRUX}}
\label{\detokenize{docs/userguide/index-user_guide:navigating-thrux}}

\subsection{Workspace Toolbar}
\label{\detokenize{docs/userguide/index-user_guide:workspace-toolbar}}\label{\detokenize{docs/userguide/index-user_guide:id1}}
The left-side toolbar is the Workspace Toolbar.  The purple shading indicates an open and active Workspace, while grey shading only indicates an open Workspace.

\begin{figure}[H]
\centering
\capstart

\noindent\sphinxincludegraphics{{workspace-toolbar-1}.PNG}
\caption{Workspace Toolbar}\label{\detokenize{docs/userguide/index-user_guide:id15}}\end{figure}

Workspaces can be detached to separate windows by clicking and dragging the Workspace’s icon in the Workspace toolbar.

These windows can be docked by clicking Dock in the top right of the undocked window.

\begin{figure}[H]
\centering
\capstart

\noindent\sphinxincludegraphics{{workspace-toolbar-dock}.PNG}
\caption{Docked One-Line separate from the main application}\label{\detokenize{docs/userguide/index-user_guide:id16}}\end{figure}

\begin{figure}[H]
\centering
\capstart

\noindent\sphinxincludegraphics{{workspace_docking}.PNG}
\caption{Docking Workspaces allows Workspaces to be viewed on multiple monitors}\label{\detokenize{docs/userguide/index-user_guide:id17}}\end{figure}


\subsection{Explorer Toolbar}
\label{\detokenize{docs/userguide/index-user_guide:explorer-toolbar}}
Explorers and other Utility tools can be found on the {\hyperref[\detokenize{docs/userguide/index-user_guide:id10}]{\sphinxcrossref{\DUrole{std,std-ref}{Explorer Toolbar}}}} on the right-side toolbar.  These tools can be pinned to always be visible (pin icon).

\begin{figure}[H]
\centering
\capstart

\noindent\sphinxincludegraphics{{explorer-toolbar-1}.PNG}
\caption{Using the {\hyperref[\detokenize{docs/userguide/explorersandutilitytools/propertiesexplorer/index-properties_explorer:properties-explorer}]{\sphinxcrossref{\DUrole{std,std-ref}{Properties Explorer}}}} and {\hyperref[\detokenize{docs/userguide/buildingelectricalmodel/flagtracker/index-flag_tracker:flag-tracker}]{\sphinxcrossref{\DUrole{std,std-ref}{Flag Tracker}}}} while working in the {\hyperref[\detokenize{docs/userguide/buildingelectricalmodel/one-line/index-one-line:one-line}]{\sphinxcrossref{\DUrole{std,std-ref}{One-Line}}}} Workspace}\label{\detokenize{docs/userguide/index-user_guide:id18}}\end{figure}


\subsection{Navigation Bar}
\label{\detokenize{docs/userguide/index-user_guide:navigation-bar}}\label{\detokenize{docs/userguide/index-user_guide:id2}}
Designers have the ability to search for anything within the model.

For example, if searching for a piece of Equipment, simply type the name.

Options will be displayed where something is found.

\begin{figure}[H]
\centering
\capstart

\noindent\sphinxincludegraphics{{navigation_bar-1}.PNG}
\caption{Navigation Bar}\label{\detokenize{docs/userguide/index-user_guide:id19}}\end{figure}

Selection will navigate you to the location within the application.

\begin{figure}[H]
\centering
\capstart

\noindent\sphinxincludegraphics{{navigation_bar-2}.PNG}
\caption{Navigating to the Schedule for MDB-1}\label{\detokenize{docs/userguide/index-user_guide:id20}}\end{figure}


\subsection{Additional Commands}
\label{\detokenize{docs/userguide/index-user_guide:additional-commands}}
Zoom Extents allows the user to quickly pan to the center of the screen.  Double click the mouse wheel to Zoom Extents.

\index{Getting Started@\spxentry{Getting Started}}\ignorespaces 

\section{\sphinxstylestrong{Creating a New Project}}
\label{\detokenize{docs/userguide/index-user_guide:creating-a-new-project}}\label{\detokenize{docs/userguide/index-user_guide:index-0}}
Click the Home button, and then the Add (+) button to create a new Project.

\begin{figure}[H]
\centering
\capstart

\noindent\sphinxincludegraphics{{new_project-1}.PNG}
\caption{The Home Screen is a Project portal that provides quick access to recent Projects}\label{\detokenize{docs/userguide/index-user_guide:id21}}\end{figure}


\subsection{Opening an Existing Project}
\label{\detokenize{docs/userguide/index-user_guide:opening-an-existing-project}}
Existing projects or recently opened projects are shown below the Open Project section.

\begin{figure}[H]
\centering
\capstart

\noindent\sphinxincludegraphics{{open_project-1}.PNG}
\caption{Check the recently opened lists to pick up where you left off}\label{\detokenize{docs/userguide/index-user_guide:id22}}\end{figure}


\subsection{Project Settings}
\label{\detokenize{docs/userguide/index-user_guide:project-settings}}\label{\detokenize{docs/userguide/index-user_guide:id3}}
Project Settings are a set of customizable parameters on which to base the design.

\begin{figure}[H]
\centering
\capstart

\noindent\sphinxincludegraphics{{project_settings-1}.PNG}
\caption{Project Settings are accessible by clicking on File, and then Settings}\label{\detokenize{docs/userguide/index-user_guide:id23}}\end{figure}


\subsubsection{Default Model Parameters}
\label{\detokenize{docs/userguide/projectsettings/defaultmodelparameters/index-default_model_parameters:default-model-parameters}}\label{\detokenize{docs/userguide/projectsettings/defaultmodelparameters/index-default_model_parameters:id1}}\label{\detokenize{docs/userguide/projectsettings/defaultmodelparameters/index-default_model_parameters::doc}}
The entities or objects used to build the model need to have some initial or base information.  Every time a new entity is created it will default to these parameters.  These settings are meant to mimic the consulting Engineer’s specifications.  As more updates are made in the future, this section will be expanded to encapsulate more typical statements in MEP building specifications.  For example, all distribution boards can be modeled to default to having an isolated ground bus.

\begin{figure}[H]
\centering
\capstart

\noindent\sphinxincludegraphics{{default_model_parameters}.PNG}
\caption{Setting the default Conductor Type for all circuits to be copper (Cu.)}\label{\detokenize{docs/userguide/projectsettings/defaultmodelparameters/index-default_model_parameters:id2}}\end{figure}

For more information about these default parameters, or Equipment properties and definitions, click {\hyperref[\detokenize{docs/definitions/index-definitions:default-model-parameters-definitions}]{\sphinxcrossref{\DUrole{std,std-ref}{here}}}}, or see our {\hyperref[\detokenize{docs/definitions/index-definitions:definitions}]{\sphinxcrossref{\DUrole{std,std-ref}{Definitions}}}} section.


\subsubsection{Calculation Settings}
\label{\detokenize{docs/userguide/projectsettings/calculationsettings/index-calculation_settings:calculation-settings}}\label{\detokenize{docs/userguide/projectsettings/calculationsettings/index-calculation_settings:id1}}\label{\detokenize{docs/userguide/projectsettings/calculationsettings/index-calculation_settings::doc}}
Specify additional calculation settings here.

For example, the Temperature Rating Threshold (NEC Table 310.15(B)) for conductors can be specified here.  The default is 110 amps.

\begin{figure}[H]
\centering
\capstart

\noindent\sphinxincludegraphics{{calculation_settings}.PNG}
\caption{Calculation Settings}\label{\detokenize{docs/userguide/projectsettings/calculationsettings/index-calculation_settings:id2}}\end{figure}

For more information on these calculation settings see {\hyperref[\detokenize{docs/definitions/index-definitions:calculation-settings-definitions}]{\sphinxcrossref{\DUrole{std,std-ref}{here}}}}.


\subsubsection{Auto-Sizing Filters}
\label{\detokenize{docs/userguide/projectsettings/autosizingfilters/index-auto_sizing_filters:auto-sizing-filters}}\label{\detokenize{docs/userguide/projectsettings/autosizingfilters/index-auto_sizing_filters:id1}}\label{\detokenize{docs/userguide/projectsettings/autosizingfilters/index-auto_sizing_filters::doc}}
Auto-Sizing Filters present designers with an option to restrict certain materials from being used in the design.

Depending on the level of Design Assistance, the THRUX engine will be filling out certain parameters when enough information is specified to alleviate repetitive hand calculations.

For example, if a designer is specifying a 400A distribution board, THRUX will calculate the circuit’s minimum conduit size to be 3-1/2” conduit (would vary on additional parameters).  However, if a designer wants to use only 4” conduits in favor of 3-1/2”, due to a Contractor’s request, 3-1/2” can be unchecked from the Conduit Sizes and THRUX will choose the next available checked size up from code-minimum.

\begin{figure}[H]
\centering
\capstart

\noindent\sphinxincludegraphics{{auto_sizing_filters}.PNG}
\caption{Auto-Sizing Filters - Omitting a 400 kCMil cable from the design}\label{\detokenize{docs/userguide/projectsettings/autosizingfilters/index-auto_sizing_filters:id2}}\end{figure}


\subsubsection{Flag Settings}
\label{\detokenize{docs/userguide/projectsettings/flagsettings/index-flag_settings:flag-settings}}\label{\detokenize{docs/userguide/projectsettings/flagsettings/index-flag_settings:id1}}\label{\detokenize{docs/userguide/projectsettings/flagsettings/index-flag_settings::doc}}
THRUX provides code-validated designs and presents designers with notifications or Flags if a scenario is violating a Project’s applicable safety code or standard.  Certain Flags can be ignored or have the visibility setting omitted.  Deselect to omit a Flag.

Yellow Flags are warnings of violations of applicable safety codes.

Orange Flags are violations of applicable safety codes.

Red Flags are program errors, such as a piece of Equipment without a name.

\begin{figure}[H]
\centering
\capstart

\noindent\sphinxincludegraphics{{flag-settings-1}.PNG}
\caption{Identifying Overload and Voltage Drop Flags while creating the One-Line}\label{\detokenize{docs/userguide/projectsettings/flagsettings/index-flag_settings:id2}}\end{figure}

For more information about Flags and their conditions, see {\hyperref[\detokenize{docs/definitions/index-definitions:flag-settings-definitions}]{\sphinxcrossref{\DUrole{std,std-ref}{Flag Settings Definitions}}}}.


\subsubsection{Custom Design Assistance Settings}
\label{\detokenize{docs/userguide/projectsettings/customdesignassistancesettings/index-custom_design_assistance_settings:custom-design-assistance-settings}}\label{\detokenize{docs/userguide/projectsettings/customdesignassistancesettings/index-custom_design_assistance_settings:id1}}\label{\detokenize{docs/userguide/projectsettings/customdesignassistancesettings/index-custom_design_assistance_settings::doc}}
THRUX offers three (3) different levels of {\hyperref[\detokenize{docs/userguide/explorersandutilitytools/statusbar/index-status_bar:design-assistance}]{\sphinxcrossref{\DUrole{std,std-ref}{Design Assistance}}}}: {\hyperref[\detokenize{docs/userguide/explorersandutilitytools/statusbar/index-status_bar:full-design-assistance}]{\sphinxcrossref{\DUrole{std,std-ref}{Full Design Assistance}}}}, {\hyperref[\detokenize{docs/userguide/explorersandutilitytools/statusbar/index-status_bar:custom-design-assistance}]{\sphinxcrossref{\DUrole{std,std-ref}{Custom Design Assistance}}}}, and {\hyperref[\detokenize{docs/userguide/explorersandutilitytools/statusbar/index-status_bar:no-design-assistance}]{\sphinxcrossref{\DUrole{std,std-ref}{No Design Assistance}}}}.

Under normal conditions, THRUX is in {\hyperref[\detokenize{docs/userguide/explorersandutilitytools/statusbar/index-status_bar:full-design-assistance}]{\sphinxcrossref{\DUrole{std,std-ref}{Full Design Assistance}}}} mode and will calculate equipment sizes automatically.  {\hyperref[\detokenize{docs/userguide/explorersandutilitytools/statusbar/index-status_bar:no-design-assistance}]{\sphinxcrossref{\DUrole{std,std-ref}{No Design Assistance}}}} mode allows full control of the model, and will still display Flags or code-violations.  However, in {\hyperref[\detokenize{docs/userguide/explorersandutilitytools/statusbar/index-status_bar:custom-design-assistance}]{\sphinxcrossref{\DUrole{std,std-ref}{Custom Design Assistance}}}} mode, certain parameters can be excluded from equipment sizing.

For example, if the designer would like a conduit size to remain constant or intentionally oversized to account for shaft space, deselect Conduit Size and turn Custom Design Assistance mode on as shown below.  As the designer is accommodating load, the conduit size will not recalculate.

\begin{figure}[H]
\centering
\capstart

\noindent\sphinxincludegraphics{{custom_design_assistance_settings}.PNG}
\caption{Deselect items to exclude them from equipment sizing}\label{\detokenize{docs/userguide/projectsettings/customdesignassistancesettings/index-custom_design_assistance_settings:id2}}\end{figure}


\subsubsection{\sphinxstylestrong{Labor Rates}}
\label{\detokenize{docs/userguide/projectsettings/laborrates/index-labor_rates:labor-rates}}\label{\detokenize{docs/userguide/projectsettings/laborrates/index-labor_rates:id1}}\label{\detokenize{docs/userguide/projectsettings/laborrates/index-labor_rates::doc}}
Labor rates for Journeyman and Foreman are customizable.

\begin{figure}[H]
\centering
\capstart

\noindent\sphinxincludegraphics{{labor_rates}.PNG}
\caption{Change these at any time by clicking File, and then Settings}\label{\detokenize{docs/userguide/projectsettings/laborrates/index-labor_rates:id2}}\end{figure}


\subsection{Roadmap}
\label{\detokenize{docs/userguide/index-user_guide:roadmap}}
To help guide the design process, refer to the Roadmap.  Clicking on each node will bring you to that process.

\begin{figure}[H]
\centering
\capstart

\noindent\sphinxincludegraphics{{Roadmap1}.PNG}
\caption{Roadmap}\label{\detokenize{docs/userguide/index-user_guide:id24}}\end{figure}


\section{\sphinxstylestrong{Defining Architectural Elements}}
\label{\detokenize{docs/userguide/index-user_guide:defining-architectural-elements}}\label{\detokenize{docs/userguide/index-user_guide:architectural-workspaces}}

\subsection{Overview}
\label{\detokenize{docs/userguide/index-user_guide:overview}}
The goal of the Architectural Workspaces, {\hyperref[\detokenize{docs/userguide/definingarchitecturalelements/floorplans/index-floor-plans:floor-plans}]{\sphinxcrossref{\DUrole{std,std-ref}{Floor Plans}}}} and {\hyperref[\detokenize{docs/userguide/definingarchitecturalelements/archelements/index-arch-elements:arch-elements}]{\sphinxcrossref{\DUrole{std,std-ref}{Arch. Elements}}}}, is to provide a way for you to quickly mass the load of a building.  These locations aid with point-to-point calculations such as voltage drop.

However, these Workspaces are completely optional.  For a smaller project, you may not find it necessary to set up these Workspaces and instead, find it faster to manually input feeder lengths in the {\hyperref[\detokenize{docs/userguide/buildingelectricalmodel/one-line/index-one-line:one-line}]{\sphinxcrossref{\DUrole{std,std-ref}{One-Line}}}}.

This information can be {\hyperref[\detokenize{docs/userguide/index-user_guide:revit-interoperability}]{\sphinxcrossref{\DUrole{std,std-ref}{imported}}}} from an Architectural Revit model.

The {\hyperref[\detokenize{docs/userguide/definingarchitecturalelements/archelements/index-arch-elements:arch-elements}]{\sphinxcrossref{\DUrole{std,std-ref}{Arch. Elements}}}} Workspace allows you to modify the architectural elements of the model.  Here, it is possible to modify other characteristics of a Floor.  For example, when massing the load of a building, you may want to assign a load requirement or load density to each Floor.  This load is based off of the Floor’s Space Type.

These elements can be created in the {\hyperref[\detokenize{docs/userguide/definingarchitecturalelements/floorplans/index-floor-plans:floor-plans}]{\sphinxcrossref{\DUrole{std,std-ref}{Floor Plans}}}} Workspace, or the {\hyperref[\detokenize{docs/userguide/definingarchitecturalelements/archelements/index-arch-elements:arch-elements}]{\sphinxcrossref{\DUrole{std,std-ref}{Architectural Elements}}}} Workspace.


\subsubsection{Equipment Distances}
\label{\detokenize{docs/userguide/index-user_guide:equipment-distances}}

\paragraph{Calculated Length}
\label{\detokenize{docs/userguide/index-user_guide:calculated-length}}\label{\detokenize{docs/userguide/index-user_guide:id4}}
Distances between Equipment are determined by their respective Room locations.  Calc. Length (Calculated) represents the distance between two Rooms via an orthogonal route.

The vertical distance between Rooms is the difference between their respective elevations.

\begin{figure}[H]
\centering
\capstart

\noindent\sphinxincludegraphics{{equipment_distances-1}.PNG}
\caption{Route between Rooms on the same Floor, and vertical distance between stacked Rooms}\label{\detokenize{docs/userguide/index-user_guide:id25}}\end{figure}

It is often necessary to offset through a Riser.  The total distance or {\hyperref[\detokenize{docs/userguide/index-user_guide:net-length}]{\sphinxcrossref{\DUrole{std,std-ref}{Net Length}}}} is determined by the centerpoints of the respective entities.

\begin{figure}[H]
\centering
\capstart

\noindent\sphinxincludegraphics{{equipment_distances-2}.PNG}
\caption{Routing from Room A, through Riser A, and terminating at Room B}\label{\detokenize{docs/userguide/index-user_guide:id26}}\end{figure}


\paragraph{Manual Added Length}
\label{\detokenize{docs/userguide/index-user_guide:manual-added-length}}\label{\detokenize{docs/userguide/index-user_guide:id5}}
Manual Added Length is an additional factor which is added to a circuit’s {\hyperref[\detokenize{docs/userguide/index-user_guide:calculated-length}]{\sphinxcrossref{\DUrole{std,std-ref}{Calc. Length}}}} property and is a customizable default setting.  See {\hyperref[\detokenize{docs/userguide/projectsettings/defaultmodelparameters/index-default_model_parameters:default-model-parameters}]{\sphinxcrossref{\DUrole{std,std-ref}{here}}}} for more information.


\paragraph{Net Length}
\label{\detokenize{docs/userguide/index-user_guide:net-length}}\label{\detokenize{docs/userguide/index-user_guide:id6}}
The Net Length is composed of the {\hyperref[\detokenize{docs/userguide/index-user_guide:calculated-length}]{\sphinxcrossref{\DUrole{std,std-ref}{Calc. Length}}}} and the {\hyperref[\detokenize{docs/userguide/index-user_guide:manual-added-length}]{\sphinxcrossref{\DUrole{std,std-ref}{Manual Added Length}}}}.


\subsubsection{Loading System}
\label{\detokenize{docs/userguide/index-user_guide:loading-system}}
The Architectural Elements are the unifying components which Engineers base their design.  However, various engineering disciplines refine and base their calculations off of different elements.

You have the ability to abstract the architectural elements of a model to create Load Packages.  Ideally these entities would be abstracted away from a Revit model.  However, during the early stages of development, a Revit model may not exist.

Creation of these architectural entities allow for the THRUX engine to fully utilize its Loading System which applies loading factors dictated by the designer and the Project’s applicable safety codes and standards.

\begin{figure}[H]
\centering
\capstart

\noindent\sphinxincludegraphics{{loading_system-1}.PNG}
\caption{Creating Architectural Packages to model loads}\label{\detokenize{docs/userguide/index-user_guide:id27}}\end{figure}

Modeling residential loads relies on creating entities that are based off of the NEC.

\begin{figure}[H]
\centering
\capstart

\noindent\sphinxincludegraphics{{loading_system-2}.PNG}
\caption{Creating Architectural Packages to model residential loads}\label{\detokenize{docs/userguide/index-user_guide:id28}}\end{figure}

Groups of architectural entities can be created within THRUX, or imported from a Revit model.  Ultimately, Architectural Packages are loads that can be attached to Equipment.


\subsection{Floor Plans}
\label{\detokenize{docs/userguide/index-user_guide:floor-plans}}\label{\detokenize{docs/userguide/index-user_guide:floor-plans-overview}}
The {\hyperref[\detokenize{docs/userguide/definingarchitecturalelements/floorplans/index-floor-plans::doc}]{\sphinxcrossref{\DUrole{doc}{Floor Plans}}}} Workspace is a 2-D representation of the Project and is used to model locations of Equipment.


\subsubsection{Additional Commands}
\label{\detokenize{docs/userguide/definingarchitecturalelements/floorplans/index-floor-plans:additional-commands}}\label{\detokenize{docs/userguide/definingarchitecturalelements/floorplans/index-floor-plans:floor-plans}}\label{\detokenize{docs/userguide/definingarchitecturalelements/floorplans/index-floor-plans::doc}}

\begin{savenotes}\sphinxattablestart
\centering
\begin{tabulary}{\linewidth}[t]{|T|T|}
\hline
\sphinxstyletheadfamily 
\sphinxstylestrong{Command}
&\sphinxstyletheadfamily 
\sphinxstylestrong{Description}
\\
\hline
Select All
&
Use CTRL+A to select all entities.
\\
\hline
Cut/Copy
&
Use CTRL+C to copy and CTRL+C to cut.
\\
\hline
Paste
&
Use CTRL+V to paste.
\\
\hline
Find
&
Use CTRL+F to search.
\\
\hline
Zoom Extents
&
Double-click the mouse wheel to zoom and pan to the extents of the window content.
\\
\hline
\end{tabulary}
\par
\sphinxattableend\end{savenotes}


\subsubsection{Setup Wizard}
\label{\detokenize{docs/userguide/definingarchitecturalelements/floorplans/index-floor-plans:setup-wizard}}
Use the Setup Wizard located on the right to start to create columns and Floors.  Any entity created using the wizard can also be created or modified in the {\hyperref[\detokenize{docs/userguide/definingarchitecturalelements/archelements/index-arch-elements:arch-elements}]{\sphinxcrossref{\DUrole{std,std-ref}{Arch. Elements}}}} Workspace.

Selecting Create Columns will create multiple columns at a time, and Create Floors will create multiple Floors at a time.

Click on Create Columns (X) and a wizard will prompt asking for a Prefix, Quantity, Offset, and Starting Dimension.

\begin{figure}[H]
\centering
\capstart

\noindent\sphinxincludegraphics{{setup_wizard-1}.PNG}
\caption{Setup Wizard is accessible by clicking on the wand icon in the top right}\label{\detokenize{docs/userguide/definingarchitecturalelements/floorplans/index-floor-plans:id1}}\end{figure}

Offset is the distance in between each column.  The Starting Dimension is the starting X, Y, or Z coordinate.

Create 11 columns, prefixed with the name “X”, with an offset of ten (10), and a starting dimension of zero (0).

\begin{figure}[H]
\centering
\capstart

\noindent\sphinxincludegraphics{{setup_wizard-2}.PNG}
\caption{Using the Setup Wizard to create columns}\label{\detokenize{docs/userguide/definingarchitecturalelements/floorplans/index-floor-plans:id2}}\end{figure}

After clicking Create, the columns will appear in the Floor Plans.

\begin{figure}[H]
\centering
\capstart

\noindent\sphinxincludegraphics{{setup_wizard-3}.PNG}
\caption{Columns along the x-axis}\label{\detokenize{docs/userguide/definingarchitecturalelements/floorplans/index-floor-plans:id3}}\end{figure}

Click on Create Columns (Y) to repeat this process.

\begin{figure}[H]
\centering
\capstart

\noindent\sphinxincludegraphics{{setup_wizard-4}.PNG}
\caption{Columns along the x-axis and y-axis}\label{\detokenize{docs/userguide/definingarchitecturalelements/floorplans/index-floor-plans:id4}}\end{figure}

Next, click on Create Floors.  Create 11 Floors that are also vertically spaced 10 feet apart.

Cycle through Floors by selecting the Floor on the left sidebar.

\begin{figure}[H]
\centering
\capstart

\noindent\sphinxincludegraphics{{setup_wizard-5}.PNG}
\caption{100 foot by 100 foot by 100 foot building}\label{\detokenize{docs/userguide/definingarchitecturalelements/floorplans/index-floor-plans:id5}}\end{figure}


\subsubsection{Grid Editor}
\label{\detokenize{docs/userguide/definingarchitecturalelements/floorplans/index-floor-plans:grid-editor}}
Use the Grid Editor to modify the spacing in between columns.  Click on the gear icon to use the Grid Editor.

\begin{figure}[H]
\centering
\capstart

\noindent\sphinxincludegraphics{{grid_editor-1}.PNG}
\caption{Grid Editor is active}\label{\detokenize{docs/userguide/definingarchitecturalelements/floorplans/index-floor-plans:id6}}\end{figure}

Use the grips to drag columns and change their dimensions.  When finished, click on Accept or Reject to save or cancel your changes.

\begin{figure}[H]
\centering
\capstart

\noindent\sphinxincludegraphics{{grid_editor-2}.PNG}
\caption{Dragging a grip shows all associated dimensions}\label{\detokenize{docs/userguide/definingarchitecturalelements/floorplans/index-floor-plans:id7}}\end{figure}


\subsubsection{Creation Mode}
\label{\detokenize{docs/userguide/definingarchitecturalelements/floorplans/index-floor-plans:creation-mode}}
Once Floors are created, enable Creation Mode.  This allows you to create Rooms and Risers.

Create a Room by hovering the mouse between column regions and clicking Add Room.

\begin{figure}[H]
\centering
\capstart

\noindent\sphinxincludegraphics{{creation_mode}.PNG}
\caption{Creating a Room with Creation Mode enabled}\label{\detokenize{docs/userguide/definingarchitecturalelements/floorplans/index-floor-plans:id8}}\end{figure}

To resize a Room, disable Creation Mode and click on a Room.

\begin{figure}[H]
\centering
\capstart

\noindent\sphinxincludegraphics{{room_resize}.PNG}
\caption{Click and drag the grips to resize a Room}\label{\detokenize{docs/userguide/definingarchitecturalelements/floorplans/index-floor-plans:id9}}\end{figure}

Select multiple Floors by using Shift+Click.  Then hover over a grid region and select Add Room or Add Riser.

While multiple Floors are selected, selecting Add Room will create a Room in a common location which spans the selected Floors.

Create Riser will create a Riser which also spans the selected Floors.

\begin{figure}[H]
\centering
\capstart

\noindent\sphinxincludegraphics{{floorplans-multi_room-riser}.PNG}
\caption{Creating a stacked Room or a Riser while multiple Floors are selected}\label{\detokenize{docs/userguide/definingarchitecturalelements/floorplans/index-floor-plans:id10}}\end{figure}
\phantomsection\label{\detokenize{docs/userguide/definingarchitecturalelements/floorplans/index-floor-plans:add-floor}}
Another way to create Floors is to use the Add Floor button.

\begin{figure}[H]
\centering
\capstart

\noindent\sphinxincludegraphics{{add_floor-1}.PNG}
\caption{Add Floor}\label{\detokenize{docs/userguide/definingarchitecturalelements/floorplans/index-floor-plans:id11}}\end{figure}

\begin{figure}[H]
\centering
\capstart

\noindent\sphinxincludegraphics{{add_floor-2}.PNG}
\caption{Adding a Roof at an elevation of 1100 feet}\label{\detokenize{docs/userguide/definingarchitecturalelements/floorplans/index-floor-plans:id12}}\end{figure}


\subsubsection{Move Equipment}
\label{\detokenize{docs/userguide/definingarchitecturalelements/floorplans/index-floor-plans:move-equipment}}\label{\detokenize{docs/userguide/definingarchitecturalelements/floorplans/index-floor-plans:floor-plans-move-equipment}}
It is a common task to study changing the locations of Equipment.  Move Equipment is intended to quickly place Equipment in Rooms.

\begin{figure}[H]
\centering
\capstart

\noindent\sphinxincludegraphics{{move-equipment_1}.PNG}
\caption{Click Move Equipment in the top right}\label{\detokenize{docs/userguide/definingarchitecturalelements/floorplans/index-floor-plans:id13}}\end{figure}

There are two collections of Equipment: Orphaned Equipment and Hosted Equipment.

An Orphaned Equipment does not have a Room assigned to it while a Hosted Equipment has a Room assigned to it.

\begin{figure}[H]
\centering
\capstart

\noindent\sphinxincludegraphics{{move-equipment_2}.PNG}
\caption{Orphaned Equipment vs Hosted Equipment}\label{\detokenize{docs/userguide/definingarchitecturalelements/floorplans/index-floor-plans:id14}}\end{figure}

Select a single piece of Equipment or multiple at a time.  Rooms and Floors will be highlighted.

Hover over a Room and click the Move/Add icon.

\begin{figure}[H]
\centering
\capstart

\noindent\sphinxincludegraphics{{move-equipment_3}.PNG}
\caption{Highlighted regions are areas where an Equipment can be placed}\label{\detokenize{docs/userguide/definingarchitecturalelements/floorplans/index-floor-plans:id15}}\end{figure}


\subsection{Architectural Elements}
\label{\detokenize{docs/userguide/index-user_guide:architectural-elements}}
{\hyperref[\detokenize{docs/userguide/definingarchitecturalelements/archelements/index-arch-elements::doc}]{\sphinxcrossref{\DUrole{doc}{Arch. Elements}}}} are a tabular representation of the Architectural entities of the model.
\phantomsection\label{\detokenize{docs/userguide/definingarchitecturalelements/archelements/index-arch-elements:arch-elements}}
Click on each of the tabs to view tables of each of the entity’s components.

\begin{figure}[H]
\centering
\capstart

\noindent\sphinxincludegraphics{{architectural_elements-1}.PNG}
\caption{Viewing Floors in the Architectural Elements Workspace}\label{\detokenize{docs/userguide/definingarchitecturalelements/archelements/index-arch-elements:id4}}\end{figure}

These tables have additional functions which allow the ability to create, copy, and export.


\subsubsection{Additional Commands}
\label{\detokenize{docs/userguide/definingarchitecturalelements/archelements/index-arch-elements:additional-commands}}\label{\detokenize{docs/userguide/definingarchitecturalelements/archelements/index-arch-elements::doc}}

\begin{savenotes}\sphinxattablestart
\centering
\begin{tabulary}{\linewidth}[t]{|T|T|}
\hline
\sphinxstyletheadfamily 
\sphinxstylestrong{Command}
&\sphinxstyletheadfamily 
\sphinxstylestrong{Description}
\\
\hline
Select All
&
Use CTRL+A to select all entities.
\\
\hline
Cut/Copy
&
Use CTRL+C to copy and CTRL+C to cut.
\\
\hline
Paste
&
Use CTRL+V to paste.
\\
\hline
Filtering
&
Filter items by using the sorting button.
\\
\hline
Exporting
&
Use the export button to export your model into .csv, .xml, or .json.  It is also possible to export content by copying and pasting into Excel.
\\
\hline
\end{tabulary}
\par
\sphinxattableend\end{savenotes}


\paragraph{Architectural Package}
\label{\detokenize{docs/userguide/definingarchitecturalelements/archelements/index-arch-elements:architectural-package}}\label{\detokenize{docs/userguide/definingarchitecturalelements/archelements/index-arch-elements:id1}}
Architectural Packages are used to model the load or power density of a group of architectural elements.

For example, a group of Floors could each have their Space Type designated as Office, which has a specific power density.  This group of Floors can be packaged as a load, and fed from distribution Equipment.

Select a group of Floors or Rooms.  Then, in the orange text box, enter a name for the Package and then click the (+) button.  This is packaging a group of Floors as a load.

\begin{figure}[H]
\centering
\capstart

\noindent\sphinxincludegraphics{{Architectural_Package-1}.PNG}
\caption{Creating an Architectural Package for a group of Floors}\label{\detokenize{docs/userguide/definingarchitecturalelements/archelements/index-arch-elements:id5}}\end{figure}

To view the Package, click the Architectural Package tab.  These loads can be attached to any distribution Equipment in the network.

\begin{figure}[H]
\centering
\capstart

\noindent\sphinxincludegraphics{{Architectural_Package-2}.PNG}
\caption{Viewing the Load of Architectural Packages}\label{\detokenize{docs/userguide/definingarchitecturalelements/archelements/index-arch-elements:id6}}\end{figure}


\paragraph{Load Allocation}
\label{\detokenize{docs/userguide/definingarchitecturalelements/archelements/index-arch-elements:load-allocation}}\label{\detokenize{docs/userguide/definingarchitecturalelements/archelements/index-arch-elements:id2}}
Load Allocations are used to supplement the Architectural Packages.  In addition to Floor or Room power densities, power can be allocated to specific Floors.

You may want to account for a load that only occurs on Floors of a specific Space Type.  For example, if you wanted to account for a 20 hp motor on every Office Floor, you would need to create a new Load Allocation.  Filter the Load Allocation by the Floor Space Type specified for Office.

\begin{figure}[H]
\centering
\capstart

\noindent\sphinxincludegraphics{{Load_Allocation-1}.PNG}
\caption{Creating a Load Allocation}\label{\detokenize{docs/userguide/definingarchitecturalelements/archelements/index-arch-elements:id7}}\end{figure}

Then use the (+) button to associate the Load Allocation with an Architectural Package.

\begin{figure}[H]
\centering
\capstart

\noindent\sphinxincludegraphics{{Load_Allocation-2}.PNG}
\caption{Allocating additional loads to an Apartment Package}\label{\detokenize{docs/userguide/definingarchitecturalelements/archelements/index-arch-elements:id8}}\end{figure}


\paragraph{Diversification}
\label{\detokenize{docs/userguide/definingarchitecturalelements/archelements/index-arch-elements:diversification}}\label{\detokenize{docs/userguide/definingarchitecturalelements/archelements/index-arch-elements:id3}}
Diversification allows you to create customizable diversity factors which can be applied to different sections or levels of the distribution system.

A Root Diversity is a factor applied to Root Level loads.  A Root Diversity cannot be less than zero and cannot be greater than the Distribution Diversity.

A Distribution Diversity is a factor applied to Distribution Level loads.

An End-of-Line Diversity is a factor applied to End-of-Line Loads.

\begin{figure}[H]
\centering
\capstart

\noindent\sphinxincludegraphics{{Diversification-1}.PNG}
\caption{Creating a Custom Diversification Class}\label{\detokenize{docs/userguide/definingarchitecturalelements/archelements/index-arch-elements:id9}}\end{figure}

In the example below, a Custom Diversity class called DIV A is created and applied to a small distribution network.

EOL (End-of-Line) , DIST. (Distribution), and ROOT loads have diversity factors of 1.0, 0.5, and 0.25, respectively, applied to them.

At the Diversity Position DIST., or DB-1, a factor of 0.5 is applied to loads L1 and L2, and summed together.  Therefore, the Net Load of DB-1 is 10 kVA.

At DB-2, the same diversities are applied to loads L3 and L4, and the Net Load is also 10 kVA.

At the Diversity Position ROOT, or MDB, a factor of 0.25 is applied to all of its loads, L1, L2, L3, and L4, and summed together.

If any downstream Equipment of MDB has a Load Override value, that value would be diversified instead of the connected load.

\begin{figure}[H]
\centering
\capstart

\noindent\sphinxincludegraphics{{Diversification-2}.PNG}
\caption{Electrical Distribution Network utilizing a Custom Diversification Class}\label{\detokenize{docs/userguide/definingarchitecturalelements/archelements/index-arch-elements:id10}}\end{figure}


\subsubsection{Modeling Residential Loads}
\label{\detokenize{docs/userguide/definingarchitecturalelements/archelements/index-arch-elements:modeling-residential-loads}}

\paragraph{Appliances}
\label{\detokenize{docs/userguide/definingarchitecturalelements/archelements/index-arch-elements:appliances}}
Appliances are used to calculate the load of residential projects and are grouped by classes as defined by the NEC.  Appliances are assigned to a Unit Type.


\paragraph{Unit Type}
\label{\detokenize{docs/userguide/definingarchitecturalelements/archelements/index-arch-elements:unit-type}}
Unit Types are used to group Appliances together in order to calculate the load of residential projects.  A Unit Type is assigned to an Apartment.

\begin{figure}[H]
\centering
\capstart

\noindent\sphinxincludegraphics{{Unit_Type}.PNG}
\caption{Assigning Appliances to a Unit Type}\label{\detokenize{docs/userguide/definingarchitecturalelements/archelements/index-arch-elements:id11}}\end{figure}


\paragraph{Apartment}
\label{\detokenize{docs/userguide/definingarchitecturalelements/archelements/index-arch-elements:apartment}}
An Apartment contains a Unit Type.  It also contains information regarding its location and loading information.  A group of Apartments form an Apartment Package.


\paragraph{Apartment Package}
\label{\detokenize{docs/userguide/definingarchitecturalelements/archelements/index-arch-elements:apartment-package}}
Apartment Packages are used to group Apartments together in order to calculate a load.

\begin{figure}[H]
\centering
\capstart

\noindent\sphinxincludegraphics{{Apartment_Package}.PNG}
\caption{Assigning Apartments to an Apartment Package}\label{\detokenize{docs/userguide/definingarchitecturalelements/archelements/index-arch-elements:id12}}\end{figure}


\subsubsection{Mechanical Entities}
\label{\detokenize{docs/userguide/definingarchitecturalelements/archelements/index-arch-elements:mechanical-entities}}
Coming soon.


\section{\sphinxstylestrong{Building the Electrical Model}}
\label{\detokenize{docs/userguide/index-user_guide:building-the-electrical-model}}\label{\detokenize{docs/userguide/index-user_guide:electrical-workspaces}}

\subsection{Overview}
\label{\detokenize{docs/userguide/index-user_guide:id7}}
There are three (3) main Workspaces to build the electrical system.

These include: {\hyperref[\detokenize{docs/userguide/buildingelectricalmodel/riser/index-riser:riser}]{\sphinxcrossref{\DUrole{std,std-ref}{Riser}}}}, {\hyperref[\detokenize{docs/userguide/buildingelectricalmodel/one-line/index-one-line:one-line}]{\sphinxcrossref{\DUrole{std,std-ref}{One-Line}}}}, {\hyperref[\detokenize{docs/userguide/buildingelectricalmodel/schedules/index-schedules:schedules}]{\sphinxcrossref{\DUrole{std,std-ref}{Schedules}}}}.

Reports of a model can be generated at any time using the {\hyperref[\detokenize{docs/userguide/buildingelectricalmodel/studies/index-studies:studies}]{\sphinxcrossref{\DUrole{std,std-ref}{Studies}}}} Workspace.


\subsubsection{Hosted vs. Unhosted Systems}
\label{\detokenize{docs/userguide/index-user_guide:hosted-vs-unhosted-systems}}
If a piece of Equipment is not fed from a source, then it is designated as Unhosted.  During the stages of design, it is common to model hypothetical scenarios before all information is known.  For example, as part of a larger project, an Architect may decide to create a space dedicated for a computer lab, but not yet decide on its exact location.  While the location is not yet finalized, you can still model an Unhosted System with the intent of attaching the system to the main network in the future.  Modeling Unhosted Systems allows flexibility when the source of a network is not yet known or if you want to create segregated distribution networks and tie them together at a later time.


\subsection{Riser}
\label{\detokenize{docs/userguide/index-user_guide:riser}}\phantomsection\label{\detokenize{docs/userguide/buildingelectricalmodel/riser/index-riser:riser}}
The Riser Workspace is an elevational representation of the distribution system.  It is used to depict wiring routes as they disperse through a vertically scaling project.

\begin{figure}[H]
\centering
\capstart

\noindent\sphinxincludegraphics{{riser-overview}.PNG}
\caption{Example Riser Diagram}\label{\detokenize{docs/userguide/buildingelectricalmodel/riser/index-riser:id3}}\end{figure}

Note that the interactions between the Riser, {\hyperref[\detokenize{docs/userguide/buildingelectricalmodel/one-line/index-one-line:one-line}]{\sphinxcrossref{\DUrole{std,std-ref}{One-Line}}}}, and {\hyperref[\detokenize{docs/userguide/buildingelectricalmodel/schedules/index-schedules:schedules}]{\sphinxcrossref{\DUrole{std,std-ref}{Schedules}}}} are similar.

\begin{figure}[H]
\centering
\capstart

\noindent\sphinxincludegraphics{{riser-selection}.PNG}
\caption{Selection Dial}\label{\detokenize{docs/userguide/buildingelectricalmodel/riser/index-riser:id4}}\end{figure}


\subsubsection{Navigation Overview}
\label{\detokenize{docs/userguide/buildingelectricalmodel/riser/index-riser:navigation-overview}}\label{\detokenize{docs/userguide/buildingelectricalmodel/riser/index-riser::doc}}

\paragraph{Floor Navigator}
\label{\detokenize{docs/userguide/buildingelectricalmodel/riser/index-riser:floor-navigator}}
On the left, is a Floor navigator.  It will highlight and navigate to the selected Floor.

\begin{figure}[H]
\centering
\capstart

\noindent\sphinxincludegraphics{{riser-floor_navigator}.PNG}
\caption{Cycle through Floors using the Floor Navigator}\label{\detokenize{docs/userguide/buildingelectricalmodel/riser/index-riser:id5}}\end{figure}

Floors are denoted by the dashed regions, while Rooms are denoted by the solid regions.

\begin{figure}[H]
\centering
\capstart

\noindent\sphinxincludegraphics{{riser-floor-1}.PNG}
\caption{Floors vs. Rooms}\label{\detokenize{docs/userguide/buildingelectricalmodel/riser/index-riser:id6}}\end{figure}


\subparagraph{Floor Elevations}
\label{\detokenize{docs/userguide/buildingelectricalmodel/riser/index-riser:floor-elevations}}\label{\detokenize{docs/userguide/buildingelectricalmodel/riser/index-riser:riser-floor-elevations}}
In the Riser, the elevations of each Floor are annotations which are not connected to the visual spacing or distance between floors.

In other words, moving the Floor will not change the elevation of the Floor.

\begin{figure}[H]
\centering
\capstart

\noindent\sphinxincludegraphics{{Riser_Elevations-1}.PNG}
\caption{Viewing the Floor Elevation in the Riser and the Arch. Elements}\label{\detokenize{docs/userguide/buildingelectricalmodel/riser/index-riser:id7}}\end{figure}

Moving the Floor will not affect the elevation.

\begin{figure}[H]
\centering
\capstart

\noindent\sphinxincludegraphics{{Riser_Elevations-2}.PNG}
\caption{Changing the visual spacing between Floors}\label{\detokenize{docs/userguide/buildingelectricalmodel/riser/index-riser:id8}}\end{figure}

Changing the Floor annotation will affect the lengths of feeders and voltage drop calculations.

\begin{figure}[H]
\centering
\capstart

\noindent\sphinxincludegraphics{{Riser_Elevations-3}.PNG}
\caption{The Riser Floor annotations are linked to the {\hyperref[\detokenize{docs/userguide/definingarchitecturalelements/archelements/index-arch-elements:arch-elements}]{\sphinxcrossref{\DUrole{std,std-ref}{Arch. Elements}}}}}\label{\detokenize{docs/userguide/buildingelectricalmodel/riser/index-riser:id9}}\end{figure}


\paragraph{Riser Toolbox}
\label{\detokenize{docs/userguide/buildingelectricalmodel/riser/index-riser:riser-toolbox}}\label{\detokenize{docs/userguide/buildingelectricalmodel/riser/index-riser:id1}}
On the right is the Riser Toolbox, which allows you to search for Equipment, control layers of the circuits between Equipment, and view {\hyperref[\detokenize{docs/userguide/buildingelectricalmodel/riser/index-riser:hidden-elements}]{\sphinxcrossref{\DUrole{std,std-ref}{Hidden Elements}}}}.

\begin{figure}[H]
\centering
\capstart

\noindent\sphinxincludegraphics{{riser-toolbox-1}.PNG}
\caption{Riser Toolbox}\label{\detokenize{docs/userguide/buildingelectricalmodel/riser/index-riser:id10}}\end{figure}

Click on the Filters button (layers icon) to open the Layers filter.

\begin{figure}[H]
\centering
\capstart

\noindent\sphinxincludegraphics{{riser-toolbox_filters-1}.PNG}
\caption{Filters allow greater flexibility with visibility}\label{\detokenize{docs/userguide/buildingelectricalmodel/riser/index-riser:id11}}\end{figure}

Right-click on a connection to change the layer of an Equipment and note the display.

\begin{figure}[H]
\centering
\capstart

\noindent\sphinxincludegraphics{{riser-toolbox_filters-2}.PNG}
\caption{Identifying a circuit as existing}\label{\detokenize{docs/userguide/buildingelectricalmodel/riser/index-riser:id12}}\end{figure}


\subparagraph{Hidden Elements}
\label{\detokenize{docs/userguide/buildingelectricalmodel/riser/index-riser:hidden-elements}}\label{\detokenize{docs/userguide/buildingelectricalmodel/riser/index-riser:id2}}
Hidden Elements are Architectural Entities or Equipment which are not being shown on the Riser.  To display them, simply click on these elements or click and drag them onto the Riser.

Not every element in the model needs to be shown on the Riser.  To hide an element, use right-click and Hide.

\begin{figure}[H]
\centering
\capstart

\noindent\sphinxincludegraphics{{riser-toolbox_hidden_elements}.PNG}
\caption{Filtering the Hidden Elements by using the search bar}\label{\detokenize{docs/userguide/buildingelectricalmodel/riser/index-riser:id13}}\end{figure}

It’s possible to draw circuits which are hidden from the Riser.  Hover over the arrow to see which element is hidden.

\begin{figure}[H]
\centering
\capstart

\noindent\sphinxincludegraphics{{riser-toolbox_hidden_elements_draw-1}.PNG}
\caption{Hover over the arrow to view hidden elements or to begin drawing their connection.}\label{\detokenize{docs/userguide/buildingelectricalmodel/riser/index-riser:id14}}\end{figure}

Click and drag the arrow to draw the new connection.

\begin{figure}[H]
\centering
\capstart

\noindent\sphinxincludegraphics{{riser-toolbox_hidden_elements_draw-2}.PNG}
\caption{Drawing Hidden Elements}\label{\detokenize{docs/userguide/buildingelectricalmodel/riser/index-riser:id15}}\end{figure}


\subsubsection{Add Architectural Elements}
\label{\detokenize{docs/userguide/buildingelectricalmodel/riser/index-riser:add-architectural-elements}}\label{\detokenize{docs/userguide/buildingelectricalmodel/riser/index-riser:riser-toolbox-arch-elements}}
To add Floors, Rooms, Architectural Elements, or electrical Equipment, use the toolbox at the top.

\begin{figure}[H]
\centering
\capstart

\noindent\sphinxincludegraphics{{riser-arch_toolbox}.PNG}
\caption{Adding Elements}\label{\detokenize{docs/userguide/buildingelectricalmodel/riser/index-riser:id16}}\end{figure}

Drag and drop elements from the toolbox to the Workspace to place elements.

\begin{figure}[H]
\centering
\capstart

\noindent\sphinxincludegraphics{{riser-arch_toolbox_add_arch}.PNG}
\caption{Adding Architectural Elements}\label{\detokenize{docs/userguide/buildingelectricalmodel/riser/index-riser:id17}}\end{figure}


\subsubsection{Add Electrical Equipment}
\label{\detokenize{docs/userguide/buildingelectricalmodel/riser/index-riser:add-electrical-equipment}}
To place Equipment, drag and drop elements from the toolbox into the Workspace.

\begin{figure}[H]
\centering
\capstart

\noindent\sphinxincludegraphics{{riser-arch_toolbox_add_equip-1}.PNG}
\caption{Adding Electrical Equipment}\label{\detokenize{docs/userguide/buildingelectricalmodel/riser/index-riser:id18}}\end{figure}

To place an Equipment in a Room, drag the Equipment into the Room region.

\begin{figure}[H]
\centering
\capstart

\noindent\sphinxincludegraphics{{riser-arch_toolbox_add_equip-2}.PNG}
\caption{Equipment placed in a Room}\label{\detokenize{docs/userguide/buildingelectricalmodel/riser/index-riser:id19}}\end{figure}


\subsubsection{Connecting Equipment}
\label{\detokenize{docs/userguide/buildingelectricalmodel/riser/index-riser:connecting-equipment}}
There are three different types of connections between Equipment: Outbound, Tie, and Infed.  These are also known as Load, Tie, and Source, respectively.

To create a connection between Equipment, select the Equipment.  Then choose the type of connection.  An outbound arrow will create an Outbound connection, indicating you are drawing a connection to a load.  An inward facing arrow will create an Infed connection, indicating that you are drawing a connection to a source.

\begin{figure}[H]
\centering
\capstart

\noindent\sphinxincludegraphics{{riser-create_connection-1}.PNG}
\caption{Using the Selection Dial to create connections}\label{\detokenize{docs/userguide/buildingelectricalmodel/riser/index-riser:id20}}\end{figure}

Draw out the connection using the mouse and use Enter to create an Equipment.

\begin{figure}[H]
\centering
\capstart

\noindent\sphinxincludegraphics{{riser-create_connection-2}.PNG}
\caption{Use Enter to place Equipment}\label{\detokenize{docs/userguide/buildingelectricalmodel/riser/index-riser:id21}}\end{figure}


\paragraph{Resetting Connections}
\label{\detokenize{docs/userguide/buildingelectricalmodel/riser/index-riser:resetting-connections}}
To quickly redraw a connection between equipment, use the reset command.  Right-click on the circuit.  Then choose Reset.


\subsubsection{Copying Equipment}
\label{\detokenize{docs/userguide/buildingelectricalmodel/riser/index-riser:copying-equipment}}
Select the Equipment and use CTRL+C to copy.

\begin{figure}[H]
\centering
\capstart

\noindent\sphinxincludegraphics{{riser-copy_equipment-1}.PNG}
\caption{Copying multiple pieces of Equipment}\label{\detokenize{docs/userguide/buildingelectricalmodel/riser/index-riser:id22}}\end{figure}

Use CTRL+V to paste.

\begin{figure}[H]
\centering
\capstart

\noindent\sphinxincludegraphics{{riser-copy_equipment-2}.PNG}
\caption{Pasting multiple pieces of Equipment}\label{\detokenize{docs/userguide/buildingelectricalmodel/riser/index-riser:id23}}\end{figure}


\subsubsection{Moving Equipment}
\label{\detokenize{docs/userguide/buildingelectricalmodel/riser/index-riser:moving-equipment}}
It is possible to move Equipment by an individual or a group basis.

Select a single Equipment or select multiple by using CTRL+Click, or drag and drop a selection box.

\begin{figure}[H]
\centering
\capstart

\noindent\sphinxincludegraphics{{riser-move_equipment-1}.PNG}
\caption{Selecting multiple pieces of Equipment}\label{\detokenize{docs/userguide/buildingelectricalmodel/riser/index-riser:id24}}\end{figure}

Then click and drag to move Equipment.


\subsubsection{Resizing Equipment}
\label{\detokenize{docs/userguide/buildingelectricalmodel/riser/index-riser:resizing-equipment}}
To resize equipment, select on a piece of equipment.  Use the grips to change the size.

\begin{figure}[H]
\centering
\capstart

\noindent\sphinxincludegraphics{{riser-resizing}.PNG}
\caption{Resizing an Equipment}\label{\detokenize{docs/userguide/buildingelectricalmodel/riser/index-riser:id25}}\end{figure}


\subsubsection{Navigate}
\label{\detokenize{docs/userguide/buildingelectricalmodel/riser/index-riser:navigate}}
Similarly to the {\hyperref[\detokenize{docs/userguide/buildingelectricalmodel/one-line/index-one-line:one-line}]{\sphinxcrossref{\DUrole{std,std-ref}{One-Line}}}} and the {\hyperref[\detokenize{docs/userguide/buildingelectricalmodel/schedules/index-schedules:schedules}]{\sphinxcrossref{\DUrole{std,std-ref}{Schedules}}}}, it is possible to jump between Workspaces.

A piece of Equipment may be found on multiple Workspaces.

Selecting an Equipment and using the Navigate button to jump to another Workspace.

\begin{figure}[H]
\centering
\capstart

\noindent\sphinxincludegraphics{{Riser-Navigate-1}.PNG}
\caption{Using the Navigate button}\label{\detokenize{docs/userguide/buildingelectricalmodel/riser/index-riser:id26}}\end{figure}

Choose the Workspace to jump to.

\begin{figure}[H]
\centering
\capstart

\noindent\sphinxincludegraphics{{Riser-Navigate-2}.PNG}
\caption{An Equipment can be Found on Multiple Workspaces}\label{\detokenize{docs/userguide/buildingelectricalmodel/riser/index-riser:id27}}\end{figure}

Navigating to the One-Line will expand the network and select the node.

\begin{figure}[H]
\centering
\capstart

\noindent\sphinxincludegraphics{{Riser-Navigate-3}.PNG}
\caption{Navigating to the One-Line}\label{\detokenize{docs/userguide/buildingelectricalmodel/riser/index-riser:id28}}\end{figure}


\subsubsection{Converting to Bus Duct}
\label{\detokenize{docs/userguide/buildingelectricalmodel/riser/index-riser:converting-to-bus-duct}}
To convert pipe and wire connections to a bus duct, first delete the existing circuits on the Riser.

Select a circuit and use the delete (Del) key.

\begin{figure}[H]
\centering
\capstart

\noindent\sphinxincludegraphics{{Riser-ConvertToBusDuct-1}.PNG}
\caption{Deleting Existing Circuits}\label{\detokenize{docs/userguide/buildingelectricalmodel/riser/index-riser:id29}}\end{figure}

Delete all of the existing circuits which will be fed from the Bus Duct.

\begin{figure}[H]
\centering
\capstart

\noindent\sphinxincludegraphics{{Riser-ConvertToBusDuct-2}.PNG}
\caption{Deleting the Existing Circuits}\label{\detokenize{docs/userguide/buildingelectricalmodel/riser/index-riser:id30}}\end{figure}

Create a Bus Duct by dragging and dropping it onto the Riser.

\begin{figure}[H]
\centering
\capstart

\noindent\sphinxincludegraphics{{Riser-ConvertToBusDuct-3}.PNG}
\caption{Creating a Bus Duct}\label{\detokenize{docs/userguide/buildingelectricalmodel/riser/index-riser:id31}}\end{figure}

Then feed it from MDB by creating an Inbound connection.

Create an Inbound connection by clicking the arrow facing inwards.

\begin{figure}[H]
\centering
\capstart

\noindent\sphinxincludegraphics{{Riser-ConvertToBusDuct-4}.PNG}
\caption{Creating a Bus Duct}\label{\detokenize{docs/userguide/buildingelectricalmodel/riser/index-riser:id32}}\end{figure}

Draw the connection to the source.

\begin{figure}[H]
\centering
\capstart

\noindent\sphinxincludegraphics{{Riser-ConvertToBusDuct-5}.PNG}
\caption{Connecting a Bus Duct to a Source}\label{\detokenize{docs/userguide/buildingelectricalmodel/riser/index-riser:id33}}\end{figure}

Select your Bus Duct, resize it, and use the Outbound arrow to draw connections to your loads.

\begin{figure}[H]
\centering
\capstart

\noindent\sphinxincludegraphics{{Riser-ConvertToBusDuct-6}.PNG}
\caption{Connecting to Existing Equipment}\label{\detokenize{docs/userguide/buildingelectricalmodel/riser/index-riser:id34}}\end{figure}

Draw the remaining electrical connections.

\begin{figure}[H]
\centering
\capstart

\noindent\sphinxincludegraphics{{Riser-ConvertToBusDuct-7}.PNG}
\caption{Connecting to Existing Equipment}\label{\detokenize{docs/userguide/buildingelectricalmodel/riser/index-riser:id35}}\end{figure}


\subsection{One-Line}
\label{\detokenize{docs/userguide/index-user_guide:one-line}}\phantomsection\label{\detokenize{docs/userguide/buildingelectricalmodel/one-line/index-one-line:one-line}}
The {\hyperref[\detokenize{docs/userguide/buildingelectricalmodel/one-line/index-one-line:one-line}]{\sphinxcrossref{\DUrole{std,std-ref}{One-Line}}}} represents power flow of the distribution system shown from top to bottom, and generally consists of Sources, Distribution Equipment, and Loads.  Types of sources include a Utility or a Generator.

Using the {\hyperref[\detokenize{docs/userguide/index-user_guide:workspace-toolbar}]{\sphinxcrossref{\DUrole{std,std-ref}{Workspace Toolbar}}}} on the left, click the One-Line icon to open the Workspace.

\index{How to Create a Source@\spxentry{How to Create a Source}}\ignorespaces 

\subsubsection{Add a Source}
\label{\detokenize{docs/userguide/buildingelectricalmodel/one-line/index-one-line:add-a-source}}\label{\detokenize{docs/userguide/buildingelectricalmodel/one-line/index-one-line:one-line-adding-a-source}}\label{\detokenize{docs/userguide/buildingelectricalmodel/one-line/index-one-line:index-0}}\label{\detokenize{docs/userguide/buildingelectricalmodel/one-line/index-one-line::doc}}
Use the Setup Wizard in the top right and select Create a Source.

Sources specify the level of short circuit current (SCC) that is potentially available in faulted systems.  Short circuit current contributions could stem from the generator’s sub-transient reactance or the utility takeoff secondary terminals.

\begin{figure}[H]
\centering
\capstart

\noindent\sphinxincludegraphics{{one-line-create_source-1}.PNG}
\caption{Using the Setup Wizard}\label{\detokenize{docs/userguide/buildingelectricalmodel/one-line/index-one-line:id2}}\end{figure}

Using the New Network wizard, create a Utility Equipment.

When using the New Network wizard, only the minimum pieces of information are required to create a piece of Equipment.  This information will vary from Equipment to Equipment.  For example, when creating a transformer the secondary voltage must be specified, but for a mechanical load only the size of the motor is required.  More detailed properties can be specified and modified later.

Or right-click inside the Workspace, click Add Source, then click Utility Source.

Note that the ability to add Unhosted Equipment is available.


\paragraph{Short Circuit Current}
\label{\detokenize{docs/userguide/buildingelectricalmodel/one-line/index-one-line:short-circuit-current}}\label{\detokenize{docs/userguide/buildingelectricalmodel/one-line/index-one-line:one-line-scc}}
Select the Utility source.  Under the “Miscellaneous” property grouping, enter the value under Available SCC (kA).

\begin{figure}[H]
\centering
\capstart

\noindent\sphinxincludegraphics{{one-line-SCC}.PNG}
\caption{Entering Available SCC (kA)}\label{\detokenize{docs/userguide/buildingelectricalmodel/one-line/index-one-line:id3}}\end{figure}


\subsubsection{Add Equipment}
\label{\detokenize{docs/userguide/buildingelectricalmodel/one-line/index-one-line:add-equipment}}\label{\detokenize{docs/userguide/buildingelectricalmodel/one-line/index-one-line:one-line-adding-equipment}}
Click on the Utility and the Selection Dial will display a ring of options.  Selected Equipment will be highlighted by a purple circle, with additional options to Add, Copy, Paste, Delete, and Navigate to other Workspaces.  Click the + button to Add Equipment.

\begin{figure}[H]
\centering
\capstart

\noindent\sphinxincludegraphics{{one-line-add_equipment}.PNG}
\caption{Selection Dial}\label{\detokenize{docs/userguide/buildingelectricalmodel/one-line/index-one-line:id4}}\end{figure}

Select an Equipment type from the dropdown menu.  Give the Equipment a name.  In this case, specify a {\hyperref[\detokenize{docs/faq:load-capacity}]{\sphinxcrossref{\DUrole{std,std-ref}{Load Capacity}}}}, and click Select.

\begin{figure}[H]
\centering
\capstart

\noindent\sphinxincludegraphics{{one-line-add_dist_board}.PNG}
\caption{New Network Wizard}\label{\detokenize{docs/userguide/buildingelectricalmodel/one-line/index-one-line:id5}}\end{figure}

\index{How to Copy Equipment - One-Line@\spxentry{How to Copy Equipment - One-Line}}\ignorespaces 

\subsubsection{Copy/Paste Equipment}
\label{\detokenize{docs/userguide/buildingelectricalmodel/one-line/index-one-line:copy-paste-equipment}}\label{\detokenize{docs/userguide/buildingelectricalmodel/one-line/index-one-line:one-line-copying-equipment}}\label{\detokenize{docs/userguide/buildingelectricalmodel/one-line/index-one-line:index-1}}
To copy Equipment, select the Equipment.  Then, click Copy or use CTRL+C to copy.  The selection will highlight pink and be added to the clipboard.

Then select the Equipment you want to create a pasted copy of, and click Paste or use CTRL+V.

\begin{figure}[H]
\centering
\capstart

\noindent\sphinxincludegraphics{{one-line-copy_equipment}.PNG}
\caption{Copying an Equipment will copy its entire downstream network}\label{\detokenize{docs/userguide/buildingelectricalmodel/one-line/index-one-line:id6}}\end{figure}


\subsubsection{Delete Equipment/Delete Network}
\label{\detokenize{docs/userguide/buildingelectricalmodel/one-line/index-one-line:delete-equipment-delete-network}}
To delete Equipment, select the Equipment.  Then Click Delete (trash symbol) or use DEL to delete.

If the selected Equipment is feeding downstream Equipment, you have the option to either delete the selected Equipment or the entire network.  Deleting Selected Equipment will only delete what is selected and any downstream equipment or children of the Selected Equipment will be disconnected and considered Unhosted.

\begin{figure}[H]
\centering
\capstart

\noindent\sphinxincludegraphics{{one-line-delete_network}.PNG}
\caption{Deleting the selected Equipment or Entire Network}\label{\detokenize{docs/userguide/buildingelectricalmodel/one-line/index-one-line:id7}}\end{figure}


\subsubsection{Dragging/Rehosting Equipment}
\label{\detokenize{docs/userguide/buildingelectricalmodel/one-line/index-one-line:dragging-rehosting-equipment}}\label{\detokenize{docs/userguide/buildingelectricalmodel/one-line/index-one-line:one-line-rehosting}}
To redirect an Equipment’s source, click and drag the Equipment from its current source to a different source.

\begin{figure}[H]
\centering
\capstart

\noindent\sphinxincludegraphics{{one-line-rehost}.PNG}
\caption{Rehosting MTR-2 to DB-1}\label{\detokenize{docs/userguide/buildingelectricalmodel/one-line/index-one-line:id8}}\end{figure}


\subsubsection{Navigate}
\label{\detokenize{docs/userguide/buildingelectricalmodel/one-line/index-one-line:navigate}}
Navigate grants the ability to jump between Workspaces based on the current selection.

Some examples of navigation include viewing an Equipment’s {\hyperref[\detokenize{docs/userguide/buildingelectricalmodel/schedules/index-schedules:schedules}]{\sphinxcrossref{\DUrole{std,std-ref}{Schedule}}}}, location on the {\hyperref[\detokenize{docs/userguide/buildingelectricalmodel/riser/index-riser:riser}]{\sphinxcrossref{\DUrole{std,std-ref}{Riser}}}}, location on the {\hyperref[\detokenize{docs/userguide/definingarchitecturalelements/floorplans/index-floor-plans:floor-plans}]{\sphinxcrossref{\DUrole{std,std-ref}{Floor Plans}}}}, or the {\hyperref[\detokenize{docs/userguide/buildingelectricalmodel/studies/index-studies:studies}]{\sphinxcrossref{\DUrole{std,std-ref}{Studies}}}} Workspace.

\begin{figure}[H]
\centering
\capstart

\noindent\sphinxincludegraphics{{one-line-navigate-1}.PNG}
\caption{Using the Selection Dial to navigate to other Workspaces}\label{\detokenize{docs/userguide/buildingelectricalmodel/one-line/index-one-line:id9}}\end{figure}


\subsubsection{Expanding/Collapsing Equipment}
\label{\detokenize{docs/userguide/buildingelectricalmodel/one-line/index-one-line:expanding-collapsing-equipment}}
Sections of the distribution network can be expanded or collapsed on a group basis by using Expand All/Collapse All.

\begin{figure}[H]
\centering
\capstart

\noindent\sphinxincludegraphics{{one-line-expand_collapse-1}.PNG}
\caption{Using Expand All/Collapse All to visually maneuver the distribution network}\label{\detokenize{docs/userguide/buildingelectricalmodel/one-line/index-one-line:id10}}\end{figure}

Another way is by clicking on a distribution node, or double-clicking on the Equipment itself.

\begin{figure}[H]
\centering
\capstart

\noindent\sphinxincludegraphics{{one-line-expand_collapse-2}.PNG}
\caption{Clicking on a distribution node individually expands or collapses the network}\label{\detokenize{docs/userguide/buildingelectricalmodel/one-line/index-one-line:id11}}\end{figure}

\begin{figure}[H]
\centering
\capstart

\noindent\sphinxincludegraphics{{one-line-expand_collapse-2b}.PNG}
\caption{A distribution node fills when it is fully expanded}\label{\detokenize{docs/userguide/buildingelectricalmodel/one-line/index-one-line:id12}}\end{figure}


\subsubsection{Changing Multiple Equipment Properties}
\label{\detokenize{docs/userguide/buildingelectricalmodel/one-line/index-one-line:changing-multiple-equipment-properties}}
It is possible to change a property which is common across multiple elements.

First, drag a box to select multiple elements, or use CTRL+Click to select each element.

\begin{figure}[H]
\centering
\capstart

\noindent\sphinxincludegraphics{{one-line-multi-property-1}.PNG}
\caption{Selecting multiple motors by dragging and dropping a selection box}\label{\detokenize{docs/userguide/buildingelectricalmodel/one-line/index-one-line:id13}}\end{figure}

Then change a property such as Conductor Material from copper to aluminum.

\begin{figure}[H]
\centering
\capstart

\noindent\sphinxincludegraphics{{one-line-multi-property-2}.PNG}
\caption{Using the Properties Explorer to change the Conductor Material}\label{\detokenize{docs/userguide/buildingelectricalmodel/one-line/index-one-line:id14}}\end{figure}

Note that since Design Assistance is on, the circuit’s code-minimum values are recalculated.

\begin{figure}[H]
\centering
\capstart

\noindent\sphinxincludegraphics{{one-line-multi-property-3}.PNG}
\caption{Viewing circuit properties as a result of changing the conductor material}\label{\detokenize{docs/userguide/buildingelectricalmodel/one-line/index-one-line:id15}}\end{figure}


\subsubsection{Reset to Code Minimum}
\label{\detokenize{docs/userguide/buildingelectricalmodel/one-line/index-one-line:reset-to-code-minimum}}\label{\detokenize{docs/userguide/buildingelectricalmodel/one-line/index-one-line:one-line-reset-to-code-minimum}}
It is possible to manually modify circuit elements which cause a violation of safety codes and standards.

To recalculate or reset the values of a circuit to code-minimum values, right-click on an Equipment and use Reset to Code Minimum.

\begin{figure}[H]
\centering
\capstart

\noindent\sphinxincludegraphics{{one-line-reset_to_code_minimum}.PNG}
\caption{Using Reset to Code Minimum}\label{\detokenize{docs/userguide/buildingelectricalmodel/one-line/index-one-line:id16}}\end{figure}


\subsubsection{Workspace Toolbox}
\label{\detokenize{docs/userguide/buildingelectricalmodel/one-line/index-one-line:workspace-toolbox}}
Utility functions like searching, additional viewing properties, or calculation settings can be found on the Workspace Toolbox on the top toolbar.

\begin{figure}[H]
\centering
\capstart

\noindent\sphinxincludegraphics{{one-line-workspace_toolbox}.PNG}
\caption{One-Line Workspace Toolbox}\label{\detokenize{docs/userguide/buildingelectricalmodel/one-line/index-one-line:id17}}\end{figure}


\paragraph{Searching}
\label{\detokenize{docs/userguide/buildingelectricalmodel/one-line/index-one-line:searching}}
To search for Equipment, click the magnifying glass in the top left.  Start to type the name of an Equipment.  A dropdown will appear with any Equipment matching the specified name.  Select the Equipment, and the Workspace will navigate to the associated Equipment.

\begin{figure}[H]
\centering
\capstart

\noindent\sphinxincludegraphics{{one-line-searching}.PNG}
\caption{Searching for Equipment}\label{\detokenize{docs/userguide/buildingelectricalmodel/one-line/index-one-line:id18}}\end{figure}


\paragraph{Load Calculations}
\label{\detokenize{docs/userguide/buildingelectricalmodel/one-line/index-one-line:load-calculations}}\label{\detokenize{docs/userguide/buildingelectricalmodel/one-line/index-one-line:one-line-load-calculations}}\begin{itemize}
\item {} 
\sphinxstylestrong{Normal:} Calculations are based on the {\hyperref[\detokenize{docs/faq:net-load}]{\sphinxcrossref{\DUrole{std,std-ref}{Net Load}}}}.

\item {} 
\sphinxstylestrong{Board Capacity - 80\%} Calculations are based on 80\% of the board’s {\hyperref[\detokenize{docs/faq:load-capacity}]{\sphinxcrossref{\DUrole{std,std-ref}{Load Capacity}}}}.

\item {} 
\sphinxstylestrong{Board Capacity - 60\%} Calculations are based on 60\% of the board’s {\hyperref[\detokenize{docs/faq:load-capacity}]{\sphinxcrossref{\DUrole{std,std-ref}{Load Capacity}}}}.

\end{itemize}


\paragraph{Property Tags/Quick Views}
\label{\detokenize{docs/userguide/buildingelectricalmodel/one-line/index-one-line:property-tags-quick-views}}\label{\detokenize{docs/userguide/buildingelectricalmodel/one-line/index-one-line:property-tags}}
Property Tags can be applied to assist with design or network visualization.  They provide flexibility with viewing specific properties of the model.

Click the tag symbol in the upper left of the Workspace Toolbox.

Quick Views are preset property groupings such as Voltage Drop, Loading, Load Diversification, and Circuit Routing.

\begin{figure}[H]
\centering
\capstart

\noindent\sphinxincludegraphics{{one-line-property_tags_quick_views}.PNG}
\caption{Selecting the Voltage Drop Quick View}\label{\detokenize{docs/userguide/buildingelectricalmodel/one-line/index-one-line:id19}}\end{figure}

The number is an indicator of how many Property Tags are being displayed.

Use the Clear All button to clear the display of Property Tags.


\subparagraph{Assigning Room Locations}
\label{\detokenize{docs/userguide/buildingelectricalmodel/one-line/index-one-line:assigning-room-locations}}
The distance between two pieces of Equipment are determined by their Room location via an orthogonal route.  Open the Property Tags, and select Room.

\begin{figure}[H]
\centering
\capstart

\noindent\sphinxincludegraphics{{one-line-assigning-room-1}.PNG}
\caption{Using Property Tags to assign a Room}\label{\detokenize{docs/userguide/buildingelectricalmodel/one-line/index-one-line:id20}}\end{figure}

Assign a Room by clicking in the text box.  If no Rooms are available, create them using the {\hyperref[\detokenize{docs/userguide/index-user_guide:architectural-workspaces}]{\sphinxcrossref{\DUrole{std,std-ref}{Architectural Workspaces}}}}.

\begin{figure}[H]
\centering
\capstart

\noindent\sphinxincludegraphics{{one-line-assigning-room-2}.PNG}
\caption{Using Property Tags to assign a Room}\label{\detokenize{docs/userguide/buildingelectricalmodel/one-line/index-one-line:id21}}\end{figure}

For the Distribution Board, MDB-1, note the {\hyperref[\detokenize{docs/userguide/index-user_guide:calculated-length}]{\sphinxcrossref{\DUrole{std,std-ref}{Calc. Length}}}} before and after a Room location is assigned.

\begin{figure}[H]
\centering
\capstart

\noindent\sphinxincludegraphics{{one-line-assigning-room-3}.PNG}
\caption{Select a Room by clicking in the text box}\label{\detokenize{docs/userguide/buildingelectricalmodel/one-line/index-one-line:id22}}\end{figure}


\subparagraph{Routing Through a Riser Shaft}
\label{\detokenize{docs/userguide/buildingelectricalmodel/one-line/index-one-line:routing-through-a-riser-shaft}}
To route through the Riser, add a Property Tag for Riser under Network Properties.

\begin{figure}[H]
\centering
\capstart

\noindent\sphinxincludegraphics{{One-Line-Assign-Riser-1}.PNG}
\caption{Routing Through a Riser}\label{\detokenize{docs/userguide/buildingelectricalmodel/one-line/index-one-line:id23}}\end{figure}

Assign a Riser and note the updated length.

\begin{figure}[H]
\centering
\capstart

\noindent\sphinxincludegraphics{{One-Line-Assign-Riser-2}.PNG}
\caption{Routing Through a Riser}\label{\detokenize{docs/userguide/buildingelectricalmodel/one-line/index-one-line:id24}}\end{figure}


\paragraph{One-Line View Selectors}
\label{\detokenize{docs/userguide/buildingelectricalmodel/one-line/index-one-line:one-line-view-selectors}}
Different Views such as Isolated Systems and Load Flow can be applied to aid the designer.

By default, Normal is selected.


\subparagraph{Load Flow}
\label{\detokenize{docs/userguide/buildingelectricalmodel/one-line/index-one-line:load-flow}}
Load Flow is recommended when studying how a system is loaded.

\begin{figure}[H]
\centering
\capstart

\noindent\sphinxincludegraphics{{one-line-load_flow}.PNG}
\caption{Using Load Flow view to diagnose loading concerns of the network.}\label{\detokenize{docs/userguide/buildingelectricalmodel/one-line/index-one-line:id25}}\end{figure}


\subparagraph{Isolated Systems}
\label{\detokenize{docs/userguide/buildingelectricalmodel/one-line/index-one-line:isolated-systems}}
Isolated Systems is recommended when viewing complex, redundantally distributed systems.

\begin{figure}[H]
\centering
\capstart

\noindent\sphinxincludegraphics{{one-line-isolated_systems-1}.PNG}
\caption{Using Isolated Systems to study loading as a result of different sources}\label{\detokenize{docs/userguide/buildingelectricalmodel/one-line/index-one-line:id26}}\end{figure}

Click on the arrow buttons to jump to the section of the distribution network.

\begin{figure}[H]
\centering
\capstart

\noindent\sphinxincludegraphics{{one-line-isolated_systems-2}.PNG}
\caption{Navigating to a different section of the network}\label{\detokenize{docs/userguide/buildingelectricalmodel/one-line/index-one-line:id27}}\end{figure}


\subsubsection{Creating a Transfer Switch}
\label{\detokenize{docs/userguide/buildingelectricalmodel/one-line/index-one-line:creating-a-transfer-switch}}\label{\detokenize{docs/userguide/buildingelectricalmodel/one-line/index-one-line:one-line-transfer-switch}}
Transfer switches are connected to a primary and secondary source of power.  To create a transfer switch, click Add Equipment and choose ATS/STS.

\begin{figure}[H]
\centering
\capstart

\noindent\sphinxincludegraphics{{one-line-transfer_switch-1}.PNG}
\caption{Using Selection Dial to Add Equipment}\label{\detokenize{docs/userguide/buildingelectricalmodel/one-line/index-one-line:id28}}\end{figure}

\begin{figure}[H]
\centering
\capstart

\noindent\sphinxincludegraphics{{one-line-transfer_switch-2}.PNG}
\caption{Creating a transfer switch}\label{\detokenize{docs/userguide/buildingelectricalmodel/one-line/index-one-line:id29}}\end{figure}

To connect the secondary source of power, choose another distribution Equipment.

Then click Add Equipment and select an ATS/STS from the Existing dropdown menu.

\begin{figure}[H]
\centering
\capstart

\noindent\sphinxincludegraphics{{one-line-transfer_switch-3}.PNG}
\caption{Connecting to an existing transfer switch}\label{\detokenize{docs/userguide/buildingelectricalmodel/one-line/index-one-line:id30}}\end{figure}


\subsubsection{Settings}
\label{\detokenize{docs/userguide/buildingelectricalmodel/one-line/index-one-line:settings}}
Toggle the visibility settings of OCPD’s by clicking on Show OCPD.

\begin{figure}[H]
\centering
\capstart

\noindent\sphinxincludegraphics{{one-line-ocpd_settings}.PNG}
\caption{Showing OCPDs on the One-Line}\label{\detokenize{docs/userguide/buildingelectricalmodel/one-line/index-one-line:id31}}\end{figure}

Open an OCPD by clicking on the OCPD.

\begin{figure}[H]
\centering
\capstart

\noindent\sphinxincludegraphics{{one-line-openocpd}.PNG}
\caption{An open breaker denotes an open circuit}\label{\detokenize{docs/userguide/buildingelectricalmodel/one-line/index-one-line:id32}}\end{figure}

\index{How do I model a load shedding sequence?@\spxentry{How do I model a load shedding sequence?}}\ignorespaces 
\index{How do I use the Scenario Manager?@\spxentry{How do I use the Scenario Manager?}}\ignorespaces 

\paragraph{Scenario Manager}
\label{\detokenize{docs/userguide/buildingelectricalmodel/one-line/index-one-line:scenario-manager}}\label{\detokenize{docs/userguide/buildingelectricalmodel/one-line/index-one-line:index-3}}\label{\detokenize{docs/userguide/buildingelectricalmodel/one-line/index-one-line:id1}}
The Scenario Manager can be used to model different scenarios representing the state of protective devices.  It is generally used in conjunction with the One-Line.

For example, a designer may want to perform a load flow study of their electrical system as certain protective devices are opened or closed.

Open the Scenario Manager and show the OCPD’s on the One-Line.

Right-click on a protective device to add it to a Scenario.

\begin{figure}[H]
\centering
\capstart

\noindent\sphinxincludegraphics{{one-line-scenario_manager-1}.PNG}
\caption{Adding a protective device to the Scenario Manager}\label{\detokenize{docs/userguide/buildingelectricalmodel/one-line/index-one-line:id33}}\end{figure}

In each Scenario, toggle the different states of protective devices, and also toggle between different Scenarios as shown below.  Note that the color of live Equipment changes when a protective device is opened or closed.

\begin{figure}[H]
\centering
\capstart

\noindent\sphinxincludegraphics{{one-line-scenario_manager-2}.PNG}
\caption{Scenario 1 with Sub-Scenario 1 Active}\label{\detokenize{docs/userguide/buildingelectricalmodel/one-line/index-one-line:id34}}\end{figure}

\begin{figure}[H]
\centering
\capstart

\noindent\sphinxincludegraphics{{one-line-scenario_manager-3}.PNG}
\caption{Scenario 1 with Sub-Scenario 2 Active}\label{\detokenize{docs/userguide/buildingelectricalmodel/one-line/index-one-line:id35}}\end{figure}


\subsubsection{Bus Duct}
\label{\detokenize{docs/userguide/buildingelectricalmodel/one-line/index-one-line:bus-duct}}\label{\detokenize{docs/userguide/buildingelectricalmodel/one-line/index-one-line:one-line-bus-duct}}
Select an Equipment.  Then select Add Equipment to create a Bus Duct.

\begin{figure}[H]
\centering
\capstart

\noindent\sphinxincludegraphics{{one-line-bus_duct-1}.PNG}
\caption{Using the New Network Wizard to create a Bus Duct}\label{\detokenize{docs/userguide/buildingelectricalmodel/one-line/index-one-line:id36}}\end{figure}

Visually, the representation of a Bus Duct is misleading and will be changed in an upcoming update.  See {\hyperref[\detokenize{docs/faq:bus-duct-calculations}]{\sphinxcrossref{\DUrole{std,std-ref}{here}}}} for a reference to how loading and voltage drop calculations apply to a Bus Duct.

\begin{figure}[H]
\centering
\capstart

\noindent\sphinxincludegraphics{{one-line-bus_duct-2}.PNG}
\caption{A Bus Duct with Panelboards as branch loads}\label{\detokenize{docs/userguide/buildingelectricalmodel/one-line/index-one-line:id37}}\end{figure}


\subsection{Schedules}
\label{\detokenize{docs/userguide/index-user_guide:schedules}}\phantomsection\label{\detokenize{docs/userguide/buildingelectricalmodel/schedules/index-schedules:schedules}}
Schedules are a tabular representation of the distribution system.  This is a great environment to rapidly create distribution systems.  Schedules are primarily used for Distribution Boards, Switchboards, and Panelboards.  Not every piece of Equipment has a dedicated Schedule.

The top portion of the Workspace shows all open Schedules.  A selected item will highlight on the corresponding Schedule.  Schedules can be exported to AutoCAD or Excel.

\begin{figure}[H]
\centering
\capstart

\noindent\sphinxincludegraphics{{schedules-1}.PNG}
\caption{Open Schedules and selecting circuit breaker options}\label{\detokenize{docs/userguide/buildingelectricalmodel/schedules/index-schedules:id1}}\end{figure}


\subsubsection{Open Schedule}
\label{\detokenize{docs/userguide/buildingelectricalmodel/schedules/index-schedules:open-schedule}}\label{\detokenize{docs/userguide/buildingelectricalmodel/schedules/index-schedules::doc}}
To open an existing Schedule use the Add (+) button.

\begin{figure}[H]
\centering
\capstart

\noindent\sphinxincludegraphics{{schedules-add_schedule}.PNG}
\caption{Using the Add button to open Schedules}\label{\detokenize{docs/userguide/buildingelectricalmodel/schedules/index-schedules:id2}}\end{figure}

\index{Schedules - Adding@\spxentry{Schedules - Adding}}\index{Deleting@\spxentry{Deleting}}\index{Editing Equipment@\spxentry{Editing Equipment}}\ignorespaces 

\subsubsection{Adding, Deleting, Editing Equipment}
\label{\detokenize{docs/userguide/buildingelectricalmodel/schedules/index-schedules:adding-deleting-editing-equipment}}\label{\detokenize{docs/userguide/buildingelectricalmodel/schedules/index-schedules:schedules-copying-equipment}}\label{\detokenize{docs/userguide/buildingelectricalmodel/schedules/index-schedules:index-0}}
Clicking on a circuit number selects a circuit.  With an empty circuit selected, note the options to Add Equipment (+ icon), or Delete (trash icon), or Edit Properties (gear icon).

\begin{figure}[H]
\centering
\capstart

\noindent\sphinxincludegraphics{{schedules-add-delete-edit}.PNG}
\caption{Editing a space on a Schedule}\label{\detokenize{docs/userguide/buildingelectricalmodel/schedules/index-schedules:id3}}\end{figure}

Deleting a circuit that has a connected load will prompt you to delete the selected Equipment or the entire downstream network.  The OCPD will reset to the smallest available size.

\begin{figure}[H]
\centering
\capstart

\noindent\sphinxincludegraphics{{schedules-delete_circuit-1}.PNG}
\caption{Deleting the selected Equipment or its entire downstream network}\label{\detokenize{docs/userguide/buildingelectricalmodel/schedules/index-schedules:id4}}\end{figure}

\begin{figure}[H]
\centering
\capstart

\noindent\sphinxincludegraphics{{schedules-delete_circuit-2}.PNG}
\caption{Deleting a circuit with a connected load resets the OCPD to the smallest possible value}\label{\detokenize{docs/userguide/buildingelectricalmodel/schedules/index-schedules:id5}}\end{figure}

\begin{figure}[H]
\centering
\capstart

\noindent\sphinxincludegraphics{{schedules-delete_circuit-3}.PNG}
\caption{Deleting an empty circuit or a circuit without a connected load deletes the OCPD}\label{\detokenize{docs/userguide/buildingelectricalmodel/schedules/index-schedules:id6}}\end{figure}


\subsubsection{Copy/Paste Circuits}
\label{\detokenize{docs/userguide/buildingelectricalmodel/schedules/index-schedules:copy-paste-circuits}}
Copying, Deleting, and Moving Equipment in the Schedules Workspace is similar to the interactions in the {\hyperref[\detokenize{docs/userguide/buildingelectricalmodel/one-line/index-one-line:one-line}]{\sphinxcrossref{\DUrole{std,std-ref}{One-Line}}}}.

Once a Schedule is open, to copy a circuit select the circuit number or click the circuit breaker number.  Selection will be highlighted.

\begin{figure}[H]
\centering
\capstart

\noindent\sphinxincludegraphics{{schedules-copy_circuit_1}.PNG}
\caption{Selecting a circuit}\label{\detokenize{docs/userguide/buildingelectricalmodel/schedules/index-schedules:id7}}\end{figure}

Next use CTRL+C to copy or CTRL+X to cut. The selection will highlight to a different color and be added to the clipboard.

\begin{figure}[H]
\centering
\capstart

\noindent\sphinxincludegraphics{{schedules-copy_circuit_2}.PNG}
\caption{Copying or cutting a circuit}\label{\detokenize{docs/userguide/buildingelectricalmodel/schedules/index-schedules:id8}}\end{figure}

Then, select the source or Equipment to paste to and click Paste or use CTRL+V.  Equipment can be pasted on the same Schedule, or other open Schedule.


\subsubsection{Select All}
\label{\detokenize{docs/userguide/buildingelectricalmodel/schedules/index-schedules:select-all}}
To select all circuits of a Schedule, first select the circuit number or click the circuit breaker number.

Then use CTRL+A to Select All circuits.

\begin{figure}[H]
\centering
\capstart

\noindent\sphinxincludegraphics{{schedules-select_all-1}.PNG}
\caption{All circuits selected are highlighted}\label{\detokenize{docs/userguide/buildingelectricalmodel/schedules/index-schedules:id9}}\end{figure}

\index{I've manually modified the circuit elements.  Is there a way to reset to the code minimum values?@\spxentry{I've manually modified the circuit elements.  Is there a way to reset to the code minimum values?}}\ignorespaces 
\index{How do I Reset to the Code Minimum values?@\spxentry{How do I Reset to the Code Minimum values?}}\ignorespaces 

\subsubsection{Reset to Code Minimum}
\label{\detokenize{docs/userguide/buildingelectricalmodel/schedules/index-schedules:reset-to-code-minimum}}\label{\detokenize{docs/userguide/buildingelectricalmodel/schedules/index-schedules:schedules-reset-to-code-minimum}}\label{\detokenize{docs/userguide/buildingelectricalmodel/schedules/index-schedules:index-2}}
THRUX can automate the selection of many parametric components, but these can also be manually modified.  You can reset the circuit properties to be the code-minimum value, based on the {\hyperref[\detokenize{docs/faq:load-capacity}]{\sphinxcrossref{\DUrole{std,std-ref}{Load Capacity}}}}.

Select the circuit number, then click Reset to Code Minimum (wand symbol).  Or, right-click the circuit, and then select Reset to Code Minimum.

With {\hyperref[\detokenize{docs/userguide/explorersandutilitytools/statusbar/index-status_bar:design-assistance}]{\sphinxcrossref{\DUrole{std,std-ref}{Design Assistance}}}} enabled, THRUX will automatically calculate code minimum values, such as OCPD size, conductor size, and conduit size, when the Equipment’s Load Capacity is changed.

\begin{figure}[H]
\centering
\capstart

\noindent\sphinxincludegraphics{{schedules-reset_to_code_minimum}.PNG}
\caption{Select a circuit and use the wand symbol to reset the circuit its code-minimum values}\label{\detokenize{docs/userguide/buildingelectricalmodel/schedules/index-schedules:id10}}\end{figure}

\index{How do I edit the properties of a circuit? - Schedules@\spxentry{How do I edit the properties of a circuit? - Schedules}}\ignorespaces \phantomsection\label{\detokenize{docs/userguide/buildingelectricalmodel/schedules/index-schedules:schedules-rehosting}}
\index{Is there a way to reorder circuits on my distribution equipment?@\spxentry{Is there a way to reorder circuits on my distribution equipment?}}\ignorespaces 

\subsubsection{Moving, Rehosting Equipment, or Reordering Circuits}
\label{\detokenize{docs/userguide/buildingelectricalmodel/schedules/index-schedules:moving-rehosting-equipment-or-reordering-circuits}}\label{\detokenize{docs/userguide/buildingelectricalmodel/schedules/index-schedules:index-4}}
The ordering of circuits can affect the overall construction of the board.  To move circuits, select the circuit number and then click and drag the Grip (grip icon).

\begin{figure}[H]
\centering
\capstart

\noindent\sphinxincludegraphics{{schedules-rehost}.PNG}
\caption{Selected circuit is highlighted.  Use the grip to move the circuit to a separate space}\label{\detokenize{docs/userguide/buildingelectricalmodel/schedules/index-schedules:id11}}\end{figure}

Another way to rehost circuits is to right-click on a circuit, and select Move/Rehost.

\begin{figure}[H]
\centering
\capstart

\noindent\sphinxincludegraphics{{schedules-rehost_alt}.PNG}
\caption{Right-click on a circuit to rehost it}\label{\detokenize{docs/userguide/buildingelectricalmodel/schedules/index-schedules:id12}}\end{figure}


\subsubsection{Lock/Unlock}
\label{\detokenize{docs/userguide/buildingelectricalmodel/schedules/index-schedules:lock-unlock}}
You can lock a selection which will prevent elements from being modified.  Select a circuit number and then click the Lock/Unlock (lock symbol) button.

\begin{figure}[H]
\centering
\capstart

\noindent\sphinxincludegraphics{{schedules-lock_unlock_1}.PNG}
\caption{Unlocked circuits are free to be modified}\label{\detokenize{docs/userguide/buildingelectricalmodel/schedules/index-schedules:id13}}\end{figure}

\begin{figure}[H]
\centering
\capstart

\noindent\sphinxincludegraphics{{schedules-lock_unlock_2}.PNG}
\caption{Locked circuits are shaded}\label{\detokenize{docs/userguide/buildingelectricalmodel/schedules/index-schedules:id14}}\end{figure}

\index{Why Can't I Add Equipment to this Distribution Board?@\spxentry{Why Can't I Add Equipment to this Distribution Board?}}\ignorespaces 

\subsubsection{Adding OCPD’s}
\label{\detokenize{docs/userguide/buildingelectricalmodel/schedules/index-schedules:adding-ocpd-s}}\label{\detokenize{docs/userguide/buildingelectricalmodel/schedules/index-schedules:index-5}}
The amount of protective devices a distribution board supports is proportional to its physical installation.  Equipment cannot be added to a distribution board unless there is space alotted.

To add a protective device, click Add OCPD.

\begin{figure}[H]
\centering
\capstart

\noindent\sphinxincludegraphics{{schedules-add_ocpd-1}.PNG}
\caption{Adding an OCPD to a Schedule}\label{\detokenize{docs/userguide/buildingelectricalmodel/schedules/index-schedules:id15}}\end{figure}

\index{Is there a way to navigate from the One-Line to another Workspace like the Schedules?@\spxentry{Is there a way to navigate from the One-Line to another Workspace like the Schedules?}}\ignorespaces 

\subsubsection{Navigate}
\label{\detokenize{docs/userguide/buildingelectricalmodel/schedules/index-schedules:navigate}}\label{\detokenize{docs/userguide/buildingelectricalmodel/schedules/index-schedules:scope-to-one-line}}\label{\detokenize{docs/userguide/buildingelectricalmodel/schedules/index-schedules:index-6}}
Right-click on a circuit element to open a utility menu.  You can navigate to other Schedules or to the One-Line by selecting Scope to Schedule or Scope to One-Line.

\begin{figure}[H]
\centering
\capstart

\noindent\sphinxincludegraphics{{schedules-navigate}.PNG}
\caption{Navigating to the One-Line from a Schedule}\label{\detokenize{docs/userguide/buildingelectricalmodel/schedules/index-schedules:id16}}\end{figure}


\subsubsection{Comparing Schedules Between Branches}
\label{\detokenize{docs/userguide/buildingelectricalmodel/schedules/index-schedules:comparing-schedules-between-branches}}
It is common to identify changes between Issuances or Branches.

Use Compare Against Prior Issuance to compare Issuances.  In the example below, we’re comparing a Schedule of a main distribution board used in two Issuances.

In the BASE Issuance we’ve used copper conductors, while the other Issuance is using aluminum conductors.

Open a Branch and open the Schedules you would like to compare.

This feature allows the designer to track changes between Issuances.

\begin{figure}[H]
\centering
\capstart

\noindent\sphinxincludegraphics{{schedules-compare-issuance-1}.PNG}
\caption{Changes Between Issuances}\label{\detokenize{docs/userguide/buildingelectricalmodel/schedules/index-schedules:id17}}\end{figure}


\subsubsection{MLO - Main Lug Only}
\label{\detokenize{docs/userguide/buildingelectricalmodel/schedules/index-schedules:mlo-main-lug-only}}
Distribution equipment can be equipped to have a main protective device or without, otherwise known as MLO (Main Lug Only).

To configure your distribution equipment as MLO, click on the settings icon (gear).  Then choose Change to MLO.

\begin{figure}[H]
\centering
\capstart

\noindent\sphinxincludegraphics{{schedules-mlo-1}.PNG}
\caption{Change to MLO}\label{\detokenize{docs/userguide/buildingelectricalmodel/schedules/index-schedules:id18}}\end{figure}

\begin{figure}[H]
\centering
\capstart

\noindent\sphinxincludegraphics{{schedules-mlo-2}.PNG}
\caption{Changing the Load Description to denote MLO}\label{\detokenize{docs/userguide/buildingelectricalmodel/schedules/index-schedules:id19}}\end{figure}


\subsubsection{Configure as Tap}
\label{\detokenize{docs/userguide/buildingelectricalmodel/schedules/index-schedules:configure-as-tap}}
A common type of electrical connection is a tap.  Taps can be within the electrical equipment, or external.

If they are internal, right-click on a circuit and use Configure as Tap.  If they are external, create a Tap Node.

\begin{figure}[H]
\centering
\capstart

\noindent\sphinxincludegraphics{{schedules-configure_as_tap-1}.PNG}
\caption{Right-click a circuit and use Configure as Tap}\label{\detokenize{docs/userguide/buildingelectricalmodel/schedules/index-schedules:id20}}\end{figure}

Internal taps remove OCPDs on the line side of distribution equipment.

\begin{figure}[H]
\centering
\capstart

\noindent\sphinxincludegraphics{{schedules-configure_as_tap-2}.PNG}
\caption{Right-click a circuit and use Configure as Tap}\label{\detokenize{docs/userguide/buildingelectricalmodel/schedules/index-schedules:id21}}\end{figure}


\subsubsection{Converting Breaker/Switch and Fuse}
\label{\detokenize{docs/userguide/buildingelectricalmodel/schedules/index-schedules:converting-breaker-switch-and-fuse}}
By default, Distribution Boards use breakers as protective devices, while Switchboards use switch and fuse as protective devices.  To convert a board’s protective devices, click the Settings (gear symbol) button in the top left of the Schedule.  Then, under Change Schedule Type, select Convert to Switch/Fuse.

\begin{figure}[H]
\centering
\capstart

\noindent\sphinxincludegraphics{{schedules-ocpd_conversion}.PNG}
\caption{Converting a Distribution Board to a Switchboard will change its protective devices from circuit breakers to switch and fuses}\label{\detokenize{docs/userguide/buildingelectricalmodel/schedules/index-schedules:id22}}\end{figure}


\subsubsection{Multi-Pole Circuits}
\label{\detokenize{docs/userguide/buildingelectricalmodel/schedules/index-schedules:multi-pole-circuits}}
Panelboard Schedules have the ability to convert single pole circuits to multi-pole circuits.

\begin{figure}[H]
\centering
\capstart

\noindent\sphinxincludegraphics{{schedules-multi_pole-1}.PNG}
\caption{Select multiple circuits, then right-click, and choose Convert to Multi-Pole}\label{\detokenize{docs/userguide/buildingelectricalmodel/schedules/index-schedules:id23}}\end{figure}


\subsubsection{Poles}
\label{\detokenize{docs/userguide/buildingelectricalmodel/schedules/index-schedules:poles}}
Panelboard Schedules are commonly installed in configurations of 18, 24, 30, 36, and 42 Poles.

By default, Panelboards are installed with 42 Poles.

\begin{figure}[H]
\centering
\capstart

\noindent\sphinxincludegraphics{{schedules-poles}.PNG}
\caption{Changing the number of Poles}\label{\detokenize{docs/userguide/buildingelectricalmodel/schedules/index-schedules:id24}}\end{figure}


\subsubsection{Schedule Views}
\label{\detokenize{docs/userguide/buildingelectricalmodel/schedules/index-schedules:schedule-views}}
Groups of Schedules can be saved for a later viewing.  Open all of the Schedules to be grouped, then click Save As New View, and enter a name.  Return to this view at any time.

In this example, two Schedule Views have been created, and you can toggle between them.

\begin{figure}[H]
\centering
\capstart

\noindent\sphinxincludegraphics{{schedules-save_new_view-1}.PNG}
\caption{Saving Schedule Views allows groups of Schedules for later viewing}\label{\detokenize{docs/userguide/buildingelectricalmodel/schedules/index-schedules:id25}}\end{figure}

\index{Exporting - Schedules@\spxentry{Exporting - Schedules}}\ignorespaces 

\subsubsection{Exporting}
\label{\detokenize{docs/userguide/buildingelectricalmodel/schedules/index-schedules:exporting}}\label{\detokenize{docs/userguide/buildingelectricalmodel/schedules/index-schedules:exporting-schedules}}\label{\detokenize{docs/userguide/buildingelectricalmodel/schedules/index-schedules:index-7}}
Schedules are exportable to AutoCAD or Excel.

To export Schedules, click Export (down arrow) button in the top right of the Workspace.

\begin{figure}[H]
\centering
\capstart

\noindent\sphinxincludegraphics{{schedules-exporting-1}.PNG}
\caption{Exporting active Schedules}\label{\detokenize{docs/userguide/buildingelectricalmodel/schedules/index-schedules:id26}}\end{figure}

You have the option to export all of the active Schedules or all of the Schedules in the entire model.

Active Schedules are listed in the top.

Choose Export to AutoCAD: Selected

or

Export to AutoCAD: All.


\subsection{Flag Tracker}
\label{\detokenize{docs/userguide/index-user_guide:flag-tracker}}

\subsubsection{Using the Flag Tracker}
\label{\detokenize{docs/userguide/buildingelectricalmodel/flagtracker/index-flag_tracker:using-the-flag-tracker}}\label{\detokenize{docs/userguide/buildingelectricalmodel/flagtracker/index-flag_tracker:flag-tracker}}\label{\detokenize{docs/userguide/buildingelectricalmodel/flagtracker/index-flag_tracker::doc}}
Below is an example showing a Utility source and a main Distribution Board that feeds downstream Distribution Boards.

\begin{figure}[H]
\centering
\capstart

\noindent\sphinxincludegraphics{{flag_tracker_1}.PNG}
\caption{Small example of a distribution network}\label{\detokenize{docs/userguide/buildingelectricalmodel/flagtracker/index-flag_tracker:id1}}\end{figure}

Open the Flag Tracker by clicking on the Flag Tracker icon located on the right sidebar.  Note that the current list of Flags is empty.

\begin{figure}[H]
\centering
\capstart

\noindent\sphinxincludegraphics{{flag_tracker_2}.PNG}
\caption{Opening the Flag Tracker}\label{\detokenize{docs/userguide/buildingelectricalmodel/flagtracker/index-flag_tracker:id2}}\end{figure}

Enable the {\hyperref[\detokenize{docs/userguide/buildingelectricalmodel/one-line/index-one-line:property-tags}]{\sphinxcrossref{\DUrole{std,std-ref}{Quick View}}}} for Voltage Drop and the Property Tags for {\hyperref[\detokenize{docs/userguide/index-user_guide:calculated-length}]{\sphinxcrossref{\DUrole{std,std-ref}{Calculated Length}}}} and {\hyperref[\detokenize{docs/userguide/index-user_guide:manual-added-length}]{\sphinxcrossref{\DUrole{std,std-ref}{Manual Added Length}}}}.

\begin{figure}[H]
\centering
\capstart

\noindent\sphinxincludegraphics{{flag_tracker_3}.PNG}
\caption{Distribution network with Voltage Drop Quick View and the Calculated Length Property Tag enabled}\label{\detokenize{docs/userguide/buildingelectricalmodel/flagtracker/index-flag_tracker:id3}}\end{figure}

Increase the Manual Added Length to 15,000 for the circuit feeding the MDB Distribution Board and note the list of current Flags.

\begin{figure}[H]
\centering
\capstart

\noindent\sphinxincludegraphics{{flag_tracker_4}.PNG}
\caption{Increasing the length raised voltage drop Flags}\label{\detokenize{docs/userguide/buildingelectricalmodel/flagtracker/index-flag_tracker:id4}}\end{figure}


\paragraph{Navigation}
\label{\detokenize{docs/userguide/buildingelectricalmodel/flagtracker/index-flag_tracker:navigation}}
Designers can quickly jump to find issues in their model by using the Flag Tracker.

\begin{figure}[H]
\centering
\capstart

\noindent\sphinxincludegraphics{{flag_tracker_5}.PNG}
\caption{Use the Navigate button to jump to where the issue is in the model}\label{\detokenize{docs/userguide/buildingelectricalmodel/flagtracker/index-flag_tracker:id5}}\end{figure}


\subsection{Studies}
\label{\detokenize{docs/userguide/index-user_guide:studies}}\phantomsection\label{\detokenize{docs/userguide/buildingelectricalmodel/studies/index-studies:studies}}
The Studies are used as a tabular and flexible reporting tool.  Designers have the ability to customize the columns for the data they wish to view.

Designers can also view Studies which are sorted by Equipment Type.

\begin{figure}[H]
\centering
\capstart

\noindent\sphinxincludegraphics{{studies-11}.PNG}
\caption{Click on the tag symbol to customize the columns of your reports}\label{\detokenize{docs/userguide/buildingelectricalmodel/studies/index-studies:id1}}\end{figure}


\subsubsection{Queries - Voltage Drop, Short Circuit, Loading Studies}
\label{\detokenize{docs/userguide/buildingelectricalmodel/studies/index-studies:queries-voltage-drop-short-circuit-loading-studies}}\label{\detokenize{docs/userguide/buildingelectricalmodel/studies/index-studies::doc}}
Use the available queries to view reports for common studies like voltage drop, short circuit, or loading.

It’s also possible to view equipment which are loaded and which equipment is hidden.

\begin{figure}[H]
\centering
\capstart

\noindent\sphinxincludegraphics{{studies-2}.PNG}
\caption{Voltage Drop Study}\label{\detokenize{docs/userguide/buildingelectricalmodel/studies/index-studies:id2}}\end{figure}


\subsubsection{Searching}
\label{\detokenize{docs/userguide/buildingelectricalmodel/studies/index-studies:searching}}
Search for specific pieces of Equipment by name.

\begin{figure}[H]
\centering
\capstart

\noindent\sphinxincludegraphics{{studies-search}.PNG}
\caption{Searching for Equipment}\label{\detokenize{docs/userguide/buildingelectricalmodel/studies/index-studies:id3}}\end{figure}


\subsubsection{Navigate}
\label{\detokenize{docs/userguide/buildingelectricalmodel/studies/index-studies:navigate}}
Navigate to other Workspaces by using the Navigate button.

\begin{figure}[H]
\centering
\capstart

\noindent\sphinxincludegraphics{{studies-navigate-1}.PNG}
\caption{Using the Navigate command}\label{\detokenize{docs/userguide/buildingelectricalmodel/studies/index-studies:id4}}\end{figure}

Select a destination and hit Go.

\begin{figure}[H]
\centering
\capstart

\noindent\sphinxincludegraphics{{studies-navigate-2}.PNG}
\caption{Navigating to the One-Line}\label{\detokenize{docs/userguide/buildingelectricalmodel/studies/index-studies:id5}}\end{figure}


\subsubsection{Batch Changes and Editing}
\label{\detokenize{docs/userguide/buildingelectricalmodel/studies/index-studies:batch-changes-and-editing}}
Designers have the ability to modify properties which are common across multiple items.

\begin{figure}[H]
\centering
\capstart

\noindent\sphinxincludegraphics{{studies-batch_changes-1}.PNG}
\caption{Selecting multiple circuits}\label{\detokenize{docs/userguide/buildingelectricalmodel/studies/index-studies:id6}}\end{figure}

For example, in this case, we’re changing the length of each run to a custom length.

\begin{figure}[H]
\centering
\capstart

\noindent\sphinxincludegraphics{{studies-batch_changes-2}.PNG}
\caption{Changing the Net Length across multiple runs.  An indicator notifies the user that the value is overridden and not calculated.}\label{\detokenize{docs/userguide/buildingelectricalmodel/studies/index-studies:id7}}\end{figure}


\subsubsection{Printing and Exporting}
\label{\detokenize{docs/userguide/buildingelectricalmodel/studies/index-studies:printing-and-exporting}}
The {\hyperref[\detokenize{docs/userguide/buildingelectricalmodel/studies/index-studies:studies}]{\sphinxcrossref{\DUrole{std,std-ref}{Studies}}}} and {\hyperref[\detokenize{docs/userguide/pricingmodel/pricingreport/index-pricing_report:pricing-report}]{\sphinxcrossref{\DUrole{std,std-ref}{Pricing Report}}}} can be printed by clicking the Print icon located at the top toolbar.  They can also be exported to .xml, .csv, or .json.

Use the Copy to Clipboard button to copy and paste to Excel.

\begin{figure}[H]
\centering
\capstart

\noindent\sphinxincludegraphics{{studies-print}.PNG}
\caption{Printing}\label{\detokenize{docs/userguide/buildingelectricalmodel/studies/index-studies:id8}}\end{figure}


\section{\sphinxstylestrong{Building the Mechanical Model}}
\label{\detokenize{docs/userguide/index-user_guide:building-the-mechanical-model}}
Coming soon!


\subsection{Building the Mechanical Model}
\label{\detokenize{docs/userguide/buildingmechanicalmodel/index-building_mechanical_model:building-the-mechanical-model}}\label{\detokenize{docs/userguide/buildingmechanicalmodel/index-building_mechanical_model:id1}}\label{\detokenize{docs/userguide/buildingmechanicalmodel/index-building_mechanical_model::doc}}
Coming soon!


\section{\sphinxstylestrong{Building the Plumbing and Fire Protection Model}}
\label{\detokenize{docs/userguide/index-user_guide:building-the-plumbing-and-fire-protection-model}}
Coming soon!


\subsection{Building the PFP Model}
\label{\detokenize{docs/userguide/buildingpfpmodel/index-building_pfp_model:building-the-pfp-model}}\label{\detokenize{docs/userguide/buildingpfpmodel/index-building_pfp_model:id1}}\label{\detokenize{docs/userguide/buildingpfpmodel/index-building_pfp_model::doc}}
Coming soon!


\section{\sphinxstylestrong{Pricing Model}}
\label{\detokenize{docs/userguide/index-user_guide:pricing-model}}\label{\detokenize{docs/userguide/index-user_guide:id8}}
The Pricing Model is built around the Electrical model and can be viewed in a few Workspaces.  The {\hyperref[\detokenize{docs/userguide/pricingmodel/pricetracker/index-price_tracker:price-tracker}]{\sphinxcrossref{\DUrole{std,std-ref}{Price Tracker}}}} is used to live monitor the price of the model.  For example, as you change the location of a major equipment room, the Price Tracker would display order of magnitude estimates for that change.  For a more complete tabular report, use the {\hyperref[\detokenize{docs/userguide/pricingmodel/pricingreport/index-pricing_report:pricing-report}]{\sphinxcrossref{\DUrole{std,std-ref}{Pricing Report}}}} Workspace.

{\hyperref[\detokenize{docs/userguide/pricingmodel/equipmentrates/index-equipment_rates:equipment-rates}]{\sphinxcrossref{\DUrole{std,std-ref}{Equipment Rates}}}} is a customizable catalog which composes the bid for materials.


\subsection{Pricing Report}
\label{\detokenize{docs/userguide/pricingmodel/pricingreport/index-pricing_report:pricing-report}}\label{\detokenize{docs/userguide/pricingmodel/pricingreport/index-pricing_report:id1}}\label{\detokenize{docs/userguide/pricingmodel/pricingreport/index-pricing_report::doc}}
Pricing Report is a reporting Workspace for the order of magnitude estimates of the Project.

\begin{figure}[H]
\centering
\capstart

\noindent\sphinxincludegraphics{{pricing_report}.PNG}
\caption{Pricing Report showing labor and material rates for a Project}\label{\detokenize{docs/userguide/pricingmodel/pricingreport/index-pricing_report:id2}}\end{figure}


\subsection{Price Tracker}
\label{\detokenize{docs/userguide/pricingmodel/pricetracker/index-price_tracker:price-tracker}}\label{\detokenize{docs/userguide/pricingmodel/pricetracker/index-price_tracker:id1}}\label{\detokenize{docs/userguide/pricingmodel/pricetracker/index-price_tracker::doc}}
The price of the model is constantly being evaluated as the design changes.  Open the Price Tracker to view live cost impacts.

Shown below, note that the price as the location of an Electrical Equipment is shifted from one Room to another Room.

\begin{figure}[H]
\centering
\capstart

\noindent\sphinxincludegraphics{{price_tracker_1}.PNG}
\caption{Open the Price Tracker (dollar sign icon) by using the {\hyperref[\detokenize{docs/userguide/index-user_guide:id10}]{\sphinxcrossref{\DUrole{std,std-ref}{Explorer Toolbar}}}}}\label{\detokenize{docs/userguide/pricingmodel/pricetracker/index-price_tracker:id2}}\end{figure}

\begin{figure}[H]
\centering
\capstart

\noindent\sphinxincludegraphics{{price_tracker_2}.PNG}
\caption{Shifting a main distribution board to a different Room with the Price Tracker active}\label{\detokenize{docs/userguide/pricingmodel/pricetracker/index-price_tracker:id3}}\end{figure}

A tabular format can be viewed by opening the {\hyperref[\detokenize{docs/userguide/pricingmodel/pricingreport/index-pricing_report:pricing-report}]{\sphinxcrossref{\DUrole{std,std-ref}{Pricing Report}}}} Workspace.


\subsection{Equipment Rates}
\label{\detokenize{docs/userguide/pricingmodel/equipmentrates/index-equipment_rates:equipment-rates}}\label{\detokenize{docs/userguide/pricingmodel/equipmentrates/index-equipment_rates:id1}}\label{\detokenize{docs/userguide/pricingmodel/equipmentrates/index-equipment_rates::doc}}
Equipment Rates is a customizable catalog that composes the bid for materials.

\begin{figure}[H]
\centering
\capstart

\noindent\sphinxincludegraphics{{equipment_rates_1}.PNG}
\caption{ATS Equipment Rates}\label{\detokenize{docs/userguide/pricingmodel/equipmentrates/index-equipment_rates:id2}}\end{figure}

Use the Filters to sort the Equipment Rates.

\begin{figure}[H]
\centering
\capstart

\noindent\sphinxincludegraphics{{equipment_rates_2}.PNG}
\caption{Sorting Equipment Rates by Voltage}\label{\detokenize{docs/userguide/pricingmodel/equipmentrates/index-equipment_rates:id3}}\end{figure}


\section{\sphinxstylestrong{Project Management}}
\label{\detokenize{docs/userguide/index-user_guide:project-management}}\label{\detokenize{docs/userguide/index-user_guide:id9}}
Engineers are often tasked to study different alternatives or schemes and present them to the Owner.

The {\hyperref[\detokenize{docs/userguide/projectmanagement/issuancelog/index-issuance_log:issuance-log}]{\sphinxcrossref{\DUrole{std,std-ref}{Issuance Log}}}} allows you to create Branches, while the {\hyperref[\detokenize{docs/userguide/projectmanagement/changetracking/index-change_tracking:change-tracking}]{\sphinxcrossref{\DUrole{std,std-ref}{Change Tracking}}}} Workspace allows you to compare Branches against the base Branch.

\begin{figure}[H]
\centering
\capstart

\noindent\sphinxincludegraphics{{project_management-1}.PNG}
\caption{Project Management tools - Issuance Log, Change Tracking}\label{\detokenize{docs/userguide/index-user_guide:id29}}\end{figure}


\subsection{Branching}
\label{\detokenize{docs/userguide/projectmanagement/issuancelog/index-issuance_log:branching}}\label{\detokenize{docs/userguide/projectmanagement/issuancelog/index-issuance_log:issuance-log}}\label{\detokenize{docs/userguide/projectmanagement/issuancelog/index-issuance_log::doc}}
Designers are often tasked to study different options in order to determine the best option for the Owner.  The {\hyperref[\detokenize{docs/userguide/projectmanagement/issuancelog/index-issuance_log:issuance-log}]{\sphinxcrossref{\DUrole{std,std-ref}{Issuance Log}}}} is a tool which allows the designer to create Branches of their model.

The initial Base scheme is also known as the Base Issuance, and should be the primary working Branch.  If a designer is asked to study an alternative option, create a Branch.  Designers can switch between Branches as necessary.

To create a Branch, open the Issuance Log on the right sidebar.

\begin{figure}[H]
\centering
\capstart

\noindent\sphinxincludegraphics{{issuance_log_1}.PNG}
\caption{The Active Issuance is displayed in the top right of the application}\label{\detokenize{docs/userguide/projectmanagement/issuancelog/index-issuance_log:id1}}\end{figure}

Click the branch symbol to create a Sub-Branch.

\begin{figure}[H]
\centering
\capstart

\noindent\sphinxincludegraphics{{issuance_log_2}.PNG}
\caption{Creating a Sub-Branch off of the Base Branch}\label{\detokenize{docs/userguide/projectmanagement/issuancelog/index-issuance_log:id2}}\end{figure}

Give the Branch a name and click on Create.  By default, a Branch will be created off of the current working model.

You have the option to ignore or abandon any changes since you last saved, or since you opened the Project.  This is also known as the current working model, or off of the last saved model.

\begin{figure}[H]
\centering
\capstart

\noindent\sphinxincludegraphics{{issuance_log_3}.PNG}
\caption{Using the New Branch Wizard}\label{\detokenize{docs/userguide/projectmanagement/issuancelog/index-issuance_log:id3}}\end{figure}

Swap between Branches by clicking the arrow symbols.  Note the active Branch is shown in the top right navigation bar.  A Branching map is displayed at the bottom of the Issuance Log.  Note the differences between the two One-Lines of each Branch.

\begin{figure}[H]
\centering
\capstart

\noindent\sphinxincludegraphics{{issuance_log_4}.PNG}
\caption{Switching between Branches}\label{\detokenize{docs/userguide/projectmanagement/issuancelog/index-issuance_log:id4}}\end{figure}

\begin{figure}[H]
\centering
\capstart

\noindent\sphinxincludegraphics{{issuance_log_5}.PNG}
\caption{Switching between Branches}\label{\detokenize{docs/userguide/projectmanagement/issuancelog/index-issuance_log:id5}}\end{figure}

In addition, designers can compare changes between Branches, by using the {\hyperref[\detokenize{docs/userguide/projectmanagement/changetracking/index-change_tracking:change-tracking}]{\sphinxcrossref{\DUrole{std,std-ref}{Change Tracking}}}} Workspace.


\subsection{Change Tracking}
\label{\detokenize{docs/userguide/projectmanagement/changetracking/index-change_tracking:change-tracking}}\label{\detokenize{docs/userguide/projectmanagement/changetracking/index-change_tracking:id1}}\label{\detokenize{docs/userguide/projectmanagement/changetracking/index-change_tracking::doc}}
Change Tracking allows the designer to compare changes between Branches.

\begin{figure}[H]
\centering
\capstart

\noindent\sphinxincludegraphics{{change-tracking_overview_1}.PNG}
\caption{Utilizing the Change Tracking Workspace}\label{\detokenize{docs/userguide/projectmanagement/changetracking/index-change_tracking:id2}}\end{figure}

Select the Branch to compare against the current Branch.  Select the entities to compare against and select Refresh.  This list can be exported to .csv (Excel) by using Copy and Paste.

\begin{figure}[H]
\centering
\capstart

\noindent\sphinxincludegraphics{{change-tracking_overview_2}.PNG}
\caption{Comparing a Branch against the base Branch}\label{\detokenize{docs/userguide/projectmanagement/changetracking/index-change_tracking:id3}}\end{figure}


\subsection{Accessibility}
\label{\detokenize{docs/userguide/projectmanagement/accessibility/index-accessibility:accessibility}}\label{\detokenize{docs/userguide/projectmanagement/accessibility/index-accessibility:id1}}\label{\detokenize{docs/userguide/projectmanagement/accessibility/index-accessibility::doc}}
Designers have the ability to add or invite multiple users to their projects.

To invite other users to your project, first log into the Account Portal.

\begin{figure}[H]
\centering
\capstart

\noindent\sphinxincludegraphics{{invitation-1}.PNG}
\caption{Click on your name, and then click on Manage Account.}\label{\detokenize{docs/userguide/projectmanagement/accessibility/index-accessibility:id2}}\end{figure}

Sign in to the \sphinxhref{http://thruxcoreweb.azurewebsites.net/AccountPortal}{Account Portal}:

\begin{figure}[H]
\centering
\capstart

\noindent\sphinxincludegraphics{{invitation-2}.PNG}
\caption{Account Portal}\label{\detokenize{docs/userguide/projectmanagement/accessibility/index-accessibility:id3}}\end{figure}

Click on Projects and find your project.

\begin{figure}[H]
\centering
\capstart

\noindent\sphinxincludegraphics{{invitation-3}.PNG}
\caption{Click on Manage}\label{\detokenize{docs/userguide/projectmanagement/accessibility/index-accessibility:id4}}\end{figure}

\begin{figure}[H]
\centering
\capstart

\noindent\sphinxincludegraphics{{invitation-4}.PNG}
\caption{Enter their email address and click on Invite to send an invitation}\label{\detokenize{docs/userguide/projectmanagement/accessibility/index-accessibility:id5}}\end{figure}

They should receive an invitation email.

If they already have a THRUX account, the name should display on the page.  If it doesn’t appear, refresh the page.

If they do not have an account, contact our {\hyperref[\detokenize{docs/introduction/index-thrux:support}]{\sphinxcrossref{\DUrole{std,std-ref}{Support Team}}}} to create one.
\begin{itemize}
\item {} 
\sphinxhref{mailto:thruxservices@thrux.io}{thruxservices@thrux.io}

\end{itemize}


\section{\sphinxstylestrong{Explorers and other Utility Tools}}
\label{\detokenize{docs/userguide/index-user_guide:explorers-and-other-utility-tools}}\label{\detokenize{docs/userguide/index-user_guide:id10}}
The right-side toolbar of THRUX is generally where the explorers or utility tools are located.  Explorers can be pinned to always be visible while other explorers are being used.

\begin{figure}[H]
\centering
\capstart

\noindent\sphinxincludegraphics{{explorersandutilitytools-1}.PNG}
\caption{Explorer Toolbar}\label{\detokenize{docs/userguide/index-user_guide:id30}}\end{figure}


\subsection{Properties Explorer}
\label{\detokenize{docs/userguide/explorersandutilitytools/propertiesexplorer/index-properties_explorer:properties-explorer}}\label{\detokenize{docs/userguide/explorersandutilitytools/propertiesexplorer/index-properties_explorer:id1}}\label{\detokenize{docs/userguide/explorersandutilitytools/propertiesexplorer/index-properties_explorer::doc}}
The Properties Explorer displays the various properties associated with the current selection.

\begin{figure}[H]
\centering

\noindent\sphinxincludegraphics{{properties_explorer_1}.PNG}
\end{figure}

Note that the Explorer window can scroll, collapse, and expand sections to view additional properties.


\subsubsection{Tags and Load Groups}
\label{\detokenize{docs/userguide/explorersandutilitytools/propertiesexplorer/index-properties_explorer:tags-and-load-groups}}
Tags are a way to easily find and organize information.  They can be used in the Studies to further query data about your models.  Load Groups are used in a similar fashion.

\begin{figure}[H]
\centering
\capstart

\noindent\sphinxincludegraphics{{properties_explorer_4}.PNG}
\caption{Tags and Load Groups}\label{\detokenize{docs/userguide/explorersandutilitytools/propertiesexplorer/index-properties_explorer:id2}}\end{figure}

For more property definitions and information, see our {\hyperref[\detokenize{docs/definitions/index-definitions:definitions}]{\sphinxcrossref{\DUrole{std,std-ref}{Equipment Properties Index}}}}:


\subsection{Cascade Monitor}
\label{\detokenize{docs/userguide/explorersandutilitytools/cascademonitor/index-cascade_monitor:cascade-monitor}}\label{\detokenize{docs/userguide/explorersandutilitytools/cascademonitor/index-cascade_monitor:id1}}\label{\detokenize{docs/userguide/explorersandutilitytools/cascademonitor/index-cascade_monitor::doc}}
Systems are interrelated, and it is important to be aware of how each system may respond to certain changes.

For example, changing the Load Capacity of a Distribution Board would cause a recalculation of the conductor arrangement of the circuit, including the conduit size.

A designer may like to view items which change as a result of a single change.  Think of this as viewing the cascading impacts.

The Cascade Monitor works with {\hyperref[\detokenize{docs/userguide/explorersandutilitytools/statusbar/index-status_bar:design-assistance}]{\sphinxcrossref{\DUrole{std,std-ref}{Design Assistance}}}} and allows the ability to respond to these cascading impacts by manually accepting or rejecting them.

Open the Cascade Monitor and turn off Design Assistance.

\begin{figure}[H]
\centering
\capstart

\noindent\sphinxincludegraphics{{cascade_monitor_1}.PNG}
\caption{Cascade Monitor with No Design Assistance}\label{\detokenize{docs/userguide/explorersandutilitytools/cascademonitor/index-cascade_monitor:id2}}\end{figure}

See the example below which shows the impacts, or cascading effects, of a decrease in the Load Capacity of a Distribution Board.

A reduction in Load Capacity will result in a reduction of the protective devices trip size, which will result in a reduction in phase size for the conductor arrangement.

\begin{figure}[H]
\centering
\capstart

\noindent\sphinxincludegraphics{{cascade_monitor_2}.PNG}
\caption{Changing the Load Capacity while the Cascade Monitor is active}\label{\detokenize{docs/userguide/explorersandutilitytools/cascademonitor/index-cascade_monitor:id3}}\end{figure}


\subsection{Data Exporter}
\label{\detokenize{docs/userguide/explorersandutilitytools/dataexporter/index-data_exporter:data-exporter}}\label{\detokenize{docs/userguide/explorersandutilitytools/dataexporter/index-data_exporter:id1}}\label{\detokenize{docs/userguide/explorersandutilitytools/dataexporter/index-data_exporter::doc}}
The Data Exporter allows the designer to export their model .csv, .xml, or .json.

\begin{figure}[H]
\centering
\capstart

\noindent\sphinxincludegraphics{{data_exporter_1}.PNG}
\caption{Exporting a model using the Data Exporter}\label{\detokenize{docs/userguide/explorersandutilitytools/dataexporter/index-data_exporter:id2}}\end{figure}


\subsection{Codes Reference}
\label{\detokenize{docs/userguide/explorersandutilitytools/codesreference/index-codes_reference:codes-reference}}\label{\detokenize{docs/userguide/explorersandutilitytools/codesreference/index-codes_reference:id1}}\label{\detokenize{docs/userguide/explorersandutilitytools/codesreference/index-codes_reference::doc}}
Codes Reference is a reference Workspace for safety codes and standards.

\begin{figure}[H]
\centering
\capstart

\noindent\sphinxincludegraphics{{ampacities}.PNG}
\caption{Ampacity tables based on the NEC}\label{\detokenize{docs/userguide/explorersandutilitytools/codesreference/index-codes_reference:id2}}\end{figure}


\subsection{Status Bar}
\label{\detokenize{docs/userguide/explorersandutilitytools/statusbar/index-status_bar:status-bar}}\label{\detokenize{docs/userguide/explorersandutilitytools/statusbar/index-status_bar:id1}}\label{\detokenize{docs/userguide/explorersandutilitytools/statusbar/index-status_bar::doc}}
The bottom left of the Status Bar indicates the state of THRUX.  It will change colors as the state changes.

The bottom right of the Status Bar provides a few more Utility functions such as the ability to Design Assistance, Refreshing Calculations, Auto Calculate, and Toggle Flag Settings.


\subsubsection{Design Assistance}
\label{\detokenize{docs/userguide/explorersandutilitytools/statusbar/index-status_bar:design-assistance}}\label{\detokenize{docs/userguide/explorersandutilitytools/statusbar/index-status_bar:id2}}
Different Design Mode Levels are togglable in THRUX, which are indicated by the three lightning symbols.  The default mode is Full Design Assistance which references the applicable safety codes, standards, and user settings to calculate code-minimum values.


\bigskip\hrule\bigskip



\paragraph{Full Design Assistance}
\label{\detokenize{docs/userguide/explorersandutilitytools/statusbar/index-status_bar:full-design-assistance}}\label{\detokenize{docs/userguide/explorersandutilitytools/statusbar/index-status_bar:id3}}
Any change in circuit properties will trigger a recalculation in all relevant code-minimum values.

\begin{figure}[H]
\centering

\noindent\sphinxincludegraphics{{status_bar_utility_1}.PNG}
\end{figure}


\paragraph{Custom Design Assistance}
\label{\detokenize{docs/userguide/explorersandutilitytools/statusbar/index-status_bar:custom-design-assistance}}\label{\detokenize{docs/userguide/explorersandutilitytools/statusbar/index-status_bar:id4}}
Any change in circuit properties will trigger a recalculation in all relevant code-minimum values that are not omitted by the Custom-Design-Assistance Settings.  Flags will still be raised as long as they are enabled.

\begin{figure}[H]
\centering

\noindent\sphinxincludegraphics{{status_bar_utility_2}.PNG}
\end{figure}


\paragraph{No Design Assistance}
\label{\detokenize{docs/userguide/explorersandutilitytools/statusbar/index-status_bar:no-design-assistance}}\label{\detokenize{docs/userguide/explorersandutilitytools/statusbar/index-status_bar:id5}}
Designer has full manual control of all circuit properties.  No auto-sizing will be performed.

\begin{figure}[H]
\centering

\noindent\sphinxincludegraphics{{status_bar_utility_3}.PNG}
\end{figure}


\bigskip\hrule\bigskip



\paragraph{Force Calculation Refresh}
\label{\detokenize{docs/userguide/explorersandutilitytools/statusbar/index-status_bar:force-calculation-refresh}}\label{\detokenize{docs/userguide/explorersandutilitytools/statusbar/index-status_bar:id6}}
Refreshes all calculations on the active Workspace.

\begin{figure}[H]
\centering
\capstart

\noindent\sphinxincludegraphics{{status_bar_utility_4}.PNG}
\caption{Force Calculation Refresh}\label{\detokenize{docs/userguide/explorersandutilitytools/statusbar/index-status_bar:id9}}\end{figure}


\paragraph{Auto Calculate}
\label{\detokenize{docs/userguide/explorersandutilitytools/statusbar/index-status_bar:auto-calculate}}\label{\detokenize{docs/userguide/explorersandutilitytools/statusbar/index-status_bar:auto-calulate}}
Toggles the calculation engine on or off.

Changing circuit properties will cause recalculations in a variety of areas.  This can be an expensive process and can cause the application to slow down.

This may not be appropriate during a presentation.

Turning off the calculation engine enables the designer to stop this process.

\begin{figure}[H]
\centering
\capstart

\noindent\sphinxincludegraphics{{status_bar_utility_5}.PNG}
\caption{Auto Calculate - Off}\label{\detokenize{docs/userguide/explorersandutilitytools/statusbar/index-status_bar:id10}}\end{figure}

\index{Is there a way to Disable all Flags?@\spxentry{Is there a way to Disable all Flags?}}\ignorespaces 

\paragraph{Toggle Flags}
\label{\detokenize{docs/userguide/explorersandutilitytools/statusbar/index-status_bar:toggle-flags}}\label{\detokenize{docs/userguide/explorersandutilitytools/statusbar/index-status_bar:index-0}}\label{\detokenize{docs/userguide/explorersandutilitytools/statusbar/index-status_bar:id7}}
The Toggle Flags button will toggle the visibility settings of all Flags.

\begin{figure}[H]
\centering
\capstart

\noindent\sphinxincludegraphics{{status_bar_utility_6}.PNG}
\caption{Flags - Off}\label{\detokenize{docs/userguide/explorersandutilitytools/statusbar/index-status_bar:id11}}\end{figure}


\paragraph{Version}
\label{\detokenize{docs/userguide/explorersandutilitytools/statusbar/index-status_bar:version}}\label{\detokenize{docs/userguide/explorersandutilitytools/statusbar/index-status_bar:id8}}
Tracks the current version of THRUX.


\section{\sphinxstylestrong{Recovery Options}}
\label{\detokenize{docs/userguide/index-user_guide:recovery-options}}\label{\detokenize{docs/userguide/index-user_guide:id11}}
THRUX models are stored in the cloud and are periodically backed up.  If THRUX crashes, reopen THRUX.  A recovery window will prompt to restore to the latest restore point, or an earlier restore point.

\begin{figure}[H]
\centering

\noindent\sphinxincludegraphics{{restore-1}.PNG}
\end{figure}

Another option is to restore a model to a restore point via the Open Project window.  Use the Restore button (restore icon) and then choose a restore point.

\begin{figure}[H]
\centering

\noindent\sphinxincludegraphics{{restore-2}.PNG}
\end{figure}


\section{\sphinxstylestrong{Automatic Updates}}
\label{\detokenize{docs/userguide/index-user_guide:automatic-updates}}\label{\detokenize{docs/userguide/index-user_guide:release-data}}
Whenever THRUX is opened, it automatically searches for updates.

Refer to the Version Number in the bottom right of the {\hyperref[\detokenize{docs/userguide/explorersandutilitytools/statusbar/index-status_bar:status-bar}]{\sphinxcrossref{\DUrole{std,std-ref}{Status Bar}}}}.

\begin{figure}[H]
\centering

\noindent\sphinxincludegraphics{{version}.PNG}
\end{figure}

To see an outline of updates between each Version, click on the information icon located in the top-right of the top menu bar.

\begin{figure}[H]
\centering

\noindent\sphinxincludegraphics{{release_data-1}.PNG}
\end{figure}

Use the arrows to navigate between each Version.

\begin{figure}[H]
\centering

\noindent\sphinxincludegraphics{{release_data-2}.PNG}
\end{figure}


\section{\sphinxstylestrong{Revit Interoperability}}
\label{\detokenize{docs/userguide/index-user_guide:revit-interoperability}}\label{\detokenize{docs/userguide/index-user_guide:id12}}
The Architectural information which comprise the Architectural Workspaces, {\hyperref[\detokenize{docs/userguide/definingarchitecturalelements/archelements/index-arch-elements:arch-elements}]{\sphinxcrossref{\DUrole{std,std-ref}{Arch. Elements}}}} and the {\hyperref[\detokenize{docs/userguide/definingarchitecturalelements/floorplans/index-floor-plans:floor-plans}]{\sphinxcrossref{\DUrole{std,std-ref}{Floor Plans}}}}, can be imported from a Revit model.

In addition, a THRUX model can be exported to Revit.  You can then fine-tune Equipment locations using Revit, and verify the integrity of the design using THRUX.

\begin{figure}[H]
\centering

\noindent\sphinxincludegraphics{{revit-interop-11}.PNG}
\end{figure}


\section{\sphinxstylestrong{AutoCAD Interoperability}}
\label{\detokenize{docs/userguide/index-user_guide:autocad-interoperability}}\label{\detokenize{docs/userguide/index-user_guide:id13}}
{\hyperref[\detokenize{docs/userguide/buildingelectricalmodel/schedules/index-schedules:schedules}]{\sphinxcrossref{\DUrole{std,std-ref}{Schedules}}}} are exportable to AutoCAD or Excel.

\begin{figure}[H]
\centering
\capstart

\noindent\sphinxincludegraphics{{schedules-exporting-11}.PNG}
\caption{Exporting Schedules to AutoCAD}\label{\detokenize{docs/userguide/index-user_guide:id31}}\end{figure}

\begin{figure}[H]
\centering
\capstart

\noindent\sphinxincludegraphics{{schedules-exporting-2}.PNG}
\caption{Exporting Schedules to AutoCAD}\label{\detokenize{docs/userguide/index-user_guide:id32}}\end{figure}


\chapter{\sphinxstylestrong{Frequently Asked Questions}}
\label{\detokenize{docs/index:frequently-asked-questions}}

\section{\sphinxstylestrong{General}}
\label{\detokenize{docs/faq:general}}\label{\detokenize{docs/faq:frequently-asked-questions}}\label{\detokenize{docs/faq::doc}}

\subsection{Is THRUX compatible with Revit?}
\label{\detokenize{docs/faq:is-thrux-compatible-with-revit}}
Yes, THRUX Pro provides the ability to Import/Export Revit models and THRUX models.


\subsection{How do I export my drawings to AutoCAD?}
\label{\detokenize{docs/faq:how-do-i-export-my-drawings-to-autocad}}
Open the {\hyperref[\detokenize{docs/userguide/buildingelectricalmodel/schedules/index-schedules::doc}]{\sphinxcrossref{\DUrole{doc}{Schedules}}}} Workspace and open the Schedules you would like to export.  Click the down arrow located in the top right, then click Export to AutoCAD.

See {\hyperref[\detokenize{docs/userguide/buildingelectricalmodel/schedules/index-schedules:exporting-schedules}]{\sphinxcrossref{\DUrole{std,std-ref}{Exporting Schedules}}}} for more details.


\subsection{How do I invite someone else to my project?}
\label{\detokenize{docs/faq:how-do-i-invite-someone-else-to-my-project}}\label{\detokenize{docs/faq:faq-invitations}}
Follow the instructions {\hyperref[\detokenize{docs/userguide/projectmanagement/accessibility/index-accessibility:accessibility}]{\sphinxcrossref{\DUrole{std,std-ref}{here}}}}.


\subsection{I’m working in the One-Line.  How do I create equipment?}
\label{\detokenize{docs/faq:i-m-working-in-the-one-line-how-do-i-create-equipment}}
All Equipment need to have a source of power, so make sure that a source such as a Utility or Generator exists first.

Use the {\hyperref[\detokenize{docs/userguide/buildingelectricalmodel/one-line/index-one-line:one-line-adding-a-source}]{\sphinxcrossref{\DUrole{std,std-ref}{Setup Wizard}}}} to create a source.  Or, right-click inside the {\hyperref[\detokenize{docs/userguide/buildingelectricalmodel/one-line/index-one-line::doc}]{\sphinxcrossref{\DUrole{doc}{One-Line}}}}, click Add Source, then choose Utility or Generator.

If the source is not yet defined or known, create an Unhosted Equipment.

Right-click inside the Workspace, and then click Add Unhosted Equipment.  Use the wizard to create the Equipment.

See {\hyperref[\detokenize{docs/userguide/buildingelectricalmodel/one-line/index-one-line:one-line-adding-a-source}]{\sphinxcrossref{\DUrole{std,std-ref}{Adding a Source}}}} or {\hyperref[\detokenize{docs/userguide/buildingelectricalmodel/one-line/index-one-line:one-line-adding-equipment}]{\sphinxcrossref{\DUrole{std,std-ref}{Adding Equipment}}}} for examples.


\subsection{I’m not seeing everything on my Riser.  I created equipment in the One-Line.  How come it doesn’t show up?}
\label{\detokenize{docs/faq:i-m-not-seeing-everything-on-my-riser-i-created-equipment-in-the-one-line-how-come-it-doesn-t-show-up}}
Designers may perform loading calculations as part of the design process.

However, they may not want to show every piece of equipment on the Riser.

Use the {\hyperref[\detokenize{docs/userguide/buildingelectricalmodel/riser/index-riser:riser-toolbox}]{\sphinxcrossref{\DUrole{std,std-ref}{Riser Toolbox}}}} to display the {\hyperref[\detokenize{docs/userguide/buildingelectricalmodel/riser/index-riser:hidden-elements}]{\sphinxcrossref{\DUrole{std,std-ref}{Hidden Elements}}}}.


\subsection{I’ve created a Riser, but I’m starting to run out of space.  Is there any way to shift everything upwards?  If I move the floor will it update the elevations?  Will it affect my lengths and voltage drop calculations?}
\label{\detokenize{docs/faq:i-ve-created-a-riser-but-i-m-starting-to-run-out-of-space-is-there-any-way-to-shift-everything-upwards-if-i-move-the-floor-will-it-update-the-elevations-will-it-affect-my-lengths-and-voltage-drop-calculations}}
The Riser Floor Elevations are not connected to the visual spacing between Floors.

Refer {\hyperref[\detokenize{docs/userguide/buildingelectricalmodel/riser/index-riser:riser-floor-elevations}]{\sphinxcrossref{\DUrole{std,std-ref}{here}}}} for more information.


\subsection{Is there a way to create groups of equipment?}
\label{\detokenize{docs/faq:is-there-a-way-to-create-groups-of-equipment}}
Copying an Equipment will copy all of its downstream Equipment.  Select the Equipment and use CTRL+C to copy.

Then select a new source and use CTRL+V to paste.

See {\hyperref[\detokenize{docs/userguide/buildingelectricalmodel/one-line/index-one-line:one-line-copying-equipment}]{\sphinxcrossref{\DUrole{std,std-ref}{here}}}} for an example in the One-Line or {\hyperref[\detokenize{docs/userguide/buildingelectricalmodel/schedules/index-schedules:schedules-copying-equipment}]{\sphinxcrossref{\DUrole{std,std-ref}{here}}}} for an example in the {\hyperref[\detokenize{docs/userguide/buildingelectricalmodel/schedules/index-schedules:schedules}]{\sphinxcrossref{\DUrole{std,std-ref}{Schedules}}}}.


\subsection{Is there a way to create a single feed for multiple motors or elevators?}
\label{\detokenize{docs/faq:is-there-a-way-to-create-a-single-feed-for-multiple-motors-or-elevators}}
By default, equipment is created on an individual circuit basis.

However, it is possible to create a single circuit which feeds multiple motors.

Create a Motor as a piece of equipment.  Change the Quantity, and enter the load capacity for multiple motors.

A single OCPD and feeder will be sized based on the total load of the motors.

It is also possible to enter multiple motors with different load unit types.

\begin{figure}[H]
\centering
\capstart

\noindent\sphinxincludegraphics{{multiple-motors}.PNG}
\caption{Using a single circuit to feed multiple motors}\label{\detokenize{docs/faq:id3}}\end{figure}

The same is true for elevators.  If multiple elevator motors are selected, different diversity factors from the NEC are applied to the load.


\subsection{I’ve created a network, but I forgot to add a distribution board, or transfer switch, or some type of intermediate node.  How can I add this without deleting what I have?}
\label{\detokenize{docs/faq:i-ve-created-a-network-but-i-forgot-to-add-a-distribution-board-or-transfer-switch-or-some-type-of-intermediate-node-how-can-i-add-this-without-deleting-what-i-have}}
It is possible to rehost circuits by dragging and dropping them from one source to another.

Simply rehost a section of the network to another source, then create the intermediate equipment, and reattach the network to the intermediate equipment.

See {\hyperref[\detokenize{docs/userguide/buildingelectricalmodel/one-line/index-one-line:one-line-rehosting}]{\sphinxcrossref{\DUrole{std,std-ref}{here}}}} for an example in the One-Line or {\hyperref[\detokenize{docs/userguide/buildingelectricalmodel/schedules/index-schedules:schedules-rehosting}]{\sphinxcrossref{\DUrole{std,std-ref}{here}}}} for an example in the Schedules.


\subsection{How do I create a bus duct?}
\label{\detokenize{docs/faq:how-do-i-create-a-bus-duct}}
Select an Equipment, then select Add Equipment to create a Bus Duct.

See {\hyperref[\detokenize{docs/userguide/buildingelectricalmodel/one-line/index-one-line:one-line-bus-duct}]{\sphinxcrossref{\DUrole{std,std-ref}{here}}}} for an example in the One-Line.


\subsection{How do I connect a transfer switch?}
\label{\detokenize{docs/faq:how-do-i-connect-a-transfer-switch}}
After a transfer switch is created, connect its sources by selecting Add Equipment, and then click the Existing dropdown to select the transfer switch.

See {\hyperref[\detokenize{docs/userguide/buildingelectricalmodel/one-line/index-one-line:one-line-transfer-switch}]{\sphinxcrossref{\DUrole{std,std-ref}{here}}}} for more details.  This is also available in the {\hyperref[\detokenize{docs/userguide/buildingelectricalmodel/schedules/index-schedules::doc}]{\sphinxcrossref{\DUrole{doc}{Schedules}}}} Workspace.


\subsection{I’ve created my distribution network, but I haven’t assigned any Rooms yet.  Is there a way to move my Equipment into Rooms?}
\label{\detokenize{docs/faq:i-ve-created-my-distribution-network-but-i-haven-t-assigned-any-rooms-yet-is-there-a-way-to-move-my-equipment-into-rooms}}
In the Floor Plans, there is a Move Equipment function.

See {\hyperref[\detokenize{docs/userguide/definingarchitecturalelements/floorplans/index-floor-plans:floor-plans-move-equipment}]{\sphinxcrossref{\DUrole{std,std-ref}{here}}}}.

An alternative is to use the Riser.  Place the Rooms on the Riser.  Then drag the Equipment into their Room locations.


\subsection{I have an existing building which I would like to model.  Is it appropriate to use THRUX?  How would I start?  What’s the fastest way to do that?}
\label{\detokenize{docs/faq:i-have-an-existing-building-which-i-would-like-to-model-is-it-appropriate-to-use-thrux-how-would-i-start-what-s-the-fastest-way-to-do-that}}
Start with {\hyperref[\detokenize{docs/userguide/explorersandutilitytools/statusbar/index-status_bar:design-assistance}]{\sphinxcrossref{\DUrole{std,std-ref}{Design Assistance}}}} turned on.

Model each piece of equipment, then turn off Design Assistance and enter the information manually.

Also, make use of the {\hyperref[\detokenize{docs/userguide/buildingelectricalmodel/one-line/index-one-line:one-line-load-calculations}]{\sphinxcrossref{\DUrole{std,std-ref}{Load Calculations}}}} to analyze different loading conditions.


\subsection{I’ve turned Design Assistance off and manually changed feeder sizes.  Is there a way to recalculate a circuit’s code-minimum requirements?}
\label{\detokenize{docs/faq:i-ve-turned-design-assistance-off-and-manually-changed-feeder-sizes-is-there-a-way-to-recalculate-a-circuit-s-code-minimum-requirements}}
One option is to use Reset to Code Minimum.  See {\hyperref[\detokenize{docs/userguide/buildingelectricalmodel/one-line/index-one-line:one-line-reset-to-code-minimum}]{\sphinxcrossref{\DUrole{std,std-ref}{here}}}} for an example in the One-Line and {\hyperref[\detokenize{docs/userguide/buildingelectricalmodel/schedules/index-schedules:schedules-reset-to-code-minimum}]{\sphinxcrossref{\DUrole{std,std-ref}{here}}}} for an example in the Schedules.

Another option is to turn Design Assistance on, and then change the Load Capacity.


\subsection{How do I see what’s included in each update?}
\label{\detokenize{docs/faq:how-do-i-see-what-s-included-in-each-update}}
See {\hyperref[\detokenize{docs/userguide/index-user_guide:release-data}]{\sphinxcrossref{\DUrole{std,std-ref}{Release Data}}}}.


\section{\sphinxstylestrong{Architectural}}
\label{\detokenize{docs/faq:architectural}}

\subsection{Do I need to create the Architectural Elements or do I need to use the Floor Plans Workspace?}
\label{\detokenize{docs/faq:do-i-need-to-create-the-architectural-elements-or-do-i-need-to-use-the-floor-plans-workspace}}
No.  These Workspaces aid in the design process and allow the designer to quickly alter the locations of Equipment in their design as the Architectural Elements change.

These Workspaces aid in calculating distances between Equipment, that affect point-to-point calculations.

Though it is highly recommended to use these Workspaces, it is also possible to manually enter all feeder and branch lengths.


\subsection{I’m working in the Floor Plans.  How do I create my columns and floors?}
\label{\detokenize{docs/faq:i-m-working-in-the-floor-plans-how-do-i-create-my-columns-and-floors}}
Use the Setup Wizard to create the XComp and YComp (column) components and Floors.  Use the Grid Editor to modify the columns or manually modify these components in the {\hyperref[\detokenize{docs/userguide/definingarchitecturalelements/archelements/index-arch-elements:arch-elements}]{\sphinxcrossref{\DUrole{std,std-ref}{Arch. Elements}}}} Workspace.

See {\hyperref[\detokenize{docs/userguide/index-user_guide:floor-plans-overview}]{\sphinxcrossref{\DUrole{std,std-ref}{Floor Plans}}}} or {\hyperref[\detokenize{docs/userguide/definingarchitecturalelements/archelements/index-arch-elements:arch-elements}]{\sphinxcrossref{\DUrole{std,std-ref}{Arch. Elements}}}} for more information.

It is also possible to import this information from an architectural Revit model.

See {\hyperref[\detokenize{docs/userguide/index-user_guide:revit-interoperability}]{\sphinxcrossref{\DUrole{std,std-ref}{here}}}} for more information.


\subsection{Is there a way to move my equipment in one Room to another location?}
\label{\detokenize{docs/faq:is-there-a-way-to-move-my-equipment-in-one-room-to-another-location}}
Use the {\hyperref[\detokenize{docs/userguide/definingarchitecturalelements/floorplans/index-floor-plans:floor-plans}]{\sphinxcrossref{\DUrole{std,std-ref}{Floor Plans}}}} to shift Room locations.  {\hyperref[\detokenize{docs/userguide/definingarchitecturalelements/floorplans/index-floor-plans:floor-plans-move-equipment}]{\sphinxcrossref{\DUrole{std,std-ref}{Move Equipment}}}} allows the ability to drag and drop Equipment into Rooms on the Floor Plans.

Or, manually modify the Room characteristics by using the {\hyperref[\detokenize{docs/userguide/definingarchitecturalelements/archelements/index-arch-elements:arch-elements}]{\sphinxcrossref{\DUrole{std,std-ref}{Arch. Elements}}}} Workspace.

All Equipment in the Room will update their feeder lengths as the location changes.


\subsection{I’m trying to enter decimal values into my floor heights.  Why am I getting an error?}
\label{\detokenize{docs/faq:i-m-trying-to-enter-decimal-values-into-my-floor-heights-why-am-i-getting-an-error}}
With floor heights, it is not always necessary to be very precise.  Lengths of conduit runs can be modified using the {\hyperref[\detokenize{docs/userguide/index-user_guide:manual-added-length}]{\sphinxcrossref{\DUrole{std,std-ref}{Manual Added Length}}}} property.  Voltage drop is also dependent on the size of the load, and the operating voltage.


\bigskip\hrule\bigskip



\section{\sphinxstylestrong{Electrical Calculations}}
\label{\detokenize{docs/faq:electrical-calculations}}

\subsection{What is Load Capacity?}
\label{\detokenize{docs/faq:what-is-load-capacity}}\label{\detokenize{docs/faq:load-capacity}}
Load Capacity is a custom size modified by the designer.

With Design Assistance on, many properties of Equipment change as a result of changing the Load Capacity.  For example, protective device sizes and conduit sizes are calculated based on the Load Capacity.

For example, if a designer entered 401A as the Load Capacity of a 3-ø Distribution Board, then a 600 AF, 450 AT breaker would be selected, fed via 3\#600 kCMil phase conductors.


\subsection{Is there a way to manually override a load?}
\label{\detokenize{docs/faq:is-there-a-way-to-manually-override-a-load}}\label{\detokenize{docs/faq:load-override}}
Net Load is a calculated value which is determined by the sum of the loads.

It can be overridden to manually override the calculated value.

Setting the Net Load will ignore all downstream loads, and set the load of the Equipment to be the specified value.

This can be helpful when designing around theoretical load conditions.

The overridden value will take precedence over normal calculated loads which are based on what the connected load.

Designers also have the option to diversify loads and apply custom diversity factors.  Load Overridden values take precedence over normal loading calculations.

In the example below, a Distribution Board feeds two Distribution Boards, each with two Generic Loads.  All Equipment do not have any \% Design Spare Capacity.  Setting the Load Override of an Equipment to be a value causes the Net Load to read that value.  If the Load Override is null, the Net Load consists of the normally calculated value based on the Connected Load.  Consequently, the Net Load of the upstream distribution board consists of the sum of its children’s Net Loads.

\begin{figure}[H]
\centering
\capstart

\noindent\sphinxincludegraphics{{load_override}.PNG}
\caption{Load Override}\label{\detokenize{docs/faq:id4}}\end{figure}

Load Override allows a designer to forecast the effects of a load on a system.  Use the Load Flow View to further analyze the system.

\begin{figure}[H]
\centering
\capstart

\noindent\sphinxincludegraphics{{One-Line-LoadOverride}.PNG}
\caption{Viewing the One-Line in Load Flow View}\label{\detokenize{docs/faq:id5}}\end{figure}


\subsection{What is \% Design Spare Capacity?}
\label{\detokenize{docs/faq:what-is-design-spare-capacity}}\label{\detokenize{docs/faq:design-spare-capacity}}
\% Design Spare Capacity is an adjustment factor that is based on the Code Demand Load.

For example, if a Distribution Board has a Connected Load of 20 kVA and also has a \% Design Spare Capacity of 50\%, the Net Load on the Distribution Board, DB-1, will be 30 kVA.  This is due to the 20 kVA Connected Load, in addition to the 10 kVA of Design Spare Capacity.

\begin{figure}[H]
\centering
\capstart

\noindent\sphinxincludegraphics{{design_spare_capacity}.PNG}
\caption{Design Spare Capacity}\label{\detokenize{docs/faq:id6}}\end{figure}


\subsection{What is Net Load?}
\label{\detokenize{docs/faq:what-is-net-load}}\label{\detokenize{docs/faq:net-load}}
Net Load consists of the connected load.  The connected load can be driven by equipment loads, residential loads, diversified loads, design spare capacity, and overridden loads.


\subsection{How do I enter the available SCC from the Utility?}
\label{\detokenize{docs/faq:how-do-i-enter-the-available-scc-from-the-utility}}\label{\detokenize{docs/faq:utility-short-circuit}}
Select the Utility source.  Under the Miscellaneous property grouping, enter the value under Available SCC (kA).

See {\hyperref[\detokenize{docs/userguide/buildingelectricalmodel/one-line/index-one-line:one-line-scc}]{\sphinxcrossref{\DUrole{std,std-ref}{here}}}} for an example.


\subsection{How is the Length of a Bus Duct Determined?}
\label{\detokenize{docs/faq:how-is-the-length-of-a-bus-duct-determined}}\label{\detokenize{docs/faq:bus-duct-length}}
A Bus Duct must be assigned to a Room.  All branch loads of the Bus Duct must also be assigned to a Room.

Pipe and wire is used until it terminates and transitions to a Bus Duct at the Room of the Bus Duct.  In other words, the length of the pipe and wire run is the distance between the Room of the source distribution equipment and the Room of the Bus Duct.

The vertical run of the Bus Duct is determined by the vertical distance between the Room of the branch load and Room of the Bus Duct.

Pipe and wire is used for branch circuits of the Bus Duct.  The length of the run is determined from the distance between the Room of the Bus Duct, and the Room of the load.


\subsection{How do bus duct voltage drop calculations work?}
\label{\detokenize{docs/faq:how-do-bus-duct-voltage-drop-calculations-work}}\label{\detokenize{docs/faq:bus-duct-calculations}}
Bus Duct voltage drop calculations are split into different sections.

The first section is the voltage drop calculated across the pipe and wire portion.  This length is dictated by the distance between the source Equipment Room location, and Bus Duct Room location.  The load is based on the Connected Load of the Bus Duct.

The second section is the voltage drop calculated across the vertical run of the Bus Duct.  The impedance of the Bus Duct is determined by its vertical run.  The vertical run is determined by the Bus Duct’s Room location, and the branch circuit load’s Room location.  The voltage drop across the vertical run is based on the Connected Load of the Bus Duct.  As multiple circuits are added to the Bus Duct, the load tapers throughout the length of the Bus Duct, and voltage drop calculations are recalculated.

The final section is the voltage drop calculated across the horizontal run to the branch circuit load.  The horizontal length of the run is determined by the Room of the Bus Duct and the Room of branch circuit load.  The load is determined by the branch circuit load.


\subsection{I know the area or square footage of a space.  How do I model that as a load?}
\label{\detokenize{docs/faq:i-know-the-area-or-square-footage-of-a-space-how-do-i-model-that-as-a-load}}\label{\detokenize{docs/faq:arch-power-density}}\label{\detokenize{docs/faq:residential-calculations}}
Floors and Rooms are Architectural Elements which can be used to perform load calculations.

Both of them have an Area and a SpaceType which allows you to calculate a load density.

This load density can be placed or attached to different points in your system.

The Area can be calculated or manually entered as a custom value.

\begin{figure}[H]
\centering
\capstart

\noindent\sphinxincludegraphics{{Architectural_Elements-1}.PNG}
\caption{The Floor tab of the Arch. Elements}\label{\detokenize{docs/faq:id7}}\end{figure}


\subsection{How do residential load calculations work?}
\label{\detokenize{docs/faq:how-do-residential-load-calculations-work}}
Residential load calculations are based on the variety of Appliances which require power in a Unit Type.  Apartments are each assigned a Unit Type.  Apartments can be grouped together into Apartment Packages.  Depending on the number of dwelling units, different diversity factors are applied to the load.

Appliances can be categorized into different classes such as Small Appliance, Fixed Appliance, Range, Dryer, Heating / Air Conditioning, as defined by the NEC.

Unit Types can hold a variety of Appliances.  Apartments are each assigned a Unit Type.  There are two types of load calculations performed on an Apartment: Standard and Optional.  Each calculation takes the sum of the load of each Unit Type’s Appliances and applies different diversity factors.

For Apartment Packages, the same two types of load calculations apply, based on the collections of Appliances.  However, there is a third calculation, known as a multi-dwelling calculation, which is used if there are more than two Apartments.  The multi-dwelling calculation applies a demand factor based on the number of Apartments.

The overall load of the Apartment Package is determined by the minimum of the three (3) load calculations.


\subsection{How do I model a tap?}
\label{\detokenize{docs/faq:how-do-i-model-a-tap}}
To model a tap, create a {\hyperref[\detokenize{docs/faq:tap-node}]{\sphinxcrossref{\DUrole{std,std-ref}{Tap Node}}}} or {\hyperref[\detokenize{docs/faq:bus-node}]{\sphinxcrossref{\DUrole{std,std-ref}{Bus Node}}}} in between the source and the load.


\subsubsection{Bus Node}
\label{\detokenize{docs/faq:bus-node}}\label{\detokenize{docs/faq:id1}}
To model a bus node, create a Bus Node in between the source and the load.

\begin{figure}[H]
\centering
\capstart

\noindent\sphinxincludegraphics{{bus_node-1}.PNG}
\caption{Creating a Bus Node in between an existing circuit}\label{\detokenize{docs/faq:id8}}\end{figure}


\subsubsection{Tap Node}
\label{\detokenize{docs/faq:tap-node}}\label{\detokenize{docs/faq:id2}}
To model a tap, create a Tap Node in between the source and the load.

\begin{figure}[H]
\centering
\capstart

\noindent\sphinxincludegraphics{{tap_node-1}.PNG}
\caption{Creating a Tap Node}\label{\detokenize{docs/faq:id9}}\end{figure}


\subsection{How do I Diversify Loads?}
\label{\detokenize{docs/faq:how-do-i-diversify-loads}}
It is possible to apply custom diversities to different sections of the distribution network.

See {\hyperref[\detokenize{docs/userguide/definingarchitecturalelements/archelements/index-arch-elements:diversification}]{\sphinxcrossref{\DUrole{std,std-ref}{Diversification}}}}.


\subsection{Is there a way to implement a conditional constraint for sizing Equipment of loads greater than a specific amperage to require copper over aluminum?}
\label{\detokenize{docs/faq:is-there-a-way-to-implement-a-conditional-constraint-for-sizing-equipment-of-loads-greater-than-a-specific-amperage-to-require-copper-over-aluminum}}
Coming soon.


\chapter{Definitions}
\label{\detokenize{docs/index:definitions}}

\section{Definitions}
\label{\detokenize{docs/definitions/index-definitions:definitions}}\label{\detokenize{docs/definitions/index-definitions:id1}}\label{\detokenize{docs/definitions/index-definitions::doc}}

\subsection{Default Model Parameters}
\label{\detokenize{docs/definitions/index-definitions:default-model-parameters}}\label{\detokenize{docs/definitions/index-definitions:default-model-parameters-definitions}}

\subsubsection{Circuit}
\label{\detokenize{docs/definitions/index-definitions:circuit}}

\begin{savenotes}\sphinxattablestart
\centering
\begin{tabulary}{\linewidth}[t]{|T|T|}
\hline
\sphinxstyletheadfamily 
\sphinxstylestrong{Property}
&\sphinxstyletheadfamily 
\sphinxstylestrong{Definition}
\\
\hline
Conductor Type
&
Specify the default conductor material for all circuits.  The default is copper (Cu.).
\\
\hline
Conduit Type
&
Specify the default conduit type of all circuits.  The default is RMC (Rigid Metal Conduit), with options for EMT, PVC (Sch. 40), PVC (Sch. 80), and LFMC.
\\
\hline
Manual Added Length
&
Manual Added Length is supplemental to Calc. Length.  The Net Length of a circuit is the sum of the Calc. Length and the Manual Added Length.
\\
\hline
Neutrals Per
&
Specify the default number of neutrals per circuit.  The default number is one.
\\
\hline
Grounds Per
&
Specify the default number of grounds per circuit.  The default number is one.
\\
\hline
Non-linear
&
A boolean property representing non-linear loads.  A non-linear load changes impedance with applied voltage.  This means the load will not be sinusoidal, and cause harmonics.
\\
\hline
\end{tabulary}
\par
\sphinxattableend\end{savenotes}


\subsubsection{MV Circuit}
\label{\detokenize{docs/definitions/index-definitions:mv-circuit}}

\begin{savenotes}\sphinxattablestart
\centering
\begin{tabulary}{\linewidth}[t]{|T|T|}
\hline
\sphinxstyletheadfamily 
\sphinxstylestrong{Property}
&\sphinxstyletheadfamily 
\sphinxstylestrong{Definition}
\\
\hline
Temperature
&
Specify the default temperature rating for all medium voltage circuits.  The default is MV-90.
\\
\hline
Conductor Assembly
&
Specify the conductor assembly of the medium voltage circuit.  The default is single conductor.
\\
\hline
\end{tabulary}
\par
\sphinxattableend\end{savenotes}


\subsubsection{Mechanical}
\label{\detokenize{docs/definitions/index-definitions:mechanical}}

\begin{savenotes}\sphinxattablestart
\centering
\begin{tabulary}{\linewidth}[t]{|T|T|}
\hline
\sphinxstyletheadfamily 
\sphinxstylestrong{Property}
&\sphinxstyletheadfamily 
\sphinxstylestrong{Definition}
\\
\hline
VFD
&
A boolean property which represents if a Mechanical Equipment has a VFD (variable frequency drive).
\\
\hline
\end{tabulary}
\par
\sphinxattableend\end{savenotes}


\subsubsection{Distribution Board}
\label{\detokenize{docs/definitions/index-definitions:distribution-board}}

\begin{savenotes}\sphinxattablestart
\centering
\begin{tabulary}{\linewidth}[t]{|T|T|}
\hline
\sphinxstyletheadfamily 
\sphinxstylestrong{Property}
&\sphinxstyletheadfamily 
\sphinxstylestrong{Definition}
\\
\hline
Qty Breakers
&
Specify the quantity of breakers for each Distribution Board.  By default, each Distribution Board has space for ten breakers.
\\
\hline
Neutral Bus
&
Specify if a Distribution Board has a neutral bus.  By default, each Distribution Board has a neutral bus.
\\
\hline
Ground Bus
&
Specify if a Distribution Board has a ground bus.  By default, each Distribution Board has a ground bus.
\\
\hline
200\% Neutral
&
Specify if a Distribution Board has a 200\% neutral bus.
\\
\hline
Iso. Ground Bus
&
Specify if a Distribution Board has an isolated ground bus.
\\
\hline
Enclosure Type
&
Specify the type of enclosure.
\\
\hline
Voltmeter
&
By default, each Distribution board has a Voltmeter.
\\
\hline
SPD
&
By default, each Distribution board has a surge protective device (SPD).
\\
\hline
Power Meter
&
By default, each Distribution board has a Power Meter.
\\
\hline
\end{tabulary}
\par
\sphinxattableend\end{savenotes}


\subsubsection{Bus Duct}
\label{\detokenize{docs/definitions/index-definitions:bus-duct}}

\begin{savenotes}\sphinxattablestart
\centering
\begin{tabulary}{\linewidth}[t]{|T|T|}
\hline
\sphinxstyletheadfamily 
\sphinxstylestrong{Property}
&\sphinxstyletheadfamily 
\sphinxstylestrong{Definition}
\\
\hline
Qty Breakers
&
Specify the quantity of breakers for each Bus Duct.  By default, each Bus Duct has space for ten breakers.
\\
\hline
Neutral Bus
&
Specify if a Bus Duct has a neutral bus.  By default, each Distribution Board has a neutral bus.
\\
\hline
Ground Bus
&
Specify if a Bus Duct has a ground bus.  By default, each Bus Duct has a ground bus.
\\
\hline
200\% Neutral
&
Specify if a Bus Duct has a 200\% neutral bus.
\\
\hline
Iso. Ground Bus
&
Specify if a Bus Duct has an isolated ground bus.
\\
\hline
Top-Fed
&
Specify if a Bus Duct is constructed via top fed installation.
\\
\hline
\end{tabulary}
\par
\sphinxattableend\end{savenotes}


\subsubsection{XFMR}
\label{\detokenize{docs/definitions/index-definitions:xfmr}}

\begin{savenotes}\sphinxattablestart
\centering
\begin{tabulary}{\linewidth}[t]{|T|T|}
\hline
\sphinxstyletheadfamily 
\sphinxstylestrong{Property}
&\sphinxstyletheadfamily 
\sphinxstylestrong{Definition}
\\
\hline
Overload Rating
&
A tranformer can be overloaded and still operate under certain conditions. By default, the overload rating for each transformer is one.
\\
\hline
\% Impedance
&
The percent impedance is a measure of the volts dropped when a transformer operates under full load due to winding resistance and leakage reactance expressed as a percentage of the rated voltage.
\\
\hline
Ground Bus
&
Specify if a transformer has a ground bus.  By default, each Bus Duct has a ground bus.
\\
\hline
Conductor Type
&
Specify the conductor type for each transformer.  The default material is copper (Cu.).
\\
\hline
XFMR Type
&
The default transformer type is dry type.  Other options include Cast Coil, VPI, and Liquid.
\\
\hline
K Rating
&
The K rating of a transformer weighs the effects of harmonic load currents according to their effects on transformer heating.  A K rating of 1.0 indicates a linear load.  The higher the K rating, the greater the harmonic heating effects.
\\
\hline
Primary Winding
&
Specify the primary winding configuration to be Delta or Wye.
\\
\hline
Sec. Winding
&
Specify the primary winding configuration to be Delta or Wye.
\\
\hline
Primary BIL
&
An electrical system is subject to lightning impulses or overvoltage conditions.  Energy is discharged through surge protective devices before equipment become damaged.
The BIL (basic impulse level) affects the amount of insulation which protects the transformer from being damaged by surge voltages.
\\
\hline
Secondary BIL
&
Specify the BIL rating of the secondary winding.
\\
\hline
Ground Type
&
Specify the ground type of a transformer to be solid, low impedance, or high impedance.
\\
\hline
Insulation Rating
&
Specify the rating of the insulation system of the transformer.  The insulation rating is the maximum interal operating temperature before it begins to fail.
\\
\hline
Temp. Rise
&
The temperature rise of a transformer is defined as the average temperature rise above the ambient temperature due to operating conditions under its nameplate load rating.
\\
\hline
Surge Arrestor
&
Specify if a transformer has a surge arrestor.
\\
\hline
Elect. Shield
&
Specify if a transformer has an electrostatic shield.
\\
\hline
Snubber
&
Specify if a transformer has a snubber, which is used to suppress voltage transients.
\\
\hline
\end{tabulary}
\par
\sphinxattableend\end{savenotes}


\subsubsection{Panelboard}
\label{\detokenize{docs/definitions/index-definitions:panelboard}}

\begin{savenotes}\sphinxattablestart
\centering
\begin{tabulary}{\linewidth}[t]{|T|T|}
\hline
\sphinxstyletheadfamily 
\sphinxstylestrong{Property}
&\sphinxstyletheadfamily 
\sphinxstylestrong{Definition}
\\
\hline
Qty Poles
&
Specify the number of pole positions for each Panelboard
\\
\hline
Main Breaker
&
Specify if each Panelboard will have a main breaker
\\
\hline
Bolt-On Breaker
&
Specify if each Panelboard will have a bolt on breaker
\\
\hline
Neutral Bus
&
Specify if each Panelboard will have a neutral bus.
\\
\hline
Ground Bus
&
Specify if each Panelboard will have a ground bus.
\\
\hline
200\% Neutral
&
Specify if a Distribution Board has a 200\% neutral bus.
\\
\hline
Iso. Ground Bus
&
Specify if a Distribution Board has an isolated ground bus.
\\
\hline
Enclosure Type
&
Specify the type of enclosure.
\\
\hline
Conductor Type
&
Specify the default conductor material for all Panelboards.  The default is copper (Cu.).
\\
\hline
Door-in-Door
&
By default, each Panelboard has a Door-in-Door construction.
\\
\hline
Feed-Through
&
Specify if a Panelboard is a feed-through panel.
\\
\hline
Stainless Steel
&
Specify if a Panelboard is a stainless steel construction.
\\
\hline
Flush-Mount
&
Specify if a Panelboard is flush mounted.
\\
\hline
Split Bus Meter
&
Specify if a Panelboard has a split bus meter.
\\
\hline
Branch Ckt. Meter
&
Specify if a Panelboard has a branch circuit meter.
\\
\hline
\end{tabulary}
\par
\sphinxattableend\end{savenotes}


\subsubsection{ATS}
\label{\detokenize{docs/definitions/index-definitions:ats}}

\begin{savenotes}\sphinxattablestart
\centering
\begin{tabulary}{\linewidth}[t]{|T|T|}
\hline
\sphinxstyletheadfamily 
\sphinxstylestrong{Property}
&\sphinxstyletheadfamily 
\sphinxstylestrong{Definition}
\\
\hline
Enclosure Type
&
Specify the type of enclosure.
\\
\hline
Transition
&
Specify if an ATS is open or closed transition.  Open transition uses the break-before-make principle.  Closed transition uses the make-before-break principle.
\\
\hline
Neutral Contruction
&
Specify if the neutral construction of an ATS is solid, switched, or overlapping.
\\
\hline
Bypass Isolation
&
Specify if an ATS is bypass isolation. Bypass isolation allows the ability to perform maintenance, while still serving the load.
\\
\hline
Neutral Bus
&
Specify if an ATS has a neutral bus.  By default, each ATS has a neutral bus.
\\
\hline
Ground Bus
&
Specify if an ATS has a ground bus.  By default, each ATS has a ground bus.
\\
\hline
Static
&
Specify if an ATS is a STS (static transfer switch).  Static transfer switches allow for instantaneous transfer of power between sources.
\\
\hline
\end{tabulary}
\par
\sphinxattableend\end{savenotes}


\subsubsection{Utility}
\label{\detokenize{docs/definitions/index-definitions:utility}}

\begin{savenotes}\sphinxattablestart
\centering
\begin{tabulary}{\linewidth}[t]{|T|T|}
\hline
\sphinxstyletheadfamily 
\sphinxstylestrong{Property}
&\sphinxstyletheadfamily 
\sphinxstylestrong{Definition}
\\
\hline
Voltage Distortion
&
A tranformer can be overloaded and still operate under certain conditions. By default, the overload rating for each transformer is one.
\\
\hline
\end{tabulary}
\par
\sphinxattableend\end{savenotes}


\subsubsection{Generator}
\label{\detokenize{docs/definitions/index-definitions:generator}}

\begin{savenotes}\sphinxattablestart
\centering
\begin{tabulary}{\linewidth}[t]{|T|T|}
\hline
\sphinxstyletheadfamily 
\sphinxstylestrong{Property}
&\sphinxstyletheadfamily 
\sphinxstylestrong{Definition}
\\
\hline
Subtransient Reactance
&
The percent impedance is a measure of the volts dropped when a transformer operates under full load due to winding resistance and leakage reactance expressed as a percentage of the rated voltage.
\\
\hline
\end{tabulary}
\par
\sphinxattableend\end{savenotes}


\subsection{Calculation Settings}
\label{\detokenize{docs/definitions/index-definitions:calculation-settings}}\label{\detokenize{docs/definitions/index-definitions:calculation-settings-definitions}}

\begin{savenotes}\sphinxattablestart
\centering
\begin{tabulary}{\linewidth}[t]{|T|T|}
\hline
\sphinxstyletheadfamily 
\sphinxstylestrong{Property Name}
&\sphinxstyletheadfamily 
\sphinxstylestrong{Definition}
\\
\hline
XFMR OCP Multiplier
&
Multiplier is a factor which is multiplied by the Load Capacity of a transformer to size its protective device.
\\
\hline
Mech. OCP Multiplier
&
Multiplier is a factor which is multiplied by the Load Capacity of a Mechanical Equipment to size its protective device.
\\
\hline
Fire Pump OCP Multiplier
&
Multiplier is a factor which is multiplied by the Load Capacity of a Fire Pump to size its protective device.
\\
\hline
Temp Rating Threshold 60/75 C: (Amps)
&
Specify the amperage which is the limiting factor between switching from the 60 degrees celsius column to the 75 degrees celsius column (NEC 310.16).
\\
\hline
\end{tabulary}
\par
\sphinxattableend\end{savenotes}


\subsection{Flag Settings Definitions}
\label{\detokenize{docs/definitions/index-definitions:flag-settings-definitions}}\label{\detokenize{docs/definitions/index-definitions:id2}}

\begin{savenotes}\sphinxattablestart
\centering
\begin{tabulary}{\linewidth}[t]{|T|T|}
\hline
\sphinxstyletheadfamily 
\sphinxstylestrong{Property}
&\sphinxstyletheadfamily 
\sphinxstylestrong{Definition}
\\
\hline
Normal Priority Voltage Drop Threshold
&
A Flag is raised when the VD\% Net for a piece of distribution equipment exceeds a Specifies percentage.  The default value is 2\%.  The default threshold for end-of-line Equipment is 5\%.
\\
\hline
Non-Normal Priorty Voltage Drop Threshold
&
A Flag is raised when the VD\% Net for a piece of non-normal distribution equipment, such as a generator or other secondary source of power, exceeds a Specifies percentage.  The default value is 2\%.  The default threshold for end-of-line equipment is 5\%.
\\
\hline
Name
&
A Flag is raised when a piece of Equipment does not have a name.
\\
\hline
Breaker Undersized
&
A Flag is raised when a breaker’s trip setting is smaller than its frame size.
\\
\hline
Frame Undersized
&
A Flag is raised when an OCPD’s frame size will not accommodate the associated {\hyperref[\detokenize{docs/faq:load-capacity}]{\sphinxcrossref{\DUrole{std,std-ref}{Load Capacity}}}}.
\\
\hline
kAIC Undersized
&
A Flag is raised when the kAIC rating of an Equipment will not accommodate the SCC Net value.
\\
\hline
Sets Per
&
A Flag is raised when the current amount of sets of conductors per circuit will not accommodate the {\hyperref[\detokenize{docs/faq:load-capacity}]{\sphinxcrossref{\DUrole{std,std-ref}{Load Capacity}}}}.
\\
\hline
Phase Undersized
&
A Flag is raised when the phase size of a conductor will not accommodate the {\hyperref[\detokenize{docs/faq:load-capacity}]{\sphinxcrossref{\DUrole{std,std-ref}{Load Capacity}}}}.
\\
\hline
Ground Undersized
&
A Flag is raised when a the size of the ground conductor will not accommodate the {\hyperref[\detokenize{docs/faq:load-capacity}]{\sphinxcrossref{\DUrole{std,std-ref}{Load Capacity}}}}.
\\
\hline
Conduit Undersized
&
A Flag is raised when the amount of the cross-sectional area that is occupied by a cable or cables exceeds the maximum fill ratio.
\\
\hline
Voltage Drop Error
&
A Flag is raised when the VD\% Net maximum threshold is exceeded.
\\
\hline
Neutral Provision
&
A Flag is raised when the neutral conductor is undersized.
\\
\hline
Equipment Overloaded
&
A Flag is raised when a piece of distribution Equipment is overloaded.
\\
\hline
Equipment Voltage Drop Error
&
A Flag is raised when the VD\% Net exceeds the threshold value.
\\
\hline
\end{tabulary}
\par
\sphinxattableend\end{savenotes}


\subsection{Custom Design Assistance Settings}
\label{\detokenize{docs/definitions/index-definitions:custom-design-assistance-settings}}\label{\detokenize{docs/definitions/index-definitions:custom-design-assistance-settings-definitions}}

\begin{savenotes}\sphinxattablestart
\centering
\begin{tabulary}{\linewidth}[t]{|T|T|}
\hline
\sphinxstyletheadfamily 
\sphinxstylestrong{Property}
&\sphinxstyletheadfamily 
\sphinxstylestrong{Definition}
\\
\hline
Frame Size
&
Deselect Frame Size to omit the Frame Size from being recalculated upon changes to circuit properties.
\\
\hline
Trip Size
&
Deselect Trip Size to omit the Trip Size from being recalculated upon changes to circuit properties.
\\
\hline
kAIC Rating
&
Deselect kAIC Rating to omit the kAIC Rating from being recalculated upon changes to circuit properties.
\\
\hline
Temperature Rating
&
Deselect Temperature Rating to omit the Temperature Rating from being recalculated upon changes to circuit properties.
\\
\hline
Sets Per
&
Deselect Sets Per to omit the Sets Per or quantity of sets of conductors from being recalculated upon changes to circuit properties.
\\
\hline
Phase Size
&
Deselect Phase Size to omit the Phase Size from being recalculated upon changes to circuit properties.
\\
\hline
Neutral Size
&
Deselect Neutral Size to omit the Neutral Size from being recalculated upon changes to circuit properties.
\\
\hline
Ground Size
&
Deselect Ground Size to omit the Ground Size from being recalculated upon changes to circuit properties.
\\
\hline
Conduit Size
&
Deselect Conduit Size to omit the Conduit Size from being recalculated upon changes to circuit properties.
\\
\hline
\end{tabulary}
\par
\sphinxattableend\end{savenotes}


\subsection{Architectural Elements}
\label{\detokenize{docs/definitions/index-definitions:architectural-elements}}\label{\detokenize{docs/definitions/index-definitions:architectural-elements-definitions}}

\subsubsection{X Comp/Y Comp}
\label{\detokenize{docs/definitions/index-definitions:x-comp-y-comp}}\label{\detokenize{docs/definitions/index-definitions:ycomp-definition}}\label{\detokenize{docs/definitions/index-definitions:xcomp-definition}}
X Comp and Y Comp are used to establish a location for Rooms.  Equipment is placed inside Rooms, which are affected by point-to-point calculations.


\begin{savenotes}\sphinxattablestart
\centering
\begin{tabulary}{\linewidth}[t]{|T|T|T|}
\hline
\sphinxstyletheadfamily 
\sphinxstylestrong{Property}
&\sphinxstyletheadfamily 
\sphinxstylestrong{Data Type}
&\sphinxstyletheadfamily 
\sphinxstylestrong{Definition}
\\
\hline
Name
&
string
&
The name of the column.
\\
\hline
Dimension
&
int
&
The distance from the origin.
\\
\hline
\end{tabulary}
\par
\sphinxattableend\end{savenotes}


\subsubsection{Floor}
\label{\detokenize{docs/definitions/index-definitions:floor}}\label{\detokenize{docs/definitions/index-definitions:floor-definition}}
Floors are used for power density calculations and to also contain Rooms.


\begin{savenotes}\sphinxattablestart
\centering
\begin{tabulary}{\linewidth}[t]{|T|T|T|}
\hline
\sphinxstyletheadfamily 
\sphinxstylestrong{Property}
&\sphinxstyletheadfamily 
\sphinxstylestrong{Data Type}
&\sphinxstyletheadfamily 
\sphinxstylestrong{Definition}
\\
\hline
Name
&
string
&
The name of the column.
\\
\hline
Elevation
&
int
&
The distance from the origin.
\\
\hline
Height
&
int
&
The distance from the elevation of the level below.
\\
\hline
Space Type
&
SpaceType
&
The Floor’s SpaceType.
\\
\hline
Left Bound
&
XGrid
&
The leftmost boundary of the Floor.
\\
\hline
Right Bound
&
XGrid
&
The rightmost boundary of the Floor.
\\
\hline
Top Bound
&
YGrid
&
The topmost boundary of the Floor.
\\
\hline
Bottom Bound
&
YGrid
&
The bottommost boundary of the Floor.
\\
\hline
Calc SqFt
&
int
&
The area of the Floor’s defined boundaries.
\\
\hline
Manual SqFt
&
int
&
A custom value overriding the calculated area of the Floor.
\\
\hline
Override SqFt
&
int
&
A boolean value to use the calculated area or the manually defined area for determining the load of the Floor.
\\
\hline
Load
&
double
&
The total load or power density of the Floor determined by the SpaceType and the area of the Floor.
\\
\hline
\end{tabulary}
\par
\sphinxattableend\end{savenotes}


\subsubsection{Space Type}
\label{\detokenize{docs/definitions/index-definitions:space-type}}\label{\detokenize{docs/definitions/index-definitions:spacetype-definition}}
Space Types are used to assign power densities to architectural entities such as Floors and Rooms.

For example, Engineers perform load massing calculations in order to size their main distribution equipment.  They allocate power differently for floor loads, lighting loads and mechanical loads.

Space Types are customizable and a set of default load values are provided.  A variety of Space Types allows the Engineer to study and explore the power density requirements of a project.


\begin{savenotes}\sphinxattablestart
\centering
\begin{tabulary}{\linewidth}[t]{|T|T|T|}
\hline
\sphinxstyletheadfamily 
\sphinxstylestrong{Property}
&\sphinxstyletheadfamily 
\sphinxstylestrong{Data Type}
&\sphinxstyletheadfamily 
\sphinxstylestrong{Definition}
\\
\hline
Name
&
string
&
The name of the Space Type.
\\
\hline
Power
&
double
&
A factor used to allocate for floor loads.
\\
\hline
Lighting
&
double
&
A factor used to allocate for lighting loads.
\\
\hline
Mechanical
&
double
&
A factor used to allocate for mechanical loads.
\\
\hline
\end{tabulary}
\par
\sphinxattableend\end{savenotes}


\subsubsection{Room}
\label{\detokenize{docs/definitions/index-definitions:room}}\label{\detokenize{docs/definitions/index-definitions:room-definition}}
Rooms are used to establish locations for equipment.  Based on these locations, Engineers can perform their point-to-point calculations.  Rooms can also be packaged to create custom load packages.


\begin{savenotes}\sphinxattablestart
\centering
\begin{tabulary}{\linewidth}[t]{|T|T|T|}
\hline
\sphinxstyletheadfamily 
\sphinxstylestrong{Property}
&\sphinxstyletheadfamily 
\sphinxstylestrong{Data Type}
&\sphinxstyletheadfamily 
\sphinxstylestrong{Definition}
\\
\hline
Name
&
string
&
The name of the Room.
\\
\hline
SqFt
&
int
&
The area of the Room.
\\
\hline
Space Type
&
SpaceType
&
The SpaceType which is used to determine the load.
\\
\hline
X Column
&
XComp
&
The location of the Room along the x-axis.
\\
\hline
Y Column
&
YComp
&
The location of the Room along the y-axis.
\\
\hline
Floor
&
Floor
&
The Floor location of the Room which provides a z-coordinate.
\\
\hline
Load
&
double
&
The load determined by the area (SqFt) and the SpaceType.
\\
\hline
\end{tabulary}
\par
\sphinxattableend\end{savenotes}


\subsubsection{Riser}
\label{\detokenize{docs/definitions/index-definitions:riser}}\label{\detokenize{docs/definitions/index-definitions:riser-definition}}
A Riser is an entity used to offset conduit routes.  By default, the length of a conduit route is determined by the distance between the source and the load.  However, Engineers are often allocated Risers or shaft space to house their conduits or feeders.  The conduits are routed from their source, through the Riser and terminate at the load.


\begin{savenotes}\sphinxattablestart
\centering
\begin{tabulary}{\linewidth}[t]{|T|T|T|}
\hline
\sphinxstyletheadfamily 
\sphinxstylestrong{Property}
&\sphinxstyletheadfamily 
\sphinxstylestrong{Data Type}
&\sphinxstyletheadfamily 
\sphinxstylestrong{Definition}
\\
\hline
Name
&
string
&
The name of the Riser.
\\
\hline
X Column
&
XComp
&
The location of the Riser along the x-axis.
\\
\hline
Y Column
&
YComp
&
The location of the Riser along the y-axis.
\\
\hline
Top Bound
&
YGrid
&
The topmost boundary of the Riser.
\\
\hline
Bottom Bound
&
YGrid
&
The bottommost boundary of the Riser.
\\
\hline
\end{tabulary}
\par
\sphinxattableend\end{savenotes}


\subsubsection{Architectural Package}
\label{\detokenize{docs/definitions/index-definitions:architectural-package}}\label{\detokenize{docs/definitions/index-definitions:architectural-package-definition}}
Architectural Packages are used to model the load or power density of a group of architectural elements.


\begin{savenotes}\sphinxattablestart
\centering
\begin{tabulary}{\linewidth}[t]{|T|T|T|}
\hline
\sphinxstyletheadfamily 
\sphinxstylestrong{Property}
&\sphinxstyletheadfamily 
\sphinxstylestrong{Data Type}
&\sphinxstyletheadfamily 
\sphinxstylestrong{Definition}
\\
\hline
Name
&
string
&
The name of the Riser.
\\
\hline
NET PWR
&
double
&
The total load allocated for power determined by the total area and SpaceTypes.
\\
\hline
NET LTG
&
double
&
The total load allocated for lighting determined by the total area and SpaceTypes.
\\
\hline
NET MECH
&
double
&
The total load allocated for mechanical loads determined by the total area and SpaceTypes.
\\
\hline
NET ALLOC.
&
double
&
The total load allocated from additional Load Allocations.
\\
\hline
NET kVA
&
double
&
The total load determined by the sum of the NET PWR, NET LTG, NET MECH, and NET ALLOC.
\\
\hline
DRAW @ 480V
&
double
&
The current drawn from the NET kVA at 480V.
\\
\hline
DRAW @ 208V
&
double
&
The current drawn from the NET kVA at 208V.
\\
\hline
\end{tabulary}
\par
\sphinxattableend\end{savenotes}


\subsubsection{Appliance}
\label{\detokenize{docs/definitions/index-definitions:appliance}}\label{\detokenize{docs/definitions/index-definitions:appliance-definition}}
Appliances are used to calculate the load of residential projects and are grouped by classes as defined by the NEC.  Appliances are assigned to a Unit Type.


\begin{savenotes}\sphinxattablestart
\centering
\begin{tabulary}{\linewidth}[t]{|T|T|T|}
\hline
\sphinxstyletheadfamily 
\sphinxstylestrong{Property}
&\sphinxstyletheadfamily 
\sphinxstylestrong{Data Type}
&\sphinxstyletheadfamily 
\sphinxstylestrong{Definition}
\\
\hline
Name
&
string
&
The name of the Appliance.
\\
\hline
Class
&
int
&
The class of the Appliance as defined by the NEC.
\\
\hline
Voltage
&
int
&
The voltage of the Appliance.
\\
\hline
Poles
&
int
&
The number of hot conductors for the Appliance.
\\
\hline
Draw
&
int
&
The load of the Appliance.
\\
\hline
\end{tabulary}
\par
\sphinxattableend\end{savenotes}


\subsubsection{Unit Type}
\label{\detokenize{docs/definitions/index-definitions:unit-type}}\label{\detokenize{docs/definitions/index-definitions:unit-type-definition}}
Unit Types are used to group Appliances together in order to calculate the load of residential projects.  A Unit Type is assigned to an Apartment.


\begin{savenotes}\sphinxattablestart
\centering
\begin{tabulary}{\linewidth}[t]{|T|T|T|}
\hline
\sphinxstyletheadfamily 
\sphinxstylestrong{Property}
&\sphinxstyletheadfamily 
\sphinxstylestrong{Data Type}
&\sphinxstyletheadfamily 
\sphinxstylestrong{Definition}
\\
\hline
Name
&
string
&
The name of the Unit Type.
\\
\hline
Small-Appl Load
&
double
&
The total load of all Appliances which are classified as Small Appliance.
\\
\hline
Fixture Load
&
double
&
The total load of all Appliances which are classified as Fixture.
\\
\hline
Range Load
&
double
&
The total load of all Appliances which are classified as Range.
\\
\hline
Dryer Load
&
double
&
The total load of all Appliances which are classified as Dryer.
\\
\hline
HVAC Load
&
double
&
The total load of all Appliances which are classified as HVAC.
\\
\hline
\end{tabulary}
\par
\sphinxattableend\end{savenotes}


\subsubsection{Apartment}
\label{\detokenize{docs/definitions/index-definitions:apartment}}\label{\detokenize{docs/definitions/index-definitions:apartment-definition}}
An Apartment is assigned a Unit Type.  It also contains information regarding its location and loading information.  A group of Apartments form an Apartment Package.


\begin{savenotes}\sphinxattablestart
\centering
\begin{tabulary}{\linewidth}[t]{|T|T|T|}
\hline
\sphinxstyletheadfamily 
\sphinxstylestrong{Property}
&\sphinxstyletheadfamily 
\sphinxstylestrong{Data Type}
&\sphinxstyletheadfamily 
\sphinxstylestrong{Definition}
\\
\hline
Name
&
string
&
The name of the Apartment.
\\
\hline
Unit Type
&
double
&
The Unit Type of the Apartment.
\\
\hline
SqFt
&
int
&
The area of the Apartment.
\\
\hline
Voltage
&
int
&
The voltage of the Apartment.
\\
\hline
Poles
&
int
&
The number of hot conductors for the Apartment.
\\
\hline
X Column
&
XComp
&
The location of the Room along the x-axis.
\\
\hline
Y Column
&
YComp
&
The location of the Room along the y-axis.
\\
\hline
Floor
&
Floor
&
The Floor location of the Room which provides a z-coordinate.
\\
\hline
Optional-Calc
&
double
&
A calculation method determined by the NEC for calculating residential loads.
\\
\hline
Standard-Calc
&
double
&
A calculation method determined by the NEC for calculating residential loads.
\\
\hline
Multi. Dwell-Calc
&
double
&
A calculation method determined by the NEC for calculating residential loads.
\\
\hline
\end{tabulary}
\par
\sphinxattableend\end{savenotes}


\subsubsection{Apartment Package}
\label{\detokenize{docs/definitions/index-definitions:apartment-package}}\label{\detokenize{docs/definitions/index-definitions:apartment-package-definition}}
Apartment Packages are used to group Apartments together to aggregate a load.


\begin{savenotes}\sphinxattablestart
\centering
\begin{tabulary}{\linewidth}[t]{|T|T|T|}
\hline
\sphinxstyletheadfamily 
\sphinxstylestrong{Property}
&\sphinxstyletheadfamily 
\sphinxstylestrong{Data Type}
&\sphinxstyletheadfamily 
\sphinxstylestrong{Definition}
\\
\hline
Name
&
string
&
The name of the Apartment Package.
\\
\hline
Voltage
&
int
&
The voltage of the Apartment Package.
\\
\hline
Poles
&
int
&
The number of hot conductors for the Apartment Package.
\\
\hline
Optional-Calc
&
double
&
A calculation method determined by the NEC for calculating residential loads.
\\
\hline
Standard-Calc
&
double
&
A calculation method determined by the NEC for calculating residential loads.
\\
\hline
Multi. Dwell-Calc
&
double
&
A calculation method determined by the NEC for calculating residential loads.
\\
\hline
\end{tabulary}
\par
\sphinxattableend\end{savenotes}


\subsubsection{Load Allocation}
\label{\detokenize{docs/definitions/index-definitions:load-allocation}}\label{\detokenize{docs/definitions/index-definitions:load-allocation-definition}}
A Load Allocation is used to supplement Architectural Packages.  A designer can gather design parameters in order to determine a load.


\begin{savenotes}\sphinxattablestart
\centering
\begin{tabulary}{\linewidth}[t]{|T|T|T|}
\hline
\sphinxstyletheadfamily 
\sphinxstylestrong{Property}
&\sphinxstyletheadfamily 
\sphinxstylestrong{Data Type}
&\sphinxstyletheadfamily 
\sphinxstylestrong{Definition}
\\
\hline
Name
&
string
&
The name of the Load Allocation.
\\
\hline
Per Unit Size
&
int
&
The equipment size of each unit.
\\
\hline
Unit Type
&
int
&
The load unit of each unit.
\\
\hline
QTY
&
int
&
The quantity of units per Floor.
\\
\hline
Per Floor
&
boolean
&
A boolean property which represents the load for each Floor, or based on a certain number of Floors.
\\
\hline
Floor Space Type Filter
&
int
&
A filter which can be used to apply a load for a specific Space Type.
\\
\hline
Net Load
&
double
&
The total load as determined by the total units for each applied Floor.
\\
\hline
\end{tabulary}
\par
\sphinxattableend\end{savenotes}


\subsubsection{Diversification}
\label{\detokenize{docs/definitions/index-definitions:diversification}}\label{\detokenize{docs/definitions/index-definitions:diversification-definition}}
Diversification allows the designer to create custom load diversities that can be applied to specific sections of the network.


\begin{savenotes}\sphinxattablestart
\centering
\begin{tabulary}{\linewidth}[t]{|T|T|T|}
\hline
\sphinxstyletheadfamily 
\sphinxstylestrong{Property}
&\sphinxstyletheadfamily 
\sphinxstylestrong{Data Type}
&\sphinxstyletheadfamily 
\sphinxstylestrong{Definition}
\\
\hline
Name
&
string
&
The name of the Custom Diversity Class.
\\
\hline
Root Div.
&
double
&
The diversity factor applied to Root Level loads.
\\
\hline
Dist. Div.
&
double
&
The diversity factor applied to Distribution Level loads.
\\
\hline
EOL Div.
&
double
&
The diversity factor applied to End-of-Line Level loads.
\\
\hline
\end{tabulary}
\par
\sphinxattableend\end{savenotes}


\subsection{Equipment Definitions}
\label{\detokenize{docs/definitions/index-definitions:equipment-definitions}}\label{\detokenize{docs/definitions/index-definitions:id3}}

\subsubsection{General Properties}
\label{\detokenize{docs/definitions/index-definitions:general-properties}}\label{\detokenize{docs/definitions/index-definitions:general-properties-definitions}}

\begin{savenotes}\sphinxattablestart
\centering
\begin{tabulary}{\linewidth}[t]{|T|T|T|}
\hline
\sphinxstyletheadfamily 
\sphinxstylestrong{Property}
&\sphinxstyletheadfamily 
\sphinxstylestrong{Data Type}
&\sphinxstyletheadfamily 
\sphinxstylestrong{Definition}
\\
\hline
Name
&
string
&
The name of the Equipment.
\\
\hline
Room
&
Room
&
The Room location of the Equipment.
\\
\hline
Priority
&
int
&
The priority of the Equipment (N, EO1, EO2, LS).
\\
\hline
Is Existing
&
boolean
&
The name of the Equipment.
\\
\hline
Is Future
&
boolean
&
The name of the Equipment.
\\
\hline
Load Group
&
string
&
A custom name which a designer can use to group loads together.
\\
\hline
Tags
&
string
&
A custom set of strings which can be used to group equipment or loads.
\\
\hline
\end{tabulary}
\par
\sphinxattableend\end{savenotes}


\subsubsection{Load Properties}
\label{\detokenize{docs/definitions/index-definitions:load-properties}}\label{\detokenize{docs/definitions/index-definitions:load-properties-definitions}}

\begin{savenotes}\sphinxattablestart
\centering
\begin{tabulary}{\linewidth}[t]{|T|T|T|}
\hline

\sphinxstylestrong{Property}
&
\sphinxstylestrong{Data Type}
&
\sphinxstylestrong{Definition}
\\
\hline
Equipment Type
&
string
&
Denotes the type of Equipment.
\\
\hline
Load Capacity
&
double
&
The max capacity this Equipment will supply.
\\
\hline
Load Override
&
double
&
A custom value which ignores all downstream loads.
\\
\hline
Voltage
&
int
&
The voltage of the Equipment.
\\
\hline
Poles
&
int
&
The number of hot conductors for the Equipment.
\\
\hline
Power Factor
&
double
&
The ratio of real power absorbed by the load to the apparent power flowing in the circuit.
\\
\hline
Power Type
&
string
&
Denotes either a source of distribution of power or a load.
\\
\hline
\end{tabulary}
\par
\sphinxattableend\end{savenotes}


\subsubsection{Bus Structure Properties}
\label{\detokenize{docs/definitions/index-definitions:bus-structure-properties}}\label{\detokenize{docs/definitions/index-definitions:bus-structure-properties-definitions}}

\begin{savenotes}\sphinxattablestart
\centering
\begin{tabulary}{\linewidth}[t]{|T|T|T|}
\hline

\sphinxstylestrong{Property}
&
\sphinxstylestrong{Data Type}
&
\sphinxstylestrong{Definition}
\\
\hline
MCB
&
boolean
&
Specifies a main circuit breaker.
\\
\hline
Neutral Bus
&
boolean
&
A dedicated bus for neutral conductors.
\\
\hline
200\% Neutral Bus
&
boolean
&
An oversized neutral conductor to be 200\% the size of the phase conductor.
\\
\hline
Ground Bus
&
boolean
&
Specifies a dedicated ground bus.
\\
\hline
Isolated Ground Bus
&
boolean
&
An isolated ground bus providing a separate ground path.
\\
\hline
Top-Fed
&
boolean
&
Specify if the Equipment is constructed via top fed installation.
\\
\hline
Frame/Switch Size
&
int
&
Specifies the breaker frame size or switch size for the OCPD.
\\
\hline
Trip/Fuse Size
&
int
&
Specifies the breaker trip size or fuse trip size for the OCPD.
\\
\hline
kAIC Rating
&
string
&
Specifies the kAIC rating of the equipment.
\\
\hline
\end{tabulary}
\par
\sphinxattableend\end{savenotes}


\subsubsection{Construction Properties}
\label{\detokenize{docs/definitions/index-definitions:construction-properties}}\label{\detokenize{docs/definitions/index-definitions:construction-properties-definitions}}

\begin{savenotes}\sphinxattablestart
\centering
\begin{tabulary}{\linewidth}[t]{|T|T|T|}
\hline

\sphinxstylestrong{Property}
&
\sphinxstylestrong{Data Type}
&
\sphinxstylestrong{Definition}
\\
\hline
Top-Fed
&
boolean
&
Specify if the Equipment is constructed via top fed installation.
\\
\hline
Flush-Mount
&
boolean
&
Specifies a flush mounted construction.
\\
\hline
Enclosure
&
int
&
Specifies the type of NEMA enclosure.
\\
\hline
\end{tabulary}
\par
\sphinxattableend\end{savenotes}


\subsubsection{Accessories Properties}
\label{\detokenize{docs/definitions/index-definitions:accessories-properties}}\label{\detokenize{docs/definitions/index-definitions:accessories-properties-definitions}}

\begin{savenotes}\sphinxattablestart
\centering
\begin{tabulary}{\linewidth}[t]{|T|T|T|}
\hline

\sphinxstylestrong{Property}
&
\sphinxstylestrong{Data Type}
&
\sphinxstylestrong{Definition}
\\
\hline
Voltmeter
&
boolean
&
Specifies a voltmeter as an accessory.
\\
\hline
SPD
&
boolean
&
Specifies a surge protection device as an accessory.
\\
\hline
Power Meter
&
boolean
&
Specifies a power meter as an accessory.
\\
\hline
\end{tabulary}
\par
\sphinxattableend\end{savenotes}


\subsubsection{Load Readings Properties}
\label{\detokenize{docs/definitions/index-definitions:load-readings-properties}}\label{\detokenize{docs/definitions/index-definitions:load-readings-properties-definitions}}

\begin{savenotes}\sphinxattablestart
\centering
\begin{tabulary}{\linewidth}[t]{|T|T|T|}
\hline

\sphinxstylestrong{Property}
&
\sphinxstylestrong{Data Type}
&
\sphinxstylestrong{Definition}
\\
\hline
Net Load
&
boolean
&
Specifies a voltmeter as an accessory.
\\
\hline
Max Capacity
&
string
&
Denotes the maximum capacity determined by 80\% of the Load Capacity.  Available units are kVA, kW, and A.
\\
\hline
Custom Diversity Class
&
Diversification Class
&
Denotes the diversity class to apply the corresponding demand factor.
\\
\hline
Diversity Position
&
int
&
Denotes the level in the hierarchy of the distribution system to apply a corresponding demand factor.
\\
\hline
Connected Load
&
double
&
Net customer diversified connected load + net elevator load.
\\
\hline
Code Demand Load
&
double
&
Net custom diversified diversified load + net diversified elevator load + net diversified residential load.
\\
\hline
Diversified Resi. Load
&
boolean
&
The total load of the residential equipment based on the NEC.
\\
\hline
\% Design Spare Capacity
&
double
&
An adjustment factor based on the Code Demand Load.
\\
\hline
Net Design Spare Capacity
&
double
&
The Net Load multiplied by the \% Design Spare Capacity Factor.
\\
\hline
\% Loaded
&
double
&
The Net Load divided by the Max Capacity.
\\
\hline
\end{tabulary}
\par
\sphinxattableend\end{savenotes}


\subsubsection{Cable Properties}
\label{\detokenize{docs/definitions/index-definitions:cable-properties}}\label{\detokenize{docs/definitions/index-definitions:id4}}

\begin{savenotes}\sphinxattablestart
\centering
\begin{tabulary}{\linewidth}[t]{|T|T|T|}
\hline

\sphinxstylestrong{Property}
&
\sphinxstylestrong{Data Type}
&
\sphinxstylestrong{Definition}
\\
\hline
\# Sets
&
int
&
Specifies the number of sets of the wire arrangement.
\\
\hline
\# Phase Legs
&
int
&
Specifies the number of hot conductors per raceway.
\\
\hline
Phase Size
&
string
&
Specifies the size of the phase conductors.
\\
\hline
\# Neutrals
&
int
&
Specifies the number of neutral conductors per raceway.  A neutral conductor is required for equipment with a neutral bus.
\\
\hline
Neutral Size
&
string
&
Specifies the size of the neutral conductor(s) per raceway.  A neutral conductor is required for equipment with a neutral bus.
\\
\hline
\# Grounds
&
int
&
Specifies the number of ground conductors per raceyway.  A ground conductor is required for equipment with a ground bus.
\\
\hline
Ground Size
&
string
&
Specifies the size of the ground conductor.
\\
\hline
Insulation Type
&
int
&
Specifies insulation type for the wire arragement.
\\
\hline
Conductor Material
&
int
&
Specifies the conductor material for the wire arrangement.  The default conductor is copper.
\\
\hline
Temperature Rating
&
int
&
Specifies the temperature rating of conductors.
\\
\hline
Non-Linear
&
boolean
&
Specifies if a load is non-linear.  Non-linear loads include, but are not limited to: UPS, VFD, Power Supplies, etc.
\\
\hline
\end{tabulary}
\par
\sphinxattableend\end{savenotes}


\subsubsection{Conduit Properties}
\label{\detokenize{docs/definitions/index-definitions:conduit-properties}}\label{\detokenize{docs/definitions/index-definitions:id5}}

\begin{savenotes}\sphinxattablestart
\centering
\begin{tabulary}{\linewidth}[t]{|T|T|T|}
\hline

\sphinxstylestrong{Property}
&
\sphinxstylestrong{Data Type}
&
\sphinxstylestrong{Definition}
\\
\hline
Raceway Type
&
string
&
Specifies the type of raceway used to cover the wire arrangement.
\\
\hline
Raceway Size
&
string
&
Specifies the size of the raceway used to contain the wire arrangement.
\\
\hline
\end{tabulary}
\par
\sphinxattableend\end{savenotes}


\subsubsection{Network Reading Properties}
\label{\detokenize{docs/definitions/index-definitions:network-reading-properties}}\label{\detokenize{docs/definitions/index-definitions:id6}}

\begin{savenotes}\sphinxattablestart
\centering
\begin{tabulary}{\linewidth}[t]{|T|T|T|}
\hline

\sphinxstylestrong{Property}
&
\sphinxstylestrong{Data Type}
&
\sphinxstylestrong{Definition}
\\
\hline
Riser
&
Riser
&
Specifies if a circuit is being routed through a Riser or risershaft.
\\
\hline
Manual Added Length
&
int
&
Specifies a custom length which is supplemental to the Calc. Length.
\\
\hline
Calc. Length.
&
int
&
The length determined by an orthogonal route between two Room locations.
\\
\hline
Net Length
&
Net Length
&
The sum of the Calc. Length and Manual Added Length.
\\
\hline
VD\% Incoming
&
VD\% Incoming
&
The percentage of total volts dropped at the upstream node in the distribution system.
\\
\hline
VD\% Run
&
VD\% Run
&
The percentage of total volts dropped across the run.
\\
\hline
VD\% Net
&
VD\% Net
&
The sum of VD\% Incoming and VD\% Run.
\\
\hline
Net Amperage
&
double
&
The Net Load expressed in amps.
\\
\hline
SCC Incoming
&
double
&
The available fault current at the upstream node in the distribution system.
\\
\hline
SCC Net
&
double
&
The total available fault current.
\\
\hline
\end{tabulary}
\par
\sphinxattableend\end{savenotes}


\subsubsection{Mechanical}
\label{\detokenize{docs/definitions/index-definitions:mechanical-definition}}\label{\detokenize{docs/definitions/index-definitions:id7}}

\begin{savenotes}\sphinxattablestart
\centering
\begin{tabulary}{\linewidth}[t]{|T|T|T|}
\hline
\sphinxstyletheadfamily 
\sphinxstylestrong{Property}
&\sphinxstyletheadfamily 
\sphinxstylestrong{Data Type}
&\sphinxstyletheadfamily 
\sphinxstylestrong{Definition}
\\
\hline
Name
&
string
&
The name of the Equipment.
\\
\hline
Room
&
Room
&
The Room location of the Equipment.
\\
\hline
Priority
&
int
&
The priority of the Equipment (N, EO1, EO2, LS).
\\
\hline
Is Existing
&
boolean
&
The name of the Equipment.
\\
\hline
Is Future
&
boolean
&
The name of the Equipment.
\\
\hline
Load Capacity
&
double
&
The max capacity this Equipment will supply.
\\
\hline
Load Override
&
double
&
A custom value which ignores all downstream loads.
\\
\hline
Voltage
&
int
&
The voltage of the Equipment.
\\
\hline
Poles
&
int
&
The number of hot conductors for the Equipment.
\\
\hline
Power Factor
&
double
&
The ratio of real power absorbed by the load to the apparent power flowing in the circuit.
\\
\hline
Load Group
&
string
&
A custom name which a designer can use to group loads together.
\\
\hline
Tags
&
string
&
A custom set of strings which can be used to group equipment or loads.
\\
\hline
VFD
&
boolean
&
Specifies a variable frequency drive for the motor.
\\
\hline
\# Motors
&
int
&
Represents the quantity of motors as part of the motor collection.
\\
\hline
\end{tabulary}
\par
\sphinxattableend\end{savenotes}


\subsubsection{Distribution Board}
\label{\detokenize{docs/definitions/index-definitions:distribution-board-definition}}\label{\detokenize{docs/definitions/index-definitions:id8}}

\begin{savenotes}\sphinxattablestart
\centering
\begin{tabulary}{\linewidth}[t]{|T|T|T|}
\hline
\sphinxstyletheadfamily 
\sphinxstylestrong{Property}
&\sphinxstyletheadfamily 
\sphinxstylestrong{Data Type}
&\sphinxstyletheadfamily 
\sphinxstylestrong{Definition}
\\
\hline
Name
&
string
&
The name of the Equipment.
\\
\hline
Room
&
Room
&
The Room location of the Equipment.
\\
\hline
Priority
&
int
&
The priority of the Equipment (N, EO1, EO2, LS).
\\
\hline
Is Existing
&
boolean
&
The name of the Equipment.
\\
\hline
Is Future
&
boolean
&
The name of the Equipment.
\\
\hline
Load Capacity
&
double
&
The max capacity this Equipment will supply.
\\
\hline
Load Override
&
double
&
A custom value which ignores all downstream loads.
\\
\hline
Voltage
&
int
&
The voltage of the Equipment.
\\
\hline
Poles
&
int
&
The number of hot conductors for the Equipment.
\\
\hline
Power Factor
&
double
&
The ratio of real power absorbed by the load to the apparent power flowing in the circuit.
\\
\hline
Load Group
&
string
&
A custom name which a designer can use to group loads together.
\\
\hline
Tags
&
string
&
A custom set of strings which can be used to group equipment or loads.
\\
\hline
MCB
&
boolean
&
Specifies a main circuit breaker.
\\
\hline
Neutral Bus
&
boolean
&
A dedicated bus for neutral conductors.
\\
\hline
200\% Neutral Bus
&
boolean
&
An oversized neutral conductor to be 200\% the size of the phase conductor.
\\
\hline
Ground Bus
&
boolean
&
Specifies a dedicated ground bus.
\\
\hline
Isolated Ground Bus
&
boolean
&
An isolated ground bus providing a separate ground path.
\\
\hline
Top-Fed
&
boolean
&
Specify if the Equipment is constructed via top fed installation.
\\
\hline
Frame/Switch Size
&
int
&
Specifies the breaker frame size or switch size for the OCPD.
\\
\hline
Trip/Fuse Size
&
int
&
Specifies the breaker trip size or fuse trip size for the OCPD.
\\
\hline
kAIC Rating
&
string
&
Specifies the kAIC rating of the equipment.
\\
\hline
\end{tabulary}
\par
\sphinxattableend\end{savenotes}


\subsubsection{XFMR}
\label{\detokenize{docs/definitions/index-definitions:id9}}

\begin{savenotes}\sphinxattablestart
\centering
\begin{tabulary}{\linewidth}[t]{|T|T|T|}
\hline
\sphinxstyletheadfamily 
\sphinxstylestrong{Property}
&\sphinxstyletheadfamily 
\sphinxstylestrong{Data Type}
&\sphinxstyletheadfamily 
\sphinxstylestrong{Definition}
\\
\hline
Name
&
string
&
The name of the Equipment.
\\
\hline
Room
&
Room
&
The Room location of the Equipment.
\\
\hline
Priority
&
int
&
The priority of the Equipment (N, EO1, EO2, LS).
\\
\hline
Is Existing
&
boolean
&
The name of the Equipment.
\\
\hline
Is Future
&
boolean
&
The name of the Equipment.
\\
\hline
Load Capacity
&
double
&
The max capacity this Equipment will supply.
\\
\hline
Load Override
&
double
&
A custom value which ignores all downstream loads.
\\
\hline
Voltage
&
int
&
The voltage of the Equipment.
\\
\hline
Poles
&
int
&
The number of hot conductors for the Equipment.
\\
\hline
Power Factor
&
double
&
The ratio of real power absorbed by the load to the apparent power flowing in the circuit.
\\
\hline
Load Group
&
string
&
A custom name which a designer can use to group loads together.
\\
\hline
Tags
&
string
&
A custom set of strings which can be used to group equipment or loads.
\\
\hline
Secondary Voltage
&
int
&
The voltage across the secondary of the transformer.
\\
\hline
Size
&
double
&
Specify the load size of the transformer.
\\
\hline
Primary Winding
&
int
&
The primary winding configuration for a transformer.
\\
\hline
Secondary Winding
&
int
&
The secondary winding configuration for a transformer.
\\
\hline
Primary Voltage
&
int
&
The voltage applied to the terminals of the primary windings of a transformer.
\\
\hline
Secondary Voltage
&
int
&
The voltage applied to the terminals of the secondary windings of a transformer.
\\
\hline
Overload Rating
&
double
&
The overload rating of a transformer.
\\
\hline
Percent Impedance
&
double
&
The voltage drop on full load due to winding resistance and leakage reactance as a percentage of the rated voltage.
\\
\hline
K Rating
&
string
&
The index of a transformer’s ability to withstand harmonic content while operating within the temperature limits of its insulating system.
\\
\hline
XFMR Type
&
int
&
Specifies the type of a transformer to be dry, cast coil, vacuum pressure impregnated (VPI), or liquid.
\\
\hline
Temperature Rise
&
int
&
The average temperature rise that will occur in the coils during normal full load.
\\
\hline
Grounding Type
&
int
&
The diversity class applied to a section of the distribution network.
\\
\hline
Insulation Rating
&
int
&
The insulation class or rating for a transformer at its maximum temperature rating.
\\
\hline
Primary BIL
&
int
&
The BIL (Basic Impulse Level) is a method expressing the voltage surge that a transformer will tolerate without breakdown.
\\
\hline
Secondary BIL
&
int
&
The BIL (Basic Impulse Level) is a method expressing the voltage surge that a transformer will tolerate without breakdown.
\\
\hline
\end{tabulary}
\par
\sphinxattableend\end{savenotes}


\subsubsection{Architectural Load}
\label{\detokenize{docs/definitions/index-definitions:architectural-load}}\label{\detokenize{docs/definitions/index-definitions:architectural-load-definition}}

\begin{savenotes}\sphinxattablestart
\centering
\begin{tabulary}{\linewidth}[t]{|T|T|T|}
\hline
\sphinxstyletheadfamily 
\sphinxstylestrong{Property}
&\sphinxstyletheadfamily 
\sphinxstylestrong{Data Type}
&\sphinxstyletheadfamily 
\sphinxstylestrong{Definition}
\\
\hline
Name
&
string
&
The name of the Equipment.
\\
\hline
Room
&
Room
&
The Room location of the Equipment.
\\
\hline
Priority
&
int
&
The priority of the Equipment (N, EO1, EO2, LS).
\\
\hline
Is Existing
&
boolean
&
The name of the Equipment.
\\
\hline
Is Future
&
boolean
&
The name of the Equipment.
\\
\hline
Load Capacity
&
double
&
The max capacity this Equipment will supply.
\\
\hline
Load Override
&
double
&
A custom value which ignores all downstream loads.
\\
\hline
Voltage
&
int
&
The voltage of the Equipment.
\\
\hline
Poles
&
int
&
The number of hot conductors for the Equipment.
\\
\hline
Power Factor
&
double
&
The ratio of real power absorbed by the load to the apparent power flowing in the circuit.
\\
\hline
Package
&
ICircuitablePackage
&
Select the group of Architectural entities which compose the Package
\\
\hline
\end{tabulary}
\par
\sphinxattableend\end{savenotes}


\subsubsection{Fire Pump}
\label{\detokenize{docs/definitions/index-definitions:fire-pump}}\label{\detokenize{docs/definitions/index-definitions:fire-pump-definition}}

\begin{savenotes}\sphinxattablestart
\centering
\begin{tabulary}{\linewidth}[t]{|T|T|T|}
\hline
\sphinxstyletheadfamily 
\sphinxstylestrong{Property}
&\sphinxstyletheadfamily 
\sphinxstylestrong{Data Type}
&\sphinxstyletheadfamily 
\sphinxstylestrong{Definition}
\\
\hline
Name
&
string
&
The name of the Equipment.
\\
\hline
Room
&
Room
&
The Room location of the Equipment.
\\
\hline
Priority
&
int
&
The priority of the Equipment (N, EO1, EO2, LS).
\\
\hline
Is Existing
&
boolean
&
The name of the Equipment.
\\
\hline
Is Future
&
boolean
&
The name of the Equipment.
\\
\hline
Load Capacity
&
double
&
The max capacity this Equipment will supply.
\\
\hline
Load Override
&
double
&
A custom value which ignores all downstream loads.
\\
\hline
Voltage
&
int
&
The voltage of the Equipment.
\\
\hline
Poles
&
int
&
The number of hot conductors for the Equipment.
\\
\hline
Power Factor
&
double
&
The ratio of real power absorbed by the load to the apparent power flowing in the circuit.
\\
\hline
Load Group
&
string
&
A custom name which a designer can use to group loads together.
\\
\hline
Tags
&
string
&
A custom set of strings which can be used to group equipment or loads.
\\
\hline
\end{tabulary}
\par
\sphinxattableend\end{savenotes}


\subsubsection{Panelboard}
\label{\detokenize{docs/definitions/index-definitions:panelboard-definition}}\label{\detokenize{docs/definitions/index-definitions:id10}}

\begin{savenotes}\sphinxattablestart
\centering
\begin{tabulary}{\linewidth}[t]{|T|T|T|}
\hline
\sphinxstyletheadfamily 
\sphinxstylestrong{Property}
&\sphinxstyletheadfamily 
\sphinxstylestrong{Data Type}
&\sphinxstyletheadfamily 
\sphinxstylestrong{Definition}
\\
\hline
Name
&
string
&
The name of the Equipment.
\\
\hline
Room
&
Room
&
The Room location of the Equipment.
\\
\hline
Priority
&
int
&
The priority of the Equipment (N, EO1, EO2, LS).
\\
\hline
Is Existing
&
boolean
&
The name of the Equipment.
\\
\hline
Is Future
&
boolean
&
The name of the Equipment.
\\
\hline
Load Capacity
&
double
&
The max capacity this Equipment will supply.
\\
\hline
Load Override
&
double
&
A custom value which ignores all downstream loads.
\\
\hline
Voltage
&
int
&
The voltage of the Equipment.
\\
\hline
Poles
&
int
&
The number of hot conductors for the Equipment.
\\
\hline
Power Factor
&
double
&
The ratio of real power absorbed by the load to the apparent power flowing in the circuit.
\\
\hline
Load Group
&
string
&
A custom name which a designer can use to group loads together.
\\
\hline
Tags
&
string
&
A custom set of strings which can be used to group equipment or loads.
\\
\hline
MCB
&
boolean
&
Specifies a main circuit breaker.
\\
\hline
Neutral Bus
&
boolean
&
A dedicated bus for neutral conductors.
\\
\hline
200\% Neutral Bus
&
boolean
&
An oversized neutral conductor to be 200\% the size of the phase conductor.
\\
\hline
Ground Bus
&
boolean
&
Specifies a dedicated ground bus.
\\
\hline
Isolated Ground Bus
&
boolean
&
An isolated ground bus providing a separate ground path.
\\
\hline
Top-Fed
&
boolean
&
Specify if the Equipment is constructed via top fed installation.
\\
\hline
Bolt-On Breaker
&
boolean
&
Specifies a bolt-on breaker.
\\
\hline
Flush-Mount
&
boolean
&
Specifies a flush mounted construction.
\\
\hline
Enclosure
&
int
&
Specifies the type of NEMA enclosure.
\\
\hline
Door-in-Door
&
boolean
&
Specifies a door-in-door construction vs. hinged trim
\\
\hline
Stainless Steel
&
boolean
&
Specifies a stainless steel construction.
\\
\hline
Feed-Through
&
boolean
&
Specifies a feed-through installation.
\\
\hline
Voltmeter
&
boolean
&
Specifies a voltmeter as an accessory.
\\
\hline
SPD
&
boolean
&
Specifies a surge protection device as an accessory.
\\
\hline
Power Meter
&
boolean
&
Specifies a power meter as an accessory.
\\
\hline
Branch Circuit Meter
&
boolean
&
Specifies a branch circuit meter as an accessory.
\\
\hline
Split Bus Meter
&
boolean
&
Specifies a split bus meter as an accessory.
\\
\hline
\end{tabulary}
\par
\sphinxattableend\end{savenotes}


\subsubsection{ATS/STS}
\label{\detokenize{docs/definitions/index-definitions:ats-sts}}\label{\detokenize{docs/definitions/index-definitions:ats-sts-definition}}

\begin{savenotes}\sphinxattablestart
\centering
\begin{tabulary}{\linewidth}[t]{|T|T|T|}
\hline
\sphinxstyletheadfamily 
\sphinxstylestrong{Property}
&\sphinxstyletheadfamily 
\sphinxstylestrong{Data Type}
&\sphinxstyletheadfamily 
\sphinxstylestrong{Definition}
\\
\hline
Name
&
string
&
The name of the Equipment.
\\
\hline
Room
&
Room
&
The Room location of the Equipment.
\\
\hline
Priority
&
int
&
The priority of the Equipment (N, EO1, EO2, LS).
\\
\hline
Is Existing
&
boolean
&
The name of the Equipment.
\\
\hline
Is Future
&
boolean
&
The name of the Equipment.
\\
\hline
Load Capacity
&
double
&
The max capacity this Equipment will supply.
\\
\hline
Load Override
&
double
&
A custom value which ignores all downstream loads.
\\
\hline
Voltage
&
int
&
The voltage of the Equipment.
\\
\hline
Poles
&
int
&
The number of hot conductors for the Equipment.
\\
\hline
Power Factor
&
double
&
The ratio of real power absorbed by the load to the apparent power flowing in the circuit.
\\
\hline
Load Group
&
string
&
A custom name which a designer can use to group loads together.
\\
\hline
Tags
&
string
&
A custom set of strings which can be used to group equipment or loads.
\\
\hline
Bypass Isolation
&
boolean
&
Specifying an ATS to be include a bypass isolation switch.
\\
\hline
Neutral Construction
&
int
&
Specifies the ATS neutral construction to be solid, switched, or overlapping.
\\
\hline
Neutral Bus
&
boolean
&
A dedicated bus for neutral conductors.
\\
\hline
200\% Neutral Bus
&
boolean
&
An oversized neutral conductor to be 200\% the size of the phase conductor.
\\
\hline
Ground Bus
&
boolean
&
Specifies a dedicated ground bus.
\\
\hline
Isolated Ground Bus
&
boolean
&
An isolated ground bus providing a separate ground path.
\\
\hline
Static
&
boolean
&
Specifies a static transfer switch.
\\
\hline
Flush-Mount
&
boolean
&
Specifies a flush mounted construction.
\\
\hline
Top-Fed
&
boolean
&
Specify if the Equipment is constructed via top fed installation.
\\
\hline
Enclosure
&
int
&
Specifies the type of NEMA enclosure.
\\
\hline
\end{tabulary}
\par
\sphinxattableend\end{savenotes}


\subsubsection{Bus Duct}
\label{\detokenize{docs/definitions/index-definitions:bus-duct-definition}}\label{\detokenize{docs/definitions/index-definitions:id11}}

\begin{savenotes}\sphinxattablestart
\centering
\begin{tabulary}{\linewidth}[t]{|T|T|T|}
\hline
\sphinxstyletheadfamily 
\sphinxstylestrong{Property}
&\sphinxstyletheadfamily 
\sphinxstylestrong{Data Type}
&\sphinxstyletheadfamily 
\sphinxstylestrong{Definition}
\\
\hline
Name
&
string
&
The name of the Equipment.
\\
\hline
Room
&
Room
&
The Room location of the Equipment.
\\
\hline
Priority
&
int
&
The priority of the Equipment (N, EO1, EO2, LS).
\\
\hline
Is Existing
&
boolean
&
The name of the Equipment.
\\
\hline
Is Future
&
boolean
&
The name of the Equipment.
\\
\hline
Load Capacity
&
double
&
The max capacity this Equipment will supply.
\\
\hline
Load Override
&
double
&
A custom value which ignores all downstream loads.
\\
\hline
Voltage
&
int
&
The voltage of the Equipment.
\\
\hline
Poles
&
int
&
The number of hot conductors for the Equipment.
\\
\hline
Power Factor
&
double
&
The ratio of real power absorbed by the load to the apparent power flowing in the circuit.
\\
\hline
Load Group
&
string
&
A custom name which a designer can use to group loads together.
\\
\hline
Tags
&
string
&
A custom set of strings which can be used to group equipment or loads.
\\
\hline
Conductor Type
&
int
&
Specify the default conductor material.
\\
\hline
Neutral Bus
&
boolean
&
A dedicated bus for neutral conductors.
\\
\hline
200\% Neutral Bus
&
boolean
&
An oversized neutral conductor to be 200\% the size of the phase conductor.
\\
\hline
Ground Bus
&
boolean
&
Specifies a dedicated ground bus.
\\
\hline
Isolated Ground Bus
&
boolean
&
An isolated ground bus providing a separate ground path.
\\
\hline
Flush-Mount
&
boolean
&
Specifies a flush mounted construction.
\\
\hline
Top-Fed
&
boolean
&
Specify if the Equipment is constructed via top fed installation.
\\
\hline
Enclosure
&
int
&
Specifies the type of NEMA enclosure.
\\
\hline
\end{tabulary}
\par
\sphinxattableend\end{savenotes}


\subsubsection{Switch Board}
\label{\detokenize{docs/definitions/index-definitions:switch-board}}\label{\detokenize{docs/definitions/index-definitions:switch-board-definition}}

\begin{savenotes}\sphinxattablestart
\centering
\begin{tabulary}{\linewidth}[t]{|T|T|T|}
\hline
\sphinxstyletheadfamily 
\sphinxstylestrong{Property}
&\sphinxstyletheadfamily 
\sphinxstylestrong{Data Type}
&\sphinxstyletheadfamily 
\sphinxstylestrong{Definition}
\\
\hline
Name
&
string
&
The name of the Equipment.
\\
\hline
Room
&
Room
&
The Room location of the Equipment.
\\
\hline
Priority
&
int
&
The priority of the Equipment (N, EO1, EO2, LS).
\\
\hline
Is Existing
&
boolean
&
The name of the Equipment.
\\
\hline
Is Future
&
boolean
&
The name of the Equipment.
\\
\hline
Load Capacity
&
double
&
The max capacity this Equipment will supply.
\\
\hline
Load Override
&
double
&
A custom value which ignores all downstream loads.
\\
\hline
Voltage
&
int
&
The voltage of the Equipment.
\\
\hline
Poles
&
int
&
The number of hot conductors for the Equipment.
\\
\hline
Power Factor
&
double
&
The ratio of real power absorbed by the load to the apparent power flowing in the circuit.
\\
\hline
Load Group
&
string
&
A custom name which a designer can use to group loads together.
\\
\hline
Tags
&
string
&
A custom set of strings which can be used to group equipment or loads.
\\
\hline
Conductor Type
&
int
&
Specify the default conductor material.
\\
\hline
Neutral Bus
&
boolean
&
A dedicated bus for neutral conductors.
\\
\hline
200\% Neutral Bus
&
boolean
&
An oversized neutral conductor to be 200\% the size of the phase conductor.
\\
\hline
Ground Bus
&
boolean
&
Specifies a dedicated ground bus.
\\
\hline
Isolated Ground Bus
&
boolean
&
An isolated ground bus providing a separate ground path.
\\
\hline
Flush-Mount
&
boolean
&
Specifies a flush mounted construction.
\\
\hline
Top-Fed
&
boolean
&
Specify if the Equipment is constructed via top fed installation.
\\
\hline
Voltmeter
&
boolean
&
Specifies a voltmeter as an accessory.
\\
\hline
SPD
&
boolean
&
Specifies a surge protection device as an accessory.
\\
\hline
Power Meter
&
boolean
&
Specifies a power meter as an accessory.
\\
\hline
\end{tabulary}
\par
\sphinxattableend\end{savenotes}


\subsubsection{Generic Load}
\label{\detokenize{docs/definitions/index-definitions:generic-load}}\label{\detokenize{docs/definitions/index-definitions:generic-load-definition}}

\begin{savenotes}\sphinxattablestart
\centering
\begin{tabulary}{\linewidth}[t]{|T|T|T|}
\hline
\sphinxstyletheadfamily 
\sphinxstylestrong{Property}
&\sphinxstyletheadfamily 
\sphinxstylestrong{Data Type}
&\sphinxstyletheadfamily 
\sphinxstylestrong{Definition}
\\
\hline
Name
&
string
&
The name of the Equipment.
\\
\hline
Room
&
Room
&
The Room location of the Equipment.
\\
\hline
Priority
&
int
&
The priority of the Equipment (N, EO1, EO2, LS).
\\
\hline
Is Existing
&
boolean
&
The name of the Equipment.
\\
\hline
Is Future
&
boolean
&
The name of the Equipment.
\\
\hline
Load Capacity
&
double
&
The max capacity this Equipment will supply.
\\
\hline
Load Override
&
double
&
A custom value which ignores all downstream loads.
\\
\hline
Voltage
&
int
&
The voltage of the Equipment.
\\
\hline
Poles
&
int
&
The number of hot conductors for the Equipment.
\\
\hline
Power Factor
&
double
&
The ratio of real power absorbed by the load to the apparent power flowing in the circuit.
\\
\hline
Load Group
&
string
&
A custom name which a designer can use to group loads together.
\\
\hline
Tags
&
string
&
A custom set of strings which can be used to group equipment or loads.
\\
\hline
\end{tabulary}
\par
\sphinxattableend\end{savenotes}


\subsubsection{Bus Node}
\label{\detokenize{docs/definitions/index-definitions:bus-node}}\label{\detokenize{docs/definitions/index-definitions:bus-node-definition}}

\begin{savenotes}\sphinxattablestart
\centering
\begin{tabulary}{\linewidth}[t]{|T|T|T|}
\hline
\sphinxstyletheadfamily 
\sphinxstylestrong{Property}
&\sphinxstyletheadfamily 
\sphinxstylestrong{Data Type}
&\sphinxstyletheadfamily 
\sphinxstylestrong{Definition}
\\
\hline
Name
&
string
&
The name of the Equipment.
\\
\hline
Room
&
Room
&
The Room location of the Equipment.
\\
\hline
Priority
&
int
&
The priority of the Equipment (N, EO1, EO2, LS).
\\
\hline
Is Existing
&
boolean
&
The name of the Equipment.
\\
\hline
Is Future
&
boolean
&
The name of the Equipment.
\\
\hline
Load Capacity
&
double
&
The max capacity this Equipment will supply.
\\
\hline
Load Override
&
double
&
A custom value which ignores all downstream loads.
\\
\hline
Voltage
&
int
&
The voltage of the Equipment.
\\
\hline
Poles
&
int
&
The number of hot conductors for the Equipment.
\\
\hline
Power Factor
&
double
&
The ratio of real power absorbed by the load to the apparent power flowing in the circuit.
\\
\hline
Load Group
&
string
&
A custom name which a designer can use to group loads together.
\\
\hline
Tags
&
string
&
A custom set of strings which can be used to group equipment or loads.
\\
\hline
\end{tabulary}
\par
\sphinxattableend\end{savenotes}


\subsubsection{Elevator}
\label{\detokenize{docs/definitions/index-definitions:elevator}}\label{\detokenize{docs/definitions/index-definitions:elevator-definition}}

\begin{savenotes}\sphinxattablestart
\centering
\begin{tabulary}{\linewidth}[t]{|T|T|T|}
\hline
\sphinxstyletheadfamily 
\sphinxstylestrong{Property}
&\sphinxstyletheadfamily 
\sphinxstylestrong{Data Type}
&\sphinxstyletheadfamily 
\sphinxstylestrong{Definition}
\\
\hline
Name
&
string
&
The name of the Equipment.
\\
\hline
Room
&
Room
&
The Room location of the Equipment.
\\
\hline
Priority
&
int
&
The priority of the Equipment (N, EO1, EO2, LS).
\\
\hline
Is Existing
&
boolean
&
The name of the Equipment.
\\
\hline
Is Future
&
boolean
&
The name of the Equipment.
\\
\hline
Load Capacity
&
double
&
The max capacity this Equipment will supply.
\\
\hline
Load Override
&
double
&
A custom value which ignores all downstream loads.
\\
\hline
Voltage
&
int
&
The voltage of the Equipment.
\\
\hline
Poles
&
int
&
The number of hot conductors for the Equipment.
\\
\hline
Power Factor
&
double
&
The ratio of real power absorbed by the load to the apparent power flowing in the circuit.
\\
\hline
Load Group
&
string
&
A custom name which a designer can use to group loads together.
\\
\hline
Tags
&
string
&
A custom set of strings which can be used to group equipment or loads.
\\
\hline
\# Motors
&
int
&
Represents the quantity of motors as part of the motor collection.
\\
\hline
\end{tabulary}
\par
\sphinxattableend\end{savenotes}


\subsubsection{Tap Node}
\label{\detokenize{docs/definitions/index-definitions:tap-node}}\label{\detokenize{docs/definitions/index-definitions:tap-node-definition}}

\begin{savenotes}\sphinxattablestart
\centering
\begin{tabulary}{\linewidth}[t]{|T|T|T|}
\hline
\sphinxstyletheadfamily 
\sphinxstylestrong{Property}
&\sphinxstyletheadfamily 
\sphinxstylestrong{Data Type}
&\sphinxstyletheadfamily 
\sphinxstylestrong{Definition}
\\
\hline
Name
&
string
&
The name of the Equipment.
\\
\hline
Room
&
Room
&
The Room location of the Equipment.
\\
\hline
Priority
&
int
&
The priority of the Equipment (N, EO1, EO2, LS).
\\
\hline
Is Existing
&
boolean
&
The name of the Equipment.
\\
\hline
Is Future
&
boolean
&
The name of the Equipment.
\\
\hline
Load Capacity
&
double
&
The max capacity this Equipment will supply.
\\
\hline
Load Override
&
double
&
A custom value which ignores all downstream loads.
\\
\hline
Voltage
&
int
&
The voltage of the Equipment.
\\
\hline
Poles
&
int
&
The number of hot conductors for the Equipment.
\\
\hline
Power Factor
&
double
&
The ratio of real power absorbed by the load to the apparent power flowing in the circuit.
\\
\hline
Load Group
&
string
&
A custom name which a designer can use to group loads together.
\\
\hline
Tags
&
string
&
A custom set of strings which can be used to group equipment or loads.
\\
\hline
\end{tabulary}
\par
\sphinxattableend\end{savenotes}

Use our \DUrole{xref,std,std-ref}{Index} or \DUrole{xref,std,std-ref}{Search} for further information.



\renewcommand{\indexname}{Index}
\printindex
\end{document}